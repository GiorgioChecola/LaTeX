\label{approach}
%\vspace{5cm}

This chapter describes the approach used to deal with weather forecasts and, in general, moving obstacle prediction. The configuration space and the obstacles of a real environment are defined, and the spatio-temporal approaches are presented.\\
The entire methodology is based on the NetCDF files described in the previous chapter. 
%Before moving on to the resuThen RRT*, A* and the potential field method are presented and compared in 3 basic scenarios to evaluate their performance. Then, time windows approaches are presented and tested in non-real scenarios. Finally, the method is described in the various steps to build a real environment scenarios and to perform path planning simulations.
\section{Description of the environmental modeling}
The map generation needed to perform real path planning simulations can be summarized in 3 steps:
\begin{enumerate}[itemsep=0pt]
	\item Definition of the configuration space $\mathcal{C}$;
	\item Embedding of environmental data;
	\item Development of time window approaches for path planning.
\end{enumerate}
\subsection{Configuration space}
Witted has the vision of monitoring the biodiversity of the Italian coasts with autonomous drones, with a strong interest in the protection of \textit{Posidonia oceanica}, aquatic plants protected internationally. Hence, we decided to focus on the Mediterranean sea.

For the choice of the simulation area, different factors have been considered. Starting from \textit{Posidonia oceanica}, we could examine a georeferenced dataset in QGIS about the presence of this plant along the Mediterranean coast. The other environmental factor was the requirement of clean and shallow water to facilitate underwater surveys, which was verified through an online search and analysis of sea bathymetry.
Regarding the dimension of the sea region, we considered the possibility to navigate it in a day and, above all, the necessity to take a large sea area to see changes in atmospheric factors. Finally, the presence of islands and coasts was considered for testing the algorithms in a more complex scenario.     

The choice fell on the area of the Tyrrhenian Sea, which includes the east coast of Corsica, the Argentario archipelago, the island of Elba, and the Tuscan coast. Figure \ref{mapspace} displays an image of the sea area on Google Maps, while Figure \ref{posidonia} the presence of \textit{Posidonia oceanica}, marked with the red color.

\begin{figure}[h]
	\centering 
	\subfloat[Simulation area from Google Maps]{\includegraphics[height=0.37\textwidth]{Figures/confspacemaps.png}\label{mapspace}}
	\hspace{0.3cm}
	\subfloat[Presence of \textit{Posidonia oceanica}]{\includegraphics[height=0.37\textwidth]{Figures/posidonia.png}\label{posidonia}}
	%\captionsetup[subfloat]{font=scriptsize,labelfont=scriptsize}
	\caption{Configuration space}
	\label{Configuration spac}
\end{figure} 

Using the default grid resolution of Copernicus NetCDF files, the area is represented by a grid of $62\times47$ points ($longitude\times latitude$), which is equal to just over $200\times200$ $km^2$.
% Sea factors are considered as  spatio-temporal obstacles that changes position every hour made by dangerous wave heights from NetCDF file.
% 41°35'41''N, 8°95'83''E and 43°27'08''N 11°50'00''E
%As described in detail in the previous chapter, NetCDF data are set of multidimensional array, and so points. Consequently, for semplicity  I took two point coordinates belonging to that set as extremes of the space:  (a matrix of $62\times47$ point coordinates). 
Figure \ref{startend} shows the configuration space in Python; maps are generated by the \textit{matplotlib basemap toolkit}, a library for plotting 2D data on maps in Python. As can be seen, three points are highlighted:
\begin{itemize}[itemsep=0pt]
	\item 1 starting point: the harbor of Porto Vecchio in the east Corsica
	\item 2 ending points: WWF Oasis of Orbetello in the Argentario and the island of Elba with the National Parc of the Tuscan Archipelago, chosen as survey areas for their naturalistic importance
\end{itemize}
%5 days were considered: 18-07-21, 23-08-21, 24-08-21, 10-09-21, 18-09-21 \textcolor{red}{search more days}
%Depending on the sea conditions, the vessel can decide which survey destination to chose at the beginning, or to decide to stay moored.
\begin{figure}[H]
	\centering
	\includegraphics[width=100mm]{Figures/conf_space2223.png}
	\caption{Configuration space with start and end points}
	\label{startend}
\end{figure} 
%Porto vecchio is a port in the east Corsica and has coordinates: 41°60'07''N 9°29'49''E. By analyzing wave height at this coordinate, the values are always smaller than the threshold.
\subsection{Obstacle definition}
In path planning, an obstacle is defined first of all by its position in space.\\
In this study, we have to distinguish between static and dynamic obstacles: the former is represented by islands and coasts, while the others by wave heights. Copernicus data are available in georeferenced coordinates, so their position will be specified in longitude and latitude.

\texttt{MED-MFC\_006\_017\_mask\_bathy.nc} and \texttt{med-hcmr-wav-an-fc-h.nc} are the two files used for this purpose. The first provides binary data of land and sea, useful to distinguish between navigable points from those not, but especially for reproducing the profile of the coasts. The second one includes wave forecast data and is exploited for representing dangerous waves.

The method used to define obstacles is the same for both cases and makes use of the function \texttt{binary\_erosion} of the module \texttt{morphology} of scikit-image, an image processing toolbox for SciPy.
%If the image is concave, the final profile will be the same.
%Ocean data are in georeferenced coordinates. Therefore, also the path planning will use this configuration. The discrete map, necessary for the simulations, is gridded with the default spatial resolution, $0.042$°$\times0.042$°, which corresponds approximately to $5$ $km$. 

%And so my resolution will be the default one given by the Copernicus dataset 0.042°x0.042° that  of movement that we can consider acceptable for my purpose. ($2\pi R_e \frac{\Delta lat}{360°}$). My configuration space becomes a gridded matrix of points. The grid size is approximated to 0.04166°, and so I need to assume robot radius at least equal to the resolution, in order to avoid that the boat can go between 2 points.
\subsubsection{Static obstacle}
From the mask of the landmass, we extracted the coordinates of the points of its profile and defined them as static obstacles. The only drawback of this method is that it neglects the islands with dimensions below the data resolution. 
%By knowing which points are sea and which land, I could extract the profile of the coastline, and model it as a static obstacle. 
\subsubsection{Dynamic obstacle}
Wave height is not always constant. Indeed the positions of the obstacles will vary according to the temporal resolution of NetCDF data, which is 1 hour.

By considering a vessel drone of small dimensions (max 3 m of length), and the categories of sea roughness seen in \autoref{second}, a wave height of $0.8$ $m$ has been assumed as the threshold above which the vessel can no longer handle the navigation. We obtained a binary multi-dimensional array $=1$ where the waves are high and $=0$ where they are lower than $0.8$. At that point, the function executes the erosion obtaining the contour of the unsafe zones. \\
The contour extraction was made to define obstacles and lower the computing time of the algorithm, which has to compute the distances from each obstacle. Analyzing an internal point of the unsafe area would be unnecessary since the vessel must not overcome its contour in any case. Figure \ref{staticdynamicobs} shows the obstacles obtained through this mask processing.
\begin{figure}[h]
	\centering 
	\subfloat[Italy mask]{\includegraphics[height=0.35\textwidth]{Figures/maskera.png}\label{mask}}
	\hspace{0.5cm}
	\subfloat[Land contour]{\includegraphics[height=0.35\textwidth]{Figures/staticobs.png}\label{staticcont}}
	\hspace{0.5cm}
	\subfloat[Unsafe area at hour $t_i$]{\includegraphics[height=0.35\textwidth]{Figures/dynobscost.png}\label{3ob}}
	\hspace{0.5cm}
	\subfloat[Unsafe area contour]{\includegraphics[height=0.35\textwidth]{Figures/dynobs.png}\label{4ob}}
	\vspace{0.0cm}
	\caption{Static and dynamic obstacle definition} 
	\label{staticdynamicobs}
\end{figure}
%Figure \ref{dynamicobs} displays the configuration space with obstacle points at four time windows. 
\subsection{Vessel model approximation}
In this study, the effect of vehicle dynamics will be neglected. For the scale of the path planning problem, the vessel is regarded as a point with a radius equal to the grid size  \cite{garau2009path, SONG2017301, xiong2020rapidly, zhou2020review}.
As a first approximation, the assumption is valid since the long travel distance reduces the necessity to study how the vehicle will follow the path. Besides, the dimension of the boat is much smaller than the dimension of the sea basin, and then the focus remains on path planning. 

Even if the dynamic model is not integrated, its kinematics is needed to deal with dynamic obstacles \cite{SONG2017301}. We assumed a constant velocity $v=5$ $m/s$ for simplicity, which corresponds to the average speed of a fishing boat\footnote{\href{https://shipfever.com/how-fast-can-a-boat-go/}{`How fast can a boat go?'}}. Even if autonomous surface vessels can navigate at smaller speeds, we took it as a reference value to facilitate the processing of the time windows, as will be evident in the following section. Once the method is validated, any speed can be set.

%At the moment, it does not need to know full details about how the boat will reach the destination. We should regard the drone as a particle and ignore other factors as its specific shape and equations of motion. So we take the assumption of route planning, but applied to a smaller scale. . One of the most widely used ways of environment modeling in Route Planning is Grid (cit The Review Unmanned Surface Vehicle Path Planning: Based on Multi-modality Constrain)

\section{Spatio-temporal approaches}
After defining the configuration space and implementing wave data, we may ask how to perform path planning in this environment and with this type of obstacles. The solution consists in adding time to the configuration space $\mathcal{C}_{space}$, which raises its dimensions from two to three and makes the problem more complex.

The boat can navigate the sea in which $H_{m0}<0.8$ $m$, the so-called $\mathcal{C}_{free}$. Unsafe areas $\mathcal{C}_{obs}$, i.e., where $H_{m0}>0.8$ $m$, are bordered by a contour of point obstacles that do not allow the vehicle to pass through. According to the significant wave height value, these dangerous zones hourly change position, but especially shape. So, their shift cannot be predicted with classical methods, and generating the shortest possible path can be more complicated than at first glance.

We thought about predicting their movements by building a set of temporal maps of obstacles representing the environment at a precise hour of the day, as illustrated in Figure \ref{temporalslice}. Depending on how these maps are used and merged, the path generated by the path planning algorithm will be different, as well as the probability of failing the planning or creating a path where the safety of the USV is not guaranteed.\\
%\begin{enumerate}
%	\item the following path planning will generate a different path
%	\item it has to assure that the boat will never finish in dangerous areas
%\end{enumerate}
\begin{figure}[H]
	\centering
	\includegraphics[width=55mm]{Figures/resultsprova.png}
	\caption{Wave forecast at subsequent hour instants}
	\label{temporalslice}
\end{figure} 
\noindent
Three spatio-temporal approaches have been designed: \textbf{\textit{Sum}}, \textbf{\textit{Global}}, and \textbf{\textit{Local}}. 
\begin{comment}
	\subsection{Algorithm comparison}
	The three algorithms examined are:
	\begin{itemize}[itemsep=0pt]
		\item Potential field method
		\item a searching-based method: A*
		\item a sampling-based method: RRT (and RRT*)
	\end{itemize}
	I compared the 3 path planning and collision avoidance algorithms using 3 different approaches in order to evaluate their performance on the same level.\\
\end{comment}
\subsection{Sum}
This approach merges hourly unsafe areas in a single forecast map, which means that all predictions are considered together during path planning. Since this method is highly conservative, the resulting path will not be the shortest but assures the vehicle's safety. The step procedure is described in the pseudo-code of Algorithm \ref{summing}.
\RestyleAlgo{ruled}
\SetKwComment{Comment}{/* }{ */}

\begin{algorithm}[h]
	\caption{\textbf{\textit{Sum}} approach}\label{summing}
	\KwData{\\
		$H_{m0}\gets array(time, latitude, longitude)$\;
		$N \gets$ \textit{number of time windows}\;
		$cost \gets$ \textit{boolean} $array(time, latitude, longitude)$;
	}
	\KwResult{$[(o_x,o_y)] \gets$ \textit{list of obstacle points}}
	\For{$i$ in range($N$))}{
		$cost$\texttt{[i]} $\gets H_{m0}$\texttt{[i]} $> 0.8$ $m$\;}
	\textit{cost sum} $\gets \sum_{i=0}^{N-1} cost$\texttt{[i]}\;
	\textit{wave contour} $=$ \texttt{np.logical\_xor}(\textit{\small cost sum,} \texttt{binary\_erosion}(\textit{\small cost sum}))\;
	\textit{land contour} $=$ \texttt{np.logical\_xor}(\textit{mask,} \texttt{binary\_erosion}(\textit{mask}))\;
	%\{\textit{extract contour of wave region from $sum\_cost$}\}\;
	%\{\textit{extract land contours from the mask}\}\;
	$[(o_x,o_y)]\gets \text{list of x and y coordinates of the obstacle contour}$
\end{algorithm}

\subsection{Global}
This method takes its name from \textit{global planning} and consists in generating a path step by step using only the information of the current time window in which the boat is. It can be seen as a global path planning limited to an hour's drive, so the environment is unknown outside that time. We can summarize it in 3 steps, which are repeated till reaching the destination:
\begin{enumerate}[itemsep=0pt]
	\item Generate the complete path by considering the time window of the current boat position;
	\item Shift the starting point to the path point reached in 1 hour of travel;
	\item Change the time window to that of the next hour.
\end{enumerate}
It simulates the real case where the boat plans a route without considering how weather changes, with the purpose not to meet any danger in the following hour. We described the entire process in Algorithm \ref{global}.
%As you can imagine, this method can be useful in some cases when the path is short or static. However not considering weather prediction leads to the possibility of being in "non-safe" zones at the edges of the time windows.
There are 2 cases in which the algorithm cannot find a path:
\begin{enumerate}[itemsep=0pt]
	\item  No feasible paths to reach the goal are available at a specific hour due to obstacle regions;
	\item During navigation, the boat ends up within an unsafe zone when the temporal map updates.
\end{enumerate}
In the algorithm, the first issue is worked out by moving the vessel for 1 hour along the previous path, generated with the time window $t_{i-1}$, till the new update. The second one cannot be controlled, implying that re-planning at hourly intervals without considering future obstacle predictions does not guarantee the method's feasibility, regardless of the results in terms of travel time.\\
Lastly, we remind that this planning, as for the other methods, is performed in advance, i.e., before the survey mission starts. As a result, possible collisions or sinkings prevent the vehicle from leaving, determining only an efficiency reduction and not the approach's safety.
\newpage
% in reality, as soon as the weather map changes, the USV could suddenly be in a dangerous sea area. \textcolor{red}{spiegare/risolvere}
\subsection{Local} 
The last developed approach is \textbf{\textit{Local}}, which gets its name from \textit{local planning} since it well approximates the behavior of a local planner to avoid dynamic obstacles. Moving obstacle prediction is addressed using a unique forecast map divided into circular time bands generated with the center at the starting point. Each annulus has a width that corresponds to 1 hour of ideal travel, $18$ $km$, and will include the corresponding unsafe contributions of that specific time window. The generation of the obstacle map is described in Algorithm \ref{local}, while Figure \ref{localobs} illustrates the concept.

The current method is based on the strong assumption that the USV will cross each temporal band in precisely 1 hour. However, this condition cannot be met because the vessel should always move in a straight line. Therefore, the further away the vehicle gets, the higher will be the safety risk since the information used does not correspond to the actual forecast conditions.
\begin{algorithm}[!htbp]
	\SetKwBlock{Begin}{Begin}{}
	\Begin{
		\caption{\textbf{\textit{Global}} approach}\label{global}
		\KwData{\\
			$H_{m0}\gets array(time, latitude, longitude)$\;
			$N \gets$ \textit{number of time windows}\;
			$cost \gets$ \textit{boolean} $array(time, latitude, longitude)$\;
			$(s_x, s_y)\gets start$ , $(g_x, g_y)\gets goal$\;
			$res\gets$ \textit{grid size};}
		\KwResult{Path points and list of re-planning way-points}
		\For{$i$ in range($N$))}{
			$cost$\texttt{[i]} $\gets H_{m0}$\texttt{[i]} $> 0.8$ $m$\;
			{\small\textit{ wave contour} $=$ \texttt{np.logical\_xor}(\textit{\small cost}\texttt{[i]}, \texttt{binary\_erosion}(\textit{\small cost}\texttt{[i]}))}\;
		}
	\For{$i$ in range($N$)}{
		\For{$k$ in range(len(latitude))}{
			\For{$j$ in range(len(longitude))}{
				\If{cost\texttt{[i][k][j]}}{\textit{add x and y coordinates to the map of obstacles}}
				\If{wave contour\texttt{[i][k][j]}}{\textit{add x and y coordinates to the map of obstacle contour}}
			}
		}
	}
	}
\end{algorithm}
\begin{algorithm}[!htbp]	
	\SetKwBlock{Begin}{}{end}
	\Begin{
		$k\gets0$\;
		$dist \gets \sqrt{(s_x-g_x)^2+(s_y+g_y)^2}$\;
		\textit{path cost} $\gets0$\;
		\Comment{\textcolor{blue}{// as long as it does not reach destination}}
		\While{$dist\leq res$ \hspace{1cm} \Comment{\textcolor{blue}{// start planning the route}}}{
			$[(r_x,$ $r_y)]\gets$ \textit{\small list of resulting coordinates from path planner}\;
			$q \gets 0$ \;
			\While{\textit{path cost} $\leq 18(k+1)$}{
				{\small \textit{path cost} $\mathrel{+}=$ \textit{distance between subsequent path points in km}}\;
				\If{\textit{path cost} $\geq 18(k+1)$}{\textit{come back to the previous path point}\; \texttt{break}\;}
				$q \mathrel{+}= 1$	
			}
			${s_x}_{new} = r_x$\texttt{[q]}; \Comment{\textcolor{blue}{// re-planning way-point}}
			${s_y}_{new} = r_y$\texttt{[q]}\;
			\textit{final path}$_x$.\texttt{extend}$(r_x$\texttt{[:q]})\;
			\textit{final path}$_y$.\texttt{extend}$(r_y$\texttt{[:q]})\;
			
			\If{path is not found }{
				\Comment{\textcolor{red}{// $1^{st}$ problem}}
				\textit{move of 1 hour's drive along the path generated with the previous time window}
			}
			$k \mathrel{+}= 1$\;	
			\If{${s_x}_{new}$, ${s_y}_{new}$ are within cost\texttt{[k]}}{
				\Comment{\textcolor{red}{// $2^{st}$ problem}}
				\textit{the USV is lost}\;
				\texttt{break}}
		}
	}
\end{algorithm}

%Another method to merge the time windows and predict obstacle movement 
\begin{algorithm}[h]
	\caption{\textbf{\textit{Local}} approach}\label{local}
	\KwData{\\
		$H_{m0}\gets array(time, latitude, longitude)$\;
		$N \gets$ \textit{number of time windows}\;
		$cost \gets$ \textit{boolean} $array(time, latitude, longitude)$;}
	\KwResult{$[(o_x,o_y)] \gets$ \textit{list of obstacle points}}
	\For{$i$ in range($N$))}{
		$cost$\texttt{[i]} $\gets H_{m0}$\texttt{[i]} $> 0.8$ $m$\;}
	$image \gets $ \texttt{np.zeros}((len(latitude), len(longitude)))\;
	\For{$i$ in range($N$)}{
		\For{k in range(len(latitude))}{
			\For{j in range(len(longitude))}{{\small\textit{ compute geodetic distance between point}\texttt{[k][j]} \textit{and the harbor}}\;
				\If{$18i<$ \textit{geo distance} $<18(i+1)$ and cost\texttt{[i][k][j]}}{\textit{add x and y coordinates to the map of obstacles}\;
					image\texttt{[k][j]} = 1}
			}
		}
	}
	\textit{wave contour} $=$ \texttt{np.logical\_xor}(\textit{image,} \texttt{binary\_erosion}(\textit{image}))\;
	$[(o_x,o_y)]\gets \text{list of x and y coordinates of obstacle contour}$
	%\{\textit{extract contour of the map of obstacles}\}\;
	%\{\textit{extract land contours from the mask}\}\;
\end{algorithm}


\begin{figure}[H]
	\centering 
	\subfloat[]{\includegraphics[width=0.29\textwidth]{Figures/localexample2.png}\label{t4}}
	\hspace{9cm}
	\subfloat[]{\includegraphics[width=0.29\textwidth]{Figures/localexample3.png}\label{t4}}
	\hspace{1cm}
	\subfloat[]{\includegraphics[width=0.29\textwidth]{Figures/localexample4.png}\label{t5}}
	\vspace{0.0cm}
	\captionsetup{font=footnotesize,labelfont=footnotesize}
	\caption{(a) Unsafe marine areas at two consecutive hours. (b) Point obstacles considered in the fourth band.  (c) Point obstacles considered in the fifth band.} 
	\label{localobs}
\end{figure}

\subsection{Uncertainty of the forecasts}
Motion planning in the real world is generally subjected to uncertainty in the environment, especially when the adverse weather represents the obstacles. Due to initial condition uncertainties and model errors, forecasts are never accurate \cite{slingo2011uncertainty}, and the error becomes more prominent with the horizon length.\\
Wave forecasts of Copernicus Marine Service are updated two times daily, at 06:00 UTC and 20:00 UTC. So the USV should manage prediction uncertainty for a temporal horizon of 14 h.
%Considering a temporal horizon of 14 h, the amount of prediction uncertainty will increase with the advancing time.

In order to improve the safety, a potential solution would be to expand the unsafe regions, represented by areas of the configuration space in which $H_{m0} > 0.8$ $m$, by appropriate bounded margins, depending on the degree of uncertainty. Figure \ref{safmargin} displays possible safety margins at three times of the day from the forecast update.
It should be noted that an overestimation of the uncertainty could lead to excessive coverage of $\mathcal{C}_{free}$, affecting the optimality of the trajectory and even possibly preventing from finding a solution \cite{mothes2019trajectory}, as the conservative forecast map obtained by the \textbf{\textit{Sum}} approach.\\
%This uncertainty could be taken into account explicitly by bounded margins, i.e.,bounded uncertainty.
In this thesis work, this aspect will not be examined. Nevertheless, it was appropriate to mention it for performing more realistic simulations.
\vspace{0.5cm}
\begin{figure}[h]
	\centering 
	\subfloat[06:00 UTC]{\includegraphics[width=0.32\textwidth]{Figures/uncertainty0.png}\label{0margin}}
	\hspace{0.1cm}
	\subfloat[13:00 UTC]{\includegraphics[width=0.32\textwidth]{Figures/uncertainty1.png}\label{1margin}}
	\hspace{0.1cm}
	\subfloat[18:00 UTC]{\includegraphics[width=0.32\textwidth]{Figures/uncertainty2.png}\label{2margin}}
	\vspace{0.0cm}
	\caption{Safety margins at three different hours of the forecast} 
	\label{safmargin}
\end{figure}
%In this work, the position of obstacles, represented by wave forecasts, is considered deterministic, despite of the stocasticity of the data.Data are stochastic because they come from numerical weather models.
%Uncertainties associated with forecast data naturally exists, and become more prominent with the horizon length of the forecast. The hazardous regions can be modeled as probabilistic closed sets unsafe to fly through  (planning random obstacles)
%The immanent uncertainty in the environmental prediction is taken into account explicitly by bounded margins
\newpage
\section{Summary}
In this chapter, we approached the problem of path planning with moving obstacle prediction.\\
We chose the region of the Tyrrhenian Sea that washes the coasts of Tuscany as configuration space due to its naturalistic spots and dimensions suitable for the simulations. Areas with significant wave height over 0.8 m represent the obstacles to avoid during navigation. \\
The spatio-temporal map generation is performed through three different methods, \textbf{\textit{Sum}}, \textbf{\textit{Global}}, and \textbf{\textit{Local}}, which will be evaluated in the next chapter.

%\subsection{Implementation of sea currents}
%\begin{figure}[H]
%	\centering
%	\includegraphics[width=70mm]{Figures/current.png}
%	\caption{Sea current planning}
%	\label{current}
%\end{figure}

%\textcolor{red}{describe better how I implemented the planning part} 

%\subsection{Uncertainty of the forecasts}
%We know that predictions suffer of a grade of uncertainty: they are not exact, especially if they provide information at very small resolution, as in this case. Although they are short-range forecasts (1-2 days), it's better to account for the variability.


\begin{comment}
	\begin{enumerate}
		\item \textbf{Improved sum}: basic sum method is too much conservative. Summing the temporal maps all together leads to consider $t_5$ when the boat starts. And, same thing, to consider $t_1$ very far from the starting point. The idea would be to remove first map after 1h of theoretical travel. In a way, it could be considered as an union between basic sum and time-band (local) method 
		\item \textbf{Improved Global}: this method is based on a careful study of wavefronts movements. The classic algorithm tries to return the optimal path by remaining near the obstacle. A tolerance around dangerous waves could indicate a safety risk and reduce the possibility that the problem at the edge of time windows occurred. 
		\item \textbf{Improved Local}: instead of disjoint time window bands, consider overlapped windows to take approximate better the behavior in real situation: the boat will never travel each band in the theoretical time
	\end{enumerate}

\end{comment} 