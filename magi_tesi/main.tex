\documentclass[12pt,a4paper,twoside, openright]{book}
\usepackage[utf8]{inputenc}
\usepackage{amsmath, bm}
\usepackage{algorithm2e}
\usepackage{setspace}
\usepackage{amssymb}% http://ctan.org/pkg/amssymb
\usepackage{pifont}% http://ctan.org/pkg/pifont
\newcommand{\cmark}{\ding{51}}%
\newcommand{\xmark}{\ding{55}}%
\SetArgSty{textnormal}
%\usepackage[noend]{algpseudocode}
%\makeatletter
%\def\BState{\State\hskip-\ALG@thistlm}
%\makeatother
%\usepackage{siunitx}
\usepackage{mathcomp}
\usepackage{graphicx}
\usepackage{mathtools}  % for conditional equations
\usepackage[hidelinks]{hyperref}
\hypersetup{colorlinks,urlcolor={NavyBlue},linkcolor={NavyBlue},citecolor={NavyBlue}} 
\usepackage[numbers,sort&compress]{natbib}
\bibpunct{\color{NavyBlue}[}{\color{NavyBlue}]}{,}{n}{}{;}
\usepackage{caption}
\captionsetup[table]{position=bottom}
\usepackage{float}
\usepackage{xurl}
\usepackage[dvipsnames]{xcolor}
\usepackage{comment}
\usepackage{lscape} 	% for large tables
\usepackage{subfig}
\usepackage{nomencl}
\usepackage{enumitem}
\renewcommand{\chapterautorefname}{Chapter}
\usepackage{appendix}
\usepackage{setspace}  % interlinea
\usepackage{bigstrut}
\usepackage{booktabs}
\usepackage{multirow}
\usepackage{hyperref}
\usepackage[a4paper, width=150mm, top=25mm, bottom=25mm, bindingoffset=6mm]{geometry}
\usepackage{verbatim}
\usepackage{array}
\newcolumntype{?}{!{\vrule width 2pt}}
\usepackage{fancyhdr}
\usepackage{emptypage} % serve per fare le pagine vuote tra un capitolo e l'altro
%modifica intestazioni
\pagestyle{fancy}
\renewcommand{\headrulewidth}{0.4pt}
\renewcommand{\footrulewidth}{0.4pt} % piede
\fancyhead{}
\fancyhead[RO]{\slshape\nouppercase{\rightmark}}
\fancyhead[LE]{\slshape\nouppercase{\leftmark}}
% intestazioni il minuscolo
%\renewcommand{\chaptermark}[1]{%
%	\markboth{\chaptername
%		\ \thechapter.\ #1}{}}	
%\renewcommand{\sectionmark}[1]{\markright{\thesection\ #1}}
\fancyfoot{}
\fancyfoot[C]{\thepage}
\usepackage[T1]{fontenc}


\begin{document}

%\newgeometry{left=3cm,right=3cm,bottom=2cm,top=3cm}
\begin{comment}
\begin{center}
    \includegraphics[height=3.5cm]{Figures/logo.png}
    
    \noindent\hrulefill
    \vspace{0.5cm}
    
    {\Large \textsc{Department of Industrial Engineering}}
    \vspace{0.3cm}
    
    {\large \emph{Master's Degree in Mechatronics Engineering}}
    \vspace{1cm}
    
    {\normalsize \textsc{Master Thesis}}
    \vspace{2cm}
    
    %\linespread{1}
    %{\LARGE \textbf{\textsc{Decision-making System for an Autonomous Surface Vessel depending of Weather Forecasting }} \par}
    
    %{\LARGE \textbf{\textsc{Trajectory planning Decision-making System for an Autonomous Surface Vessel depending of Weather Forecasting }} \par}
    
    
   % \begin{spacing}{1.5}
   % 	    {\LARGE \textbf{\textsc{Decision-making of Autonomous Surface Vessel depending of Marine Weather Forecasting "non definitivo":}}}\par{\Large \textbf{\textsc{Comparison of Path Planning Algorithms}}}
  %  \end{spacing}

 	%\begin{spacing}{1.5}
	%	{\LARGE \textbf{\textsc{Path planning of Autonomous Surface Vessel depending of Marine Weather Forecasting "non definitivo":}}}\par{\Large \textbf{\textsc{Comparison of Time-Windows approaches}}}
%	\end{spacing}

 	\begin{spacing}{1.5}
	{\LARGE \textbf{\textsc{Spatio-temporal Path Planning using Wave Forecasts for Autonomous Vessels}}}
	\end{spacing}

    
    %\linespread{1}
    \vspace{2.5cm}
    \begin{minipage}[t]{0.34\textwidth}
    \begin{flushleft}
    {\large Supervisor:} \\ \textbf{Prof. Daniele Fontanelli}
    \end{flushleft}
    \end{minipage}
    \begin{minipage}[t]{0.64\textwidth}
    \begin{flushright}
   	{\large Candidate:} \\ \textbf{Giorgio Checola}

    \end{flushright}
    \end{minipage}\\
    \vspace{2cm}
    {\large Company supervisor\\ Witted Srl:}\\
    \textbf{PhD Emanuele Rocco}
    \vspace*{\stretch{1}}
    \vfill
    \vspace{0.3cm}
    \noindent\hrulefill
    \vspace{0.3cm}
    \Large
    
    Academic Year {2020-2021}
\end{center}
\end{comment}
\begin{titlepage}
	\begin{center}
		\includegraphics[height=4cm]{Figures/logo.png}
		
		\noindent\hrulefill
		\vspace{0.5cm}
		
		{\Large \textsc{Department of Industrial Engineering}}
		\vspace{0.3cm}
		
		{\large \emph{Master's Degree in Mechatronics Engineering}}
		\vspace{1cm}
		
		{\large \textsc{Master's Thesis}}
		\vspace{2cm}\\
		
		%\linespread{1}
		%{\LARGE \textbf{\textsc{Decision-making System for an Autonomous Surface Vessel depending of Weather Forecasting }} \par}
		
		%{\LARGE \textbf{\textsc{Trajectory planning Decision-making System for an Autonomous Surface Vessel depending of Weather Forecasting }} \par}
		
		
		% \begin{spacing}{1.5}
			% 	    {\LARGE \textbf{\textsc{Decision-making of Autonomous Surface Vessel depending of Marine Weather Forecasting "non definitivo":}}}\par{\Large \textbf{\textsc{Comparison of Path Planning Algorithms}}}
			%  \end{spacing}
		
		%\begin{spacing}{1.5}
		%	{\LARGE \textbf{\textsc{Path planning of Autonomous Surface Vessel depending of Marine Weather Forecasting "non definitivo":}}}\par{\Large \textbf{\textsc{Comparison of Time-Windows approaches}}}
		%	\end{spacing}
	
	%\begin{spacing}{1.5}
	\setstretch{1}
	{\LARGE
		\textbf{\textsc{Spatio-temporal Path Planning \\ using Wave Forecasts for \\ Autonomous Vessels\\}}}
	%{\LARGE \textbf{\textsc{Spatio-temporal Path Planning \\ using Wave Forecasts for \\ Autonomous Vessels}}}
	%\end{spacing}
	
	%\linespread{1}
	\vspace{2.5cm}
	\begin{minipage}[t]{0.34\textwidth}
		\begin{flushleft}
			\setstretch{1.5}
			{\large Supervisor:} \\ \textbf{Prof. Daniele Fontanelli}
		\end{flushleft}
	\end{minipage}
	\begin{minipage}[t]{0.64\textwidth}
		\begin{flushright}
			\setstretch{1.5}
			{\large Candidate:} \\ \textbf{Giorgio Checola}
			
		\end{flushright}
	\end{minipage}\\
	\vspace{4cm}
	\setstretch{1.2}
	{\large Company supervisor\\ Witted Srl:}\\
	\setstretch{1.5}
	\textbf{PhD Emanuele Rocco}
	\vspace*{\stretch{1}}
	\vfill
	\vspace{0.3cm}
	\noindent\hrulefill
	\vspace{0.3cm}
	\large
	
	\textsc{Academic year {2020 - 2021}}
\end{center}
\end{titlepage}



\pagestyle{empty}

\clearpage
\null
\clearpage

\restoregeometry

%\thispagestyle{empty}
\begin{flushright}
\null \vspace{\stretch{1}}
\emph{A chi mi vuole bene}
\vspace{\stretch{2}}\null
\end{flushright}

\clearpage
\null
%\thispagestyle{empty}
\clearpage

\frontmatter
\setstretch{1.5}
\pagestyle{plain}
\begin{flushleft}
	\textbf{\Large{Preface}}
\end{flushleft}

\noindent
This thesis constitutes my master's degree in Mechatronics Engineering at the University of Trento. 
The following work describes the research project I dedicated myself to from September 2021 to May 2022 at Witted Srl, a tech start-up of Progetto Manifattura in Rovereto. Witted does R\&D on robotic systems and artificial intelligence to explore and monitor marine habitats with the purpose of protecting the natural environment.

The project was conceived together with the CTO of the company, Emanuele Rocco, and came from the will to merge two different topics: my university studies, centered on mechatronics and robotics, and my interest in meteorology and weather data observation. I want to thank the company for the opportunity since this experience has helped me expand my knowledge in the field of autonomous navigation.

%It concerns the study of different spatio-temporal approaches for path planning to develop an intelligent system for unmanned surface vessels that makes autonomous decisions depending on marine weather forecasting.

\vspace{4cm}
\begin{center}
	Giorgio Checola \\
	\textit{Trento, May 25, 2022}
\end{center}
%\chapter{\centering{\large{Abstract}}}

\clearpage
\null
\thispagestyle{empty}
\clearpage

\begin{flushleft}
	\textbf{\Large{Abstract}}
\end{flushleft}

%during which they have to select missions, decide routes and optimize operation time 
In the last years, the development of autonomous systems technology in the maritime industry has increased rapidly. This growing activity comes mainly from their application in environmental monitoring, during which the vehicle makes bathymetric surveys, gathers oceanographic and meteorological data, and conducts seabed mapping.
The feasibility of these operations depends on weather conditions, which represent one of the most critical risks for the sea navigation of marine vessels, especially small sizes.
Their relatively low weight increases the vulnerability to adverse sea conditions, often considered external forces in the vessel's kinetic model. 
On the other hand, considering sea weather forecasts at the beginning could help the decision-making system take human-like decisions to improve target selection, mission efficiency, and route optimization.
\vspace{0.3cm}\\
Based on the studies of weather avoidance in aviation and the literature on spatio-temporal trajectory planning of autonomous cars, this thesis has focused on developing an algorithmic approach that determines the best navigation strategy depending on marine weather forecasts. The proposed methodology used sea forecast data, made available by Copernicus Marine Service as NetCDF files.
\vspace{0.3cm}\\
The concept consisted in creating specific forecast maps that included the wave heights predicted for the following hours as time-varying $\mathcal{C}_{obs}$ unsafe areas. We initially designed three spatio-temporal approaches and, based on the results, conceived an improved version of map generation.
The maps were then used to find deliberative paths in the Tyrrhenian sea area.
\vspace{0.3cm}\\
Three algorithms performed path planning, i.e., the improved version of Rapidly-Exploring Random Tree (RRT*), A*, and the Artificial Potential Field (APF), on which we made a comparison to evaluate their performances.
We present simulation results obtained during several days of navigation, assessing path length, probability of finding a viable path, and USV safety. 
\vspace{0.3cm}\\
Besides achieving excellent route length results, the last proposed spatio-temporal approach accurately describes wave heights without overestimating the obstacles' positions, ensuring the boat navigation safety. It represents, therefore, an efficient solution for considering hourly weather forecasts in a path planning problem, paving the way for the enhancement of robotic perception towards the surrounding environment.

%aprendo la strada a una pianificazione piu incentrata sui fenomeni ambientali che potrebbero influire su il 

%The last proposed spatio-temporal approach has achieved excellent route length results for considering hourly weather forecasts in a path planning problem. Besides achieving excellent route length results, the map accurately describes wave heights without overestimating the obstacles' positions, ensuring the boat navigation safety.
%represents the most efficient method for considering hourly weather forecasts in a path planning problem. 

%In conclusion, Potential implications of this study range from improving shipping and monitoring operations to minimizing human contribution and collision risk at sea.
%Further work should cover boat's behavior while effectively in motion
%A possible application could help increasing the autonomy of autonomous vessels at sea during survey missions improvement of mission planning and autonomy in the operations of surface vessels.
%The implementation of weather forecasts in the USV data sources can have potential implications for optimizing vessel operations.
%and performance of the operations with implications for shipping and environmental monitoring companies.
%Further work should examine the boat's behavior while effectively in motion. In this way 
%Results have showed that it can assure   it has been determined that implementing weather forecasts  can modify navigation routes, improving mission planning and the safety of autonomous vessels.
%improve route optimization and navigation safety, enhancing mission planning of autonomous vessels.
%We run simulations over several days, assessing path length, probability of finding a viable path, and safety of navigation. Results show the effectiveness of finding a global path that predicts the motion of the wavefields \textit{a priori} without affecting the 

%The improved approach successfully predicts the shifting of adverse weather in the nearest future and can have exciting applicability in this type of application.

%to find  that performs path planning finding the best path with time-varying $\mathcal{C}_{obs}$ in order to select missions and optimize the route considering weather forecast \textit{a priori}. 
%We initially designed three spatio-temporal approaches 
% that could outperform them.

%We designed four methods, the last of which is based on the performance results of the others.

%in multidimensional array form stored in NetCDF format. 
%\textbf{was perfected, was successfully implemented under extreme conditions, reproduced data with a precision of, additional advantages include, consist, demonstrates for the first time that, in contrast to reports in the literature}

% Results of the improved methods have concluded that the proposed approach can find optimal paths for predicting the shifting of adverse weather in the nearest future and can have exciting applicability in this type of application.

%We implemented three path planning algorithms, i.e., the improved version of Rapidly-Exploring Random Tree (RRT*), A*, and the Artificial Potential field (APF) which  
%These forecast maps are employed by three different path planning algorithms that we compared: the improved version of Rapidly-Exploring Random Tree (RRT*), A*, and the Artificial Potential field to find the one that generates the best navigation route.


%The region of interest, i.e., the configuration space, was deeply analyzed including conduction of Posidonia oceanica sensing and wave statistical analysis to understand best period to navigate, percentage of waves higher than the threshold considered, wave direction, etc.
%3 general algorithms were chosen to perform path planning at the beginning since we didn't know how to treat sea factors and approach the problem.


%From the results we considered appropriate to design a new method that could outperform the initial ones allowing its use in real applications.
%The improved approach, based on the performance of the other three, have obtained excellent results: 100\% safety and 85.3 efficiency reaching to find a path in all cases where it was possible.
%on which perform path planning: compares three types of path planning algorithms, Rapidly-Exploring Random Tree (RRT), Artificial Potential field and A*, to plan efficient and safe routes with hourly spatio-temporal forecasts. Each algorithm has been modified accordingly. 
% of time-varying $\mathcal{C}_obs$ represented by risky sea waves.
%The availability of this type of files allows to analyze more in depth our configuration space thanks to which we drew these conclusions. 

%the navigation
%often neglected or accounted for as external factors
%In order to act as an autonomous system, the USV have to process the received data, analyze them, and make decisions. These actions are performed by the decision-making system.
%As said, sea factors affect the navigation of these vehicles, especially if they are small. Often they are seen as disturbances on which some kind of control system is applied.
%During survey monitoring, it represents an even greater problem. The autonomy of these drones consists also in autonomous decision-making in order to substitute the human contribution.
%Decide when and where analyze a particular spot taking into account several environmental data could help to accelerating these operations and making these vehicles more and more autonomous.\\
%abstract: condensed version of your overall story. poi nt out all the major features of the investingation



%Unmanned vehicles are now everywhere in our today life to help making autonomous human operations. The sea has always been one of the greater uncertainties inherent in man. \\
%Monitoring this environment and its depths can make new discoveries to help humans to counter problems related to climate change which is causing natural disaster in flora and fauna.\\
%Autonomous surface vessels are accelerating this discovery process. 
%External factors as environmental disturbances have always been considered in the dynamics of the vehicle. In this thesis these factors are considered in a new way: no more external forces, but obstacles from which the vehicle have to stay away. All this in the context of time optimization and safety purpose of small ASV, the more affected by marine weather. The vehicle receives weather data and plans the survey mission accordingly.



%Domenico advices
%\begin{itemize}
%	\item Motiovation of what i did
%	\item why i did it
%	\item why it's important
%	\item what is related to, future perspectives
%\end{itemize}
%parole utili: weather routing references, netcdf planning, spatio temporal planning, decision making, safety.why alteranative method to controller, how to take into account big maps of obstacles which move \\

%Things to add in the thesis:
%\begin{itemize}[itemsep=0pt]
%	\item decide a variance in wave values to account for uncertainty
%	\item compute cost as velocity + current velocity
%	\item toy example without meteo data: limit cases in which sum global and local fail
%	\item improved methods validating them with failure ratio (without images)
%	\item study correlation betweeen current, waves direction, velocity etc
%	\item static planning using current direction
%	\item correcting APF algorithm
%	\item RRT and A* cost function which increases with the dircetion of the current velocity
%\end{itemize}
  

{\tableofcontents
\let\cleardoublepage\clearpage 
\listoffigures
\let\cleardoublepage\clearpage 
\listoftables}
%\listofalgorithms
%\makenomenclature
%\mbox{}
%\renewcommand{\nomname}{List of Symbols}
%\nomenclature{\(+a\)}{Operator}
%\nomenclature{\(2a\)}{Number}
%\nomenclature{\(:a\)}{Punctuation symbol}
%\nomenclature{\(Aa\)}{Uppercase letter}
%\nomenclature{\(aa\)}{Lowercase letter}
%\nomenclature{\(\alpha\)}{Greek character}

%\printnomenclature

\mainmatter
\pagestyle{fancy}
%\fancyhead[ER]{\nouppercase\leftmark}
%\fancyhead[OR]{\nouppercase\rightmark}
%\fancyhead[ER,OR]{\thepage}
\chapter{Introduction}
\label{intro}

%The kind of autonomous surface vessel (ASV) which I'll refer from now on is vessel of small size, our main focus.

%Small-scale operations include bathymetric surveys, pollution monitoring and data assimilation in a cluttered environment where the generation of safer way-points have the highest priority in the path planning.
An \textit{autonomous vehicle} is a comprehensive intelligent system that integrates environmental perception, path planning, decision-making, and motion controlling technologies \cite{9564580}. More specifically, an autonomous surface vehicle is a vessel that can make decisions and operate independently without human guidance, navigation, and control. They are especially used in military operations, maritime surveillance cruises, and marine environmental monitoring applications \cite{vagale2021path}, the topic on which this thesis will focus. 
%Nowadays they are used in many applications including marine explorations, oceanography, maritime search and rescue, surveillance, defense.\\
\section{Background and Motivation}
%"In the last years the progress in unmanned surface vehicles has increased rapidly. Developing autonomous ships opened many possibilities of use, previously limited due to the additional human constraints"
In the last years, the use of unmanned vehicles in the marine environment has been increasing rapidly \cite{5152052}. Thanks to their substantial autonomy and adaptability, these ships have gradually become the new research direction pursued by the current industry. For example, Unmanned Surface Vehicles (USVs) are more suitable than manned ships for dealing with complex and changeable rough sea environments \cite{s20020426,vagale2021path}. This study is motivated by continuing efforts to improve operational safety and performance.

One of the most important reasons for this growing activity is the need for better environmental monitoring, including meteorological and ecological studies. Its vast potential makes this type of vehicle essential for preserving our environment.
Thanks to renewable solar and wind power sources, sailing robotics could be used for long-term missions and semi-persistent presences in the oceans \cite{towards}.
%Many of these vessels can stay at sea for many days, if not months, without any energy recharge, because they exploit wind and wave forces and radiation of the sun to collect all the energy they need. 

The Mayflower Autonomous Ship Project is a prime example of this innovative progress. Carried on by ProMare, with IBM's support, they developed an autonomous boat (Figure \ref{mayflower}) that can spend a long time at sea, carry scientific equipment, and make its own decisions about how to optimize its route and mission \cite{Mayflower}. 
Three other leading companies are worth mentioning: the Norwegian Maritime Robotics, which has produced unmanned solutions for seabed mapping and monitoring of sheltered waters \cite{MaritimeRobotics} (USV Otter in Figure \ref{otter}); the American Saildrone company, whose vehicles (Figure \ref{saildrone}) collect ocean data that provide unprecedented intelligence for climate, mapping, and maritime security applications \cite{Saildrone}; and finally Witted \cite{witted} that designs autonomous drones for protecting the biodiversity of coastal habitats (see Figure \ref{wittsail}).

Autonomy has to be achieved not only in terms of energy but also in terms of sailing decisions. Decision-making autonomy under variable and possibly dangerous navigation conditions is an aspect that has yet to be widely explored.
\vspace{0.5cm}
\begin{figure}[H]
	\centering
	\subfloat[Mayflower Autonomous Ship]{\includegraphics[height=0.30\textwidth]{Figures/mayflower.jpeg}\label{mayflower}} 
	\hspace{0.2cm}
	\centering
	\subfloat[Otter USV]{\includegraphics[height=0.30\textwidth]{Figures/otterusv2.png}\label{otter}}
	\hspace{0.2cm}
	\subfloat[Saildrone USV]{\includegraphics[height=0.31\textwidth]{Figures/asvsaildrone.jpg}\label{saildrone}}
	\hspace{0.2cm}
	\subfloat[Argo USV by Witted]{\includegraphics[height=0.31\textwidth]{Figures/wittedusv3.jpg}\label{witted}}
	\caption{Examples of autonomous surface vessels}
	\label{wittsail}
\end{figure}
%The decision-making system in autonomous vehicles is the transition of environmental perception system and motion planning system. 
As the ``brain'' of autonomous vehicles, a decision-making system is significant for vehicles' safety and efficient driving. Its purpose for marine vessels is to generate human-like decisions prioritizing \textbf{driving efficiency}, i.e., voyage time and energy consumption, but above all, \textbf{safety} at sea. 

This research project aims to develop an autonomous surface vessel decision-making system that plans survey missions and paths depending on marine weather forecasting.
Weather variations are one of the most critical factors that should be considered since the sea environment is sparse compared to classical indoor robotics environments. So, besides possible other ships, the most significant risk is represented by environmental disturbances.

This thesis will focus on \textbf{path planning}, whose results are one of the aspects mainly considered to achieve better decisions \cite{9564580}. Autonomous path planning plays a crucial role in ship automation and practical application as the basis and premise of autonomous navigation \cite{s20020426, vagale2021path}. The basic idea is that the vessel receives marine weather data and other helpful information earlier in the day before the survey starts; it analyzes them; and plans the day's mission accordingly.

%Ideally, weather data are delivered in real-time, allowing re-plan the route and update the decisions. However, it is out of this study. 
The core will be \textbf{weather predictions} and how they can influence the path planning and the possibilities of action of small vessels. The necessity to first perform a global path planning which accounts for possible weather risks is essential for the success of the daily monitoring mission. Maritime Autonomous Surface Ships (MASS) and large-scale planning are excluded from this study and should be addressed differently. 

%The topic is related to autonomous navigation for habitat monitoring. Autonomous surface ships (MASS) and large-scale planning are not considered and should be addressed in other ways.


%Trajectory planning instead requires to include a temporal constraint to the geometric path, and so considering vehicle dynamics (vagale)
%The research makes progress in the collision avoidance techniques and International Regulations for Preventing Collisions at Sea (COLREGs).
%The lack of environmental factors in the algorithm was one of the main limitation of algorithms until some years ago. (vagale)
%As previously said, crucial aspect of ASVs is safety and the ability to safely navigate in open waters, coastal areas, and congested waters like harbours. Clearly, the safety issue is the most challenging when avoiding collisions with other dynamic vessels or land in high-traffic congested waters (vagale)
%"Finding an optimal safe path while driving in a lane can often be simpler than on unstructured roads or open areas where the distribution of obstacles is irregular. The complexity of the environment and kinodynamics makes path planning of surface vessels more challenging and different from ground vehicles." (vagale)

\begin{comment}
	\begin{itemize}	
		\item The field of operational planning covers different aspects that range from weather forecasting  to safe ship navigation, handling to routing optimization. Autonomous planning of vessel operations
		Decision-making represents a step-forward in the field of autonomous vehicles. In this case we are focused on decision-making process during the planning of ship operations.
		There are still many challenges to be solved. The goal  
		\item \textbf{Collision avoidance} problem, external environmental disturbances. How to treat them? Wind, waves and sea current constitute
		\item \textbf{spatio temporal trajectory planning} non uso kalman perchè io so come saranno gli ostacoli alla window t2, t3, t4...
		\item how to simulate \textbf{weather data}?: 
	\end{itemize}
	\begin{figure}[H]
		\centering 
		\subfloat[Otter USV]{\includegraphics[width=0.4\textwidth]{Figures/otterusv.png}\label{witted}}
		\hspace{0.2cm}
		\subfloat[Witted's USV]{\includegraphics[width=0.4\textwidth]{Figures/barca.jpg}\label{2}}
		\hspace{0.2cm}
		\subfloat[Mayflower Autonomous Ship]{\includegraphics[width=0.4\textwidth]{Figures/mayflower.jpeg}\label{3}}
		\hspace{0.2cm}
		\subfloat[Saildrone USV ]{\includegraphics[width=0.4\textwidth]{Figures/asvsaildrone.jpg}\label{4}}
		\hspace{0.2cm}
		\subfloat[L3Harris USV]{\includegraphics[width=0.4\textwidth]{Figures/asv2.jpg}\label{5}}
		\hspace{0.2cm}
		\subfloat[NAVCENT USV ]{\includegraphics[width=0.4\textwidth]{Figures/asvaraba.JPG}\label{Otter}}
		\caption{USV examples} 
		\label{usvsssss}
	\end{figure}
\end{comment}

\begin{comment}
	\subsection{Concept of safety and risk assessment}
	Safety Map Generator
	Improved safety and reduction of collision risk are the key objective of the guidance, navigation and control system of an autonomous surface vessel, especially the ones of small size.

	As it can be imagined, vessel safety depends largely on the size of the vehicle. But also on the stability properities, heading and speed of the vessel in any given sea state. More information you have, smaller will be the probability of some occurrence. For this reason testing, and simulations are fundamental before the implementation in the real environment.
	
	Environmental effects and moving obstacles have a significant role in path planning of USVs. Neglecting them can increase the potential danger besides determine a waste of energy
	\cite{singh2018constrained}. (A* constrained)
	
	Safety is directly related to the waves and the sea state in general. The best option in this case to limit the impact forces is reducing vessel speed or change course, the heading of the boat.
	Navigate in extreme conditions cause large roll angles and in the some cases capsize too.
	
	In the case of autonomous surface vessel, the human error is not accounted, but there are other factors that result in an accident.
	
	In simulation, safety can be assured in many different ways: from building a safety map \cite{vagale2020evaluation}, or considering a safety distance around the vehicle \cite{singh2018constrained}, ...
	To achieve the autonomous navigation of USVs in the practical maritime environment, the system should intelligently prioritize certain missions compared to other depending on the environments. It means, for example, to give higher priority to collision avoidance for small-scale operations in cluttered environments, and energy efficiency in large-scale missions. The other problem is the necessity to generate a feasible and optimal trajectory in a complex maritime dynamic environment. (\cite{SONG2017301}).\\
\end{comment}

%We don't speak about Trajectory planning because many other constraints are not considered. Time is taken into account, to introduce dynamic obstacles. Neither optimization of trajectory planning based on some constraints is introduced, the path does not change indeed. 


%The idea of mindset comes from the literature, but with an approach extremely new. Just like in autonomous driving you implement decision-making algorithm on the vehicle predicting the deterministic spatio-temporal trajectory of the other cars, at the same way the objective is to consider the weather forecast in a semi-deterministic way dealing with the concept of safety, already seen in many other similar problem. The vessel should be able to decide which action to take with the aim of maximum safety, and this results in choosing the final point and the best path. In the future it will replace the human work of choosing the right day mission. 

\section{Related work}
Research on autonomous surface vessels is mainly divided into autonomous path planning, navigation control, collision avoidance, and semi-autonomous task execution. This section recalls some of the works which inspired the current project.

Autonomous navigation of USVs in a practical marine environment needs to consider three important issues \cite{lavalle2006planning, statheros2008autonomous, singh2018constrained}:
\begin{itemize}[itemsep=0pt]
	\item Safety;
	\item Reliability of the mission;
	\item Probability of success;
\end{itemize} 
The \textbf{safety} issue is the most challenging and includes collision avoidance with other dynamic vessels, land in high-traffic congested waters, and also harsh weather conditions. Singh \textit{et al.} \cite{singh2018constrained} resolved this problem with a circular boundary enclosing the USV in motion planning simulations. Other ideas have been the introduction of a safety region around obstacles \cite{serigstad2018hybrid} or the direct generation of a safety map \cite{liu2017predictive}, or even the incorporation of COLREG rules, the International Regulations for Preventing Collisions at Sea, 1972.
%Several safety regulations include COLREGs, the International Regulations for Preventing Collisions at Sea 1972, to define how vessels should act in various situations when meeting other vessels to navigate through waters safely and without collisions. The safety factor can be integrated into path planning algorithms, leading to safer waypoints (circular boundary, offset around obstacles, Kalman filter integration). Most leading companies in USV operations are looking for the integration of COLREGs with optimal path planners. 
\vspace{0.3cm}

The concept at the basis of the path planning approach that will be presented in \autoref{approach} is inspired by \textbf{3D spatiotemporal trajectory planning}. Exploited in autonomous driving, this type of planning uses a 3D spatiotemporal map instead of the general grid map to account for the trajectory prediction of the surrounding vehicles.  In this way, the dynamic path planning task in a 2D plane can be transformed into trajectory planning in a 3D static environment \cite{xin2021enable, zhang2021unified, zhang2020trajectory}.\\
Rajendran \textit{et al.} \cite{rajendran2018wave} uses a similar trajectory planning approach to avoid waves generated by traffic vessels. The work proposes a global planner which accounts for large dynamic obstacles; waves are modeled as penalty regions that change spatiotemporally.
\vspace{0.3cm}

%this issue Weather avoidance systems are safety-critical .  addresses this issue  for determining  a linearly moving point, such as , will intersect a polygonal region, such as , .
The safety of navigation is also the objective of \textbf{weather routing}, a class of large-scale path planning problems that uses weather forecasts to provide the optimal route and sailing speed for a given voyage \cite{zis2020ship} (Figure \ref{weatherroute} displays a route example).
%Weather routing concerns  of finding the optimal path and sailing speed in large-scale areas for a given voyage based on the weather forecasts expected to encounter up to the arrival . 
Weather forecast can be delivered in GRIB (GRIdded Binary) format, facilitating the reading of the input data \cite{pennino2020development}. T. Fabbri and R. Vicen-Bueno exploit this concept to provide the best window of opportunity to execute a journey/operation between two predefined locations \cite{fabbri2021decision}. \\
Regarding sea data implementation, several path planning studies test Autonomous Underwater Vehicles (AUVs) in real environmental scenarios to determine energy-optimal paths. The information of the ocean current field for a particular time is stored in the NetCDF data format \cite{rao2009large, eichhorn2015optimal, garau2009path, SONG2017301}.
%In several path planning studies, real environmental data, stored in netCDF format, are implemented to calculate energy-optimal paths that account for the influence of ocean currents. The vehicles are especially underwater gliders, a class of AUVs more affected by the ocean due to their low velocity \cite{rao2009large}, \cite{eichhorn2015optimal}, \cite{garau2009path}, \cite{SONG2017301}.
\vspace{0.3cm}

Lastly, weather avoidance is an ever-present and safety-relevant task studied in the aerospace field due to the potential dangers of convective weather. Narkawicz and Hagen \cite{narkawicz2016algorithms} describe functions for determining whether an aircraft will encounter a cell of convective weather within a predetermined look-ahead time. F. Mothes \cite{mothes2019trajectory} instead proposes a method for robust trajectory planning of fixed-wing aircraft in uncertain dynamic environments based on thunderstorm nowcast. 
\begin{figure}[H]
	\centering
	\includegraphics[width=130mm]{Figures/weatherroute.png}
	\caption{Example of weather routing scale}
	\label{weatherroute}
\end{figure}
\section{Contributions}
Marine weather data are spatio-temporal dependent, so route planning needs to be enhanced by embedding time. Predicting these factors will play a crucial role in generating optimal paths.

This thesis aims to build a link between autonomous navigation of vessel drones and weather forecasts at sea by comparing different approaches for path planning based on time windows that avoid dynamic obstacles represented by rough sea (Figure \ref{topic}). An \textit{optimal trajectory} is defined as taking the shortest time to get to the goal while staying clear of potentially dangerous areas. Therefore, the results are evaluated through three main performance metrics: the \textbf{travel time}, which affects route optimization, the \textbf{efficiency}, and the \textbf{safety}, which indicate respectively the probability of finding a viable path, and the safety of navigation if that path is actually traveled.
%the energy indicates the driving efficiency, and the \textbf{travel success ratio}, or the probability of generating a feasible and safe path, 

The contributions of this thesis are:
\begin{itemize}[itemsep=0pt]
	\item A review of path planning and decision-making of autonomous vessels;
	\item A background in marine weather forecasting;
	\item A statistical study of sea waves in the region of interest;
	\item The design of different spatio-temporal approaches to predict time-varying sea conditions \textit{a priori};
	\item A comparison of path planning algorithms for navigation in complex environments.
\end{itemize} 
\begin{figure}[h]
	\centering
	\includegraphics[width=80mm]{Figures/topic.png}
	\caption{Topic of the thesis}
	\label{topic}
\end{figure}



%Spatio-temporal trajectory planning is applied in all field of autonomous vehicles: UGV in highways or to avoid congestion/traffic, AUV for currents, UAV or aircraft to avoid bad weather encountering evolving scenarios as wind fields, USV where waves are considered as penalty regions.
%Path planning is an extensively researched problem that have been studied for many years and where several approaches have been developed, both in known and unknown environment \cite{lavalle2006planning}. Path planning is an important layer in the mission management system of an USV voyage, and in general of autonomous vehicles. They need an effective and safe path planning approach when they operate in cluttered environments (a constrained a*). Focusing on USVs the three important issues in the autonomous navigation are:


%Integrate environmental data in autonomous navigation is. in 2014 a group of Greek researchers developed the AMINESS platform \cite{giannakopoulos2014aminess}, a innovative monitoring system that will integrate maritime information from multiple sources, accessible by captains and ship owners, that can be used for optimal safe route planning and delivering real-time alerts in a high-risk environment as the Aegean Sea. These vessels are not autonomous, but the idea is the same: helping make decisions by contributing to the safety of the ship and the surrounding environment. This is achieved considering spatio-temporal marine and weather data, risk estimation and optimization algorithms.


%The concept of decision-making and path planning is considered especially in dynamic environment where the risk of collision with obstacles is real, and the vehicle is subject to different options for action.\\
%The example par excellence is autonomous driving: self-driving cars are the vehicles more subject to external signals, disturbances, uncertainties, obstacles, and the risk of accident must be completely avoided.
%In the urban environment safety, comfort and economy are the objectives considered to improve tracking performance \cite{xin2021enable}. And this is better guaranteed with spatio-temporal trajectory planning in constrained dynamic environment. L. Xin et al. represented the driving map with a 3D spatio-temporal map in which the position of dynamic obstacles in the future could be illustrated. And then, a navigation algorithm was used to search the best trajectory. 

%In \cite{zhang2020trajectory} a framework for trajectory planning constrained to safety strip was proposed. The forecasting trajectory of the social vehicles was considered deterministic.
%Intelligent decision making and efficient trajectory planning are closely related especially in highway environment. T. Zhang et al. \cite{zhang2021unified} proposed a novel approach based on voxels for generating driving corridors which predict the states of the surrounding vehicles. Another example was provided by \cite{yao2019path}, where a method based on spatio-temporal congestion coefficient and improved A* Algorithm was derived to improve route planning and avoid traffic congestion.


%Spatio-temporal trajectory planning: Urban environment is not the only one for which a precise and careful attention is needed. Marine and underwater environment can be considered in the same way, despite the lower presence of vehicles. This is due to the fact that the natural environment is more dynamic: weather factors have a major effect on the ship and their measurements is not easy at all. Moreover the consequences can be more catastrophic.
 
%When we talked about decision-making together with path planning, we cannot but cite spatio-temporal trajectory planning, a very innovative approach to deal with this kind of problems. \\
%In the last few years many research groups started dealing with the problem of optimal trajectory planning through the definition of spatio-temporal maps that describe the evolution of dynamic obstacles or environment data in the future. Depending on the context, different length scales are considered. From seconds/minutes in highway scenarios to hours/days for ship routing. Weather, for example, can be predicted in real-time, but it is updated at hourly interval at least. Spatial scale forecast must account for several variables as well.
%The goal is to provide the most amount of information to the autonomous vehicle in order to help it make smart decisions in critical situations and ensure the safety of the vehicle itself.



%NetCDF path planning and weather routing: weather routing is the general problem of route generation to guarantee efficient and safe navigation to the ships. NetCDF data are employed mainly for wave gliders and AUVs. This class of vehicles is susceptible to ocean current during deployment. 
%Thus, in order to determine energy-optimal paths through the ocean current field, real ocean data, available in netCDF format, are employed in combination with path planner, as A* and RRT especially \cite{rao2009large}, \cite{garau2009path}, \cite{SONG2017301}. 




%In marine industry natural environment has a bigger influence on the movement of the ship, even more if considering small USVs. Accounting for disturbances like winds and waves generated by atmospheric events and the motion of dynamic obstacles is primary.

%The available free traversal space in the maritime environment varies with time, and the map used to compute the path for USVs must be updated at the mission time. In \cite{shah2019long} a quadtree representation of the marine environment was used for speeding up A* search. Weather- and environmental-based spatial restrictions could be easily incorporated as obstacles in quadtrees.



%Underwater drones have to operate efficiently in a spatio-temporal, cluttered, and uncertain environment due to strong, time-varying currents. Z. Zeng et al. \cite{zeng2015efficient} presented a path re-planning methodology with the reuse of information gained from the previous planning history. The ocean was approximated by analytic equations with a Gaussian noise and was updated at every $\Delta_t$ value of time.
%\cite{zeng2020exploiting} focused on developing a path planner that incorporates ocean currents forecasts for optimizing energy saving. Predictive ocean modeling was used.

%Similar studies are addressed also for unmanned aircraft systems (UASs) which encounter evolving scenarios like wind fields. A data-driven decision solution for path planning was developed in \cite{he2019spatiotemporal} in which a stochastic influence model to generate spatio-temporal wind scenario data was used.


\section{Outline of the thesis}
The thesis is structured into six main chapters. After \autoref{intro}, the introduction, \autoref{second} recaps the path planning problem and the algorithms used and gives a brief background on marine weather forecasting. A statistical analysis of wave data is also included. \autoref{approach} describes the steps for developing the forecast map to perform the global path planning in the real environment.
% a real environmental scenario. 
\autoref{results} includes the simulations on which the algorithms and approaches are tested. \autoref{improvedmethod} concerns the generation of a better version of the spatio-temporal map.
Finally, \autoref{conclusions} concludes the thesis and discusses ideas for future developments.  % wave height to is dedicated to the path planning algorithms including also some theoretical contents about forecast, weather and marine data analysis. 


\chapter{Path planning and Weather Forecast}

\label{second}
This chapter presents an overview of navigation and path planning, focusing on the algorithms exploited in the thesis work.  Then, we move on to a theoretical background on marine weather forecasting and show how meteorological data can be considered and manipulated for application purposes.
    
\section{Path planning}
\textbf{Path planning} is a critical part required to perform autonomous navigation. Besides it, the system needs to know its position and orientation, \textbf{localization}, its surroundings, \textbf{mapping}, and how to control its motion, \textbf{path following}. If any of these technologies were missing, the system would not work correctly \cite{sakai2018pythonrobotics}.\\
Path planning is the ability of a robot to search for a feasible and efficient path by finding a series of way-points to navigate from the start to the end position in the configuration space. The path is generated to satisfy spatial constraints based on obstacle positions and specific optimization criteria like path length, time, safety, risk, and energy consumption \cite{vagale2021path}.

The \textit{configuration space} $\mathcal{C}$ consists of free space $\mathcal{C}_{free}$, which contains all possible configurations (states) of the mobile system and the space occupied by obstacles $\mathcal{C}_{obs}$.
%\subsection{path planning definition and other parts of autonomous navigation}
\subsection{Global and local planning}
The path planning problem can be divided into two types according to the known degree of environmental information:
\begin{itemize}[itemsep=0pt]
	\item \textit{Global} path planning;
	\item \textit{Local} path planning.
\end{itemize}
The \textbf{global} path planning method can generate an optimal path from the initial state to the goal state based on a \textit{priori} knowledge of the environment (the position and shape of static obstacles are predetermined). It comprises the environmental map modeling and the path planning strategy.

The \textbf{local} path planning method, also called dynamic re-planning, assumes that the position of the obstacles in the environment is unknown. Therefore, the vehicle perceives its surroundings and its state only through sensors \cite{LIU20201}. Based on the route made by the global planner, it recalculates the path, modifying the trajectory to avoid unknown and dynamic obstacles, including non-static environmental forces.
This path planning strategy needs to be adjusted in real-time and can result in a deviation from the previously planned path or a change in speed.
\subsection{Moving obstacle prediction}
%Path planning with predictions is a type of local path planning.
%Maneuvers around large static obstacles are first planned deliberatively using a global path planner while maneuvers around the smaller dynamic obstacles are later planned reactively using a local reactive trajectory planner.
In a dynamic environment, by the time the vehicle reaches a particular point, an obstacle may no longer be there, or a new obstacle may appear. As time passes, a path found previously may not be the optimal path anymore. So, a recalculation is necessary.

Typically, trajectory planning for USVs consists of a global path planning that plans the static obstacle avoidance and a local reactive trajectory planning for maneuvers around smaller dynamic obstacles. The Kalman filter can be a solution for predicting the ship's trajectory in this regard \cite{huang2020ship}.\\
However, in our case, the dynamic obstacles, i.e., wavefields that can affect the navigation, are extremely large in size and need to be taken into account by the deliberative planning to improve decision-making, optimize the survey mission and the route, and avoid eventual collisions.
%Predicting obstacle movement can be achieved embedding time in the pathfinding's cost function. However this modification is not perfect.
\section{Path planning algorithms} 
Based on Souissi \textit{et al.} \cite{souissi2013path}, path planning algorithms can take a \textit{classic} approach, an \textit{advanced} approach, or a \textit{hybrid} approach. The first class includes the algorithms that perform environmental modeling as a cell decomposition of the configuration space. Graph-based methods such as A* and Voronoi diagrams belong to this category.

The algorithms with an advanced approach are commonly used to deal with dynamic obstacles, path re-planning, and local collision avoidance. Examples are machine learning algorithms, potential field methods, evolutionary algorithms, and sampling-based algorithms. The last category includes algorithms that combine several path planning methods.

Choosing the appropriate algorithm helps to ensure safe and effective navigation. This choice depends on robot geometry and constraints, especially for nonholonomic systems. For this study, we decided to consider three different types:
\begin{itemize}[itemsep=0pt]
	\item A \textit{sampling-based algorithm} $\rightarrow$ RRT*;
	\item A \textit{potential field method} $\rightarrow$ Artificial potential field;
	\item A \textit{searching-based algorithm} $\rightarrow$ A*.
\end{itemize}
%The first two are classified as advanced algorithms because environmental modeling is not necessary (\cite{vagale2021path}, \cite{souissi2013path}). 
\subsection{Rapidly-exploring Random Tree}
RRT, the abbreviation of Rapidly-exploring Random Tree, is a sampling-based method introduced by LaValle \cite{lavalle1998rapidly} at the end of the 20th century for path planning purposes, specifically designed for problems that have nonholonomic constraints. 

It works by creating a search tree incrementally by using random samples in a defined state space leading towards the goal. The algorithm follows these steps starting from the initial node $\mathbf{x}_{init}$:
%RRT is the prime example of sampling-based algorithm. A search tree is created randomly by adding nodes in the state space following these steps:
\begin{enumerate}[itemsep=0pt]
	\item The planner samples a random node $\mathbf{x}_{rand}$ from the configuration space $\mathcal{C}$;
	\item It finds the closest node of the search tree to $\mathbf{x}_{rand}$, $\mathbf{x}_{near}$;
	\item The tree expands from $\mathbf{x}_{near}$ towards $\mathbf{x}_{rand}$ based on the expansion length, reaching $\mathbf{x}_{new}$;
	\item The connection between $\mathbf{x}_{new}$ and $\mathbf{x}_{near}$ is checked for collision considering the size of the obstacles $r$;
	\item If the path resides in $\mathcal{C}_{free}$, $\mathbf{x}_{new}$ is added to the tree. Otherwise, it is discarded;
	\item The next iteration of the loop starts.
\end{enumerate} 
Progress in the direction of the goal is guaranteed by adding a bias factor, which means that the random node is the goal state with a probability based on a bias coefficient. Figure \ref{rrt} shows the tree with the most significant points.
\begin{figure}[H]
	\centering
	\includegraphics[width=75mm]{Figures/rrt2.png}
	\caption{RRT algorithm}
	\label{rrt}
\end{figure}

RRT's improved version, RRT* \cite{karaman2011sampling}, can deliver the shortest possible path to the goal when the number of nodes approaches infinity. The basic principle is the same, but it introduces two new features which permit to obtain a shorter and smoother path:
\begin{itemize}
	\item \textit{Near neighbor search}: RRT* records the distance each node has traveled relative to its parent node, i.e., the cost of each node. When $\mathbf{x}_{new}$ is found, a neighborhood of nodes within a fixed ball radius from the new node is examined, and the node with the cheapest cost replaces the closest node;
	\item \textit{Rewiring tree}: when $\mathbf{x}_{new}$ is connected to the cheapest neighbor, the neighbors are again examined to see if connecting $\mathbf{x}_{new}$ with one of these points is convenient. If so, $\mathbf{x}_{new}$ becomes the parent of a pre-existing node.
\end{itemize} 

The drawback is that it suffers from a reduction in performance. Due to the new two features, it takes much more time than the default version.
\subsection{A*}
A* is the most well-known pathfinding algorithm. It was developed in 1968 based on Dijkstra's algorithm and the GBFS (Greedy Best-First-Search) algorithm \cite{hart1968formal}.
\begin{itemize}[itemsep=0pt]
	%\item BFS calculates the shortest path in a unweighted graph reducing the number of visited edges
	\item Dijkstra's algorithm prioritizes the node with the lowest cost, the node's distance from the source. It guarantees to find the shortest path;
	\item GBFS uses a heuristic that estimates how far from the goal any node is, ignoring the path's cost so far. It runs much quicker than Dijkstra's algorithm but does not guarantee to find the shortest path.
	%It selects the node closest to the goal at that time. 

\end{itemize}
A* works by creating weighted graphs connecting the start and the goal nodes. It overcomes the other two by combining their features: the heuristic $h(n)$, which gives additional information about the cost from node $n$ to goal, and the actual cost of the path $g(n)$. The heuristic underestimates the true cost generally using Manhattan distance or Euclidean distance. \\
%A* explores each node based on the cost function 
By minimizing $f(n)=g(n)+h(n)$, A* finds the path with the least cost. \\
However, it does not give optimality when applied with a grid.

%Hence, it perform an informed rather than uninformed search: the heuristic gives A* additional information about the cost from node $n$ to goal. 
%where $n$ is the next node on the path, $g(n)$ the actual cost of the path from the start to node $n$, and $h(n)$ the heuristic.
%use a queue to en-queue each path, looking at the neighbour with distance 1 and then distance 2 until the path is found (guaranteed the shortest). It , , which calculates the same thing in a weighted path a that uses a priority queue which will take the path that is ebst at a given time, 
%It belongs to the category of Cell decomposition methods \cite{souissi2013path}} for which the first essential step is to model the environment before searching the optimal or feasible path.
%Unlike Dijkstra priority is the distance of the node from start. The ideal priority should be distance + future cost. When there are obstacles, it is difficult to know the future cost. And so we use an heuristic which is an underestimate of the future cost. We are ingoring paths that in the best case are worse than the current path. 
%one of the most efficient and popular graph search algorithm for finding paths in discrete grid maps.
%This type of algorithm is appropriate when the vehicle can be considered as a point and no motion model or kinematic equation is involved in the planning stage
\subsection{Artificial Potential Field method}
The potential field method considers the configuration space $\mathcal{C}$ as a potential map $\bm{q}=(q_1,q_2)$ where the goal $\bm{q}_g$ has the lowest potential and the starting node the maximum one \cite{khatib1986real}. The total potential $U$ is the sum of an attractive and a repulsive potential, whose negative gradient $-\nabla U(\bm{q})$ indicates the most promising local direction of motion.
\begin{equation}
	U(\bm{q})=U_{att}(\bm{q}) + U_{rep}(\bm{q})
\end{equation}
The attractive potential has the objective of guiding the robot to the goal $\bm{q}_g$. Two possible functions are normally implemented: quadratic (or paraboloidal) and conical potential. 
The quadratic potential works better in the proximity of $\bm{q}_g$, but increases indefinitely with the distance. A convenient solution is to combine the two profiles: conical far from $\bm{q}_g$ and quadratic close to $\bm{q}_g$. \\
In order to maintain the continuity of the function, the final potential is defined as:
\begin{equation}
	U_{att}(\bm{q})=
	\begin{dcases}
		\frac{1}{2}\zeta d^2(\bm{q},\bm{q}_{goal}), & d(\bm{q},\bm{q}_{goal})\leq d^*_{goal} \\
		d^*_{goal}\zeta d(\bm{q},\bm{q}_{goal}) -\frac{1}{2}\zeta(d^*_{goal})^2,  & d(\bm{q},\bm{q}_{goal}) > d^*_{goal} 
	\end{dcases}
\end{equation}
where $\zeta$ is the attractive coefficient, and $d^*_{goal}$ the threshold distance where the potential function changes. Figure \ref{potentialfield} displays the differences between these functions.
\begin{figure}[h]
	\centering
	\subfloat[Paraboloidal]{\includegraphics[width=0.4\textwidth]{Figures/potential1.png}\label{quad}} 
	\hspace{0.2cm}
	\centering
	\subfloat[Conical]{\includegraphics[width=0.4\textwidth]{Figures/potential2.png}\label{conic}}
	\hspace{0.2cm}
	\subfloat[Paraboloidal + conical ]{\includegraphics[width=0.4\textwidth]{Figures/potential3.png}\label{quad+conic}}
	\caption{Standard functions of the attractive potential} 
	\label{potentialfield}
\end{figure}

The objective of the repulsive potential is to keep the robot away from $\mathcal{C}_{obs}$. Each obstacle has a repulsive field which decreases with the distance, defined as follows:
\begin{equation}
	U_{rep}(\bm{q})=
	\begin{dcases}
		\frac{1}{2}\eta \left(\frac{1}{D(\bm{q})}-\frac{1}{Q^*}\right)^2 & D(\bm{q})\leq Q^* \\
		0 & D(\bm{q})> Q^*
	\end{dcases}
\end{equation}
where $\eta$ is the repulsive coefficient, $D(\bm{q})$ the distance of the robot from the obstacle, and $Q^*$ its size, i.e., the radius of influence.

%\begin{equation}
%	\nabla U_{att}(q)=
%	\begin{dcases}
%		\zeta(q-q_{goal}), & d(q,q_{goal})\leq d^*_{goal} \\
%		\frac{d^*_{goal}\zeta(q-q_{goal})} {d(q,q_{goal})}  & d(q,q_{goal} > d*_{goal} 
%	\end{dcases}
%\end{equation}

%\begin{equation}
%	\begin{split}
%		\nabla U_{att}(q)=\nabla (\frac{1}{2}k_{att}d^2(q,q_{goal}))=\\
%		& \frac{1}{2}\nabla d^2(q,q_{goal})=\\
%		& k_{att}d(q,q_{goal})
%	\end{split}	
%\end{equation}

%\begin{equation}
%	\nabla U_{rep}(q)=
%	\begin{dcases}
%		\eta \left(\frac{1}{Q*}-\frac{1}{D(q)}\right) \frac{1}{D^2(q)}\nabla D(q) & D(q)\leq Q* \\
%		0 & D(q)> Q*
%	\end{dcases}
%\end{equation}

The limitation of this algorithm is that it can get trapped easily in local minima, points where $-\nabla U(\bm{q}_m)=0$, primarily related to weight coefficients associated with each obstacle during the APF design.\\
An approach to dealing with the minima trap replaces the repulsive action with tangential potential fields. This solution is possible using the limit cycle methodology and assigning a force field directly rather than a potential. 

A \textit{limit cycle} is an isolated periodic orbit having the property that at least one other trajectory spirals into it for $t\rightarrow\pm\infty$ (Figure \ref{examplecycle}).\\
%(see Equation \ref{eqvortex}).
%A way to partially solve this problem is to create convex obstacles. (Principles of Robot Motion H. Choset et.al. Mit Press).\\
%\begin{equation}\label{eqvortex}
%	f_v = \pm \binom{\frac{\partial U_{r,i}}{\partial y}}{-\frac{\partial U_{r,i}}{\partial x}}
%\end{equation}
For this study, we relied on the algorithm proposed by \textit{Lounis Adouane} \cite{adouane:hal-01717955}, which uses the limit cycle characteristics of a 2nd-order nonlinear function to perform the obstacle avoidance behavior and overcome the local minima problem:\\
\begin{minipage}{.5\linewidth}
	\begin{align*}
		\dot x_s &=y_s+ x_s(R_c^2 - x_s^2 - y_s^2) \\
		\dot y_s &= -x_s + y_s(R_c^2 - x_s^2 - y_s^2)  
	\end{align*}
\end{minipage}
\begin{minipage}{.5\linewidth}
	\begin{align}
		\begin{split}
			\dot x_s &=-y_s+ x_s(R_c^2 - x_s^2 - y_s^2) \\
			\dot y_s &=x_s + y_s(R_c^2 - x_s^2 - y_s^2)  
		\end{split}
	\end{align}
\end{minipage}\\
\vspace{0.3cm}\\
The left equation indicates the clockwise trajectory motion and the other counter-clockwise direction; $R_c$ is the obstacle radius. \\
At each instant of time, it takes the closest obstacle from the potentially disturbing ones, with distance $(x_s,y_s)$,  and calculates the direction of the vector field $(\dot x_s,\dot y_s)$ based on its position to the goal. An adjustable coefficient determines the force value.
%is a closed trajectory in phase space having the property that at least one other trajectory spirals into it either as time approaches infinity or as time approaches negative infinity.
\vspace{0.5cm}
\begin{figure}[H]
	\centering
	\includegraphics[width=70mm]{Figures/limitcycle2.png}
	\caption{Example of limit cycle}
	\label{examplecycle}
\end{figure} 
\newpage
\subsection{Algorithm settings}
Starting from widespread open-source codes\footnote{\url{https://github.com/AtsushiSakai/PythonRobotics}}$^,$\footnote{\url{https://github.com/ShuiXinYun/Path_Plan}}, the three path planning algorithms have been adequately modified to adapt to the case study and make a comparison on an equal footing. Given the environmental files, which we will discuss later,  in array form, we employed their grid in latitude-longitude coordinates for A* and APF to have more realistic simulations.
\subsubsection{RRT*}
This algorithm bases its operation on computing distances and angles to move along the tree. Consequently, we preferred to use point coordinates ($x,y$) converted in meters instead of the georeferenced system to avoid unnecessary complications. The main adjustments were:
\begin{itemize}[itemsep=0pt]
	%\item Random nodes are selected from the points of the grid, used by A* and APF, to reduce the computing time;
	\item The cost function returns the geodetic distance, as A* and APF do, despite the approximation of the sea surface by 2D plane;
	\item Obstacles radius $r$ and expansion length $l$ are both equal to 4.5 $km$, which corresponds to the approximated conversion of the grid resolution (0.042°) along the longitude used by A* and APF;
	\item Ball radius is set as $150\sqrt{\frac{ln(n)}{n}}$ $km$, where $n$ is the length of the path nodes list. By increasing the ball radius, the algorithm will examine a larger neighborhood of nodes, making the path smoother at the expense of a significant increase in computing time (see Figure \ref{changinggamma});
	% \textcolor{red}{spiegare impatto sul planner, conseguenze su tempo di computaz, etc}
\end{itemize}
\begin{figure}[h]
	\centering
	\subfloat[Default ball radius: $t_{exec}=1.07$ $s$]{\includegraphics[width=0.45\textwidth]{Figures/rrtexam1.png}\label{Defaultballradius}} 
	\hspace{0.2cm}
	\centering
	\subfloat[Bigger ball radius $t_{exec}=7.18$ $s$]{\includegraphics[width=0.45\textwidth]{Figures/rrtexam2.png}\label{biggerradius}}
	\caption{Path difference changing the ball radius} 
	\label{changinggamma}
\end{figure}
%Normally, with the defaults parameters, this algorithm would perform worse than the other two. Therefore, ball radius was properly tuned to create much smoother paths, at the expense of the computing time.
\subsubsection{A*}
%A* works when the environment does not change. It is not designed for dealing with moving obstacles.
A* is grid-based and uses the $62\times47$ point grid in georeferenced coordinates. Hence, the grid size is set to $0.04166$° as well as for robot dimensions. Besides that, the heuristic $h(n)$ uses the geodetic distance, i.e., the shortest distance on the surface of an ellipsoidal model of the earth (computed with the \texttt{geopy} library).
\subsubsection{APF}
As with A*, the grid size is set to $0.04166$°. Instead, the radius of influence of the obstacles $R_c=2\cdot0.04166$° to avoid the vessel going through two nearby points.

\section{Algorithm comparison}
Once set all the parameters, we tested the three path planning algorithms, i.e., RRT*, A*, and APF, in 3 scenarios of area $100\times80$ km to evaluate their characteristics. The obstacles are represented by their shape, as will be done in the real simulations. 
\begin{itemize}[itemsep=0pt]
	\item \textbf{Scenario 1} represents a complex environment of islands;
	\item \textbf{Scenario 2} includes two small slits through which the robot must go to reach the goal;
	\item \textbf{Scenario 3} represents the U-shaped obstacle adapted to the marine environment.
\end{itemize} 

Figure \ref{basiccomparison} shows the resulting paths in the three scenarios. Table \ref{toyalgotable} displays the travel time $T_{path}(h)$, the mean of the three tests, and the variation from the fastest algorithm. The same results are present in the form of a barplot in Figure \ref{toyalgotraveltime}. 

We will mainly focus on the travel time because the computing time is negligible compared to the time interval of the survey mission. The time spent for any re-planning of the global route cannot affect the vessel's navigation.

Results show that the best algorithm in terms of path length is RRT* with a mean $T_{path}=7.1$ $h$, followed by A*, 2\% slower, and APF with a 17\% longer mean travel time. The goal attraction influences the potential field method, resulting in sub-optimal paths.

Here we have seen only static obstacles. In order to deal with moving ones represented by adverse marine weather conditions, we first need to introduce the topic of sea forecasts. 
\vspace{0.3cm}
\begin{figure}[h]
	\centering 
	\subfloat[Scenario 1]{\includegraphics[width=0.47\textwidth]{Figures/toyalg1.png}\label{sc1}}
	\hspace{0.2cm}
	\subfloat[Scenario 2]{\includegraphics[width=0.47\textwidth]{Figures/toyalg2.png}\label{sc2}}
	\hspace{0.5cm}
	\subfloat[Scenario 3]{\includegraphics[width=0.47\textwidth]{Figures/toyalg3.png}\label{sc3}}
	\vspace{0.0cm}
	\caption{Algorithm comparison} 
	\label{basiccomparison}
\end{figure}
\newpage
% Table generated by Excel2LaTeX from sheet 'Foglio1'
\begin{table}[h]
	\centering
	\begin{tabular}{|c|c|c|c?c|c|}
		\hline
		& \multicolumn{3}{c?}{\centering $T_{path}$($h$)} & \multirow{2}{*}{\parbox{3em}{\centering \small Mean $T_{path}$($h$)}} & \multirow{2}{*}{\parbox{4em}{\centering \scriptsize Variation (\%) w.r.t. $min(T_{path})$}} \bigstrut\\
		\cline{1-4}    Algorithm & Scenario 1 & Scenario 2 & Scenario 3 &       &  \bigstrut\\
		\hline
		\textbf{RRT*} & \textbf{6.18} & \textbf{8.99} & \textbf{6.05} & \textbf{7.1} & \textbf{0.0} \bigstrut\\
		\hline
		\textbf{A*} & 6.18  & 9.05  & 6.41  & 7.2   & 2.0 \bigstrut\\
		\hline
		\textbf{APF} & 7.09  & 9.91  & 7.76  & 8.3   & 16.7 \bigstrut\\
		\hline
	\end{tabular}%
	\caption{Quantitative results of the algorithm comparison}
	\label{toyalgotable}%
\end{table}%
\begin{figure}[h]
	\centering
	\includegraphics[width=75mm]{Figures/toyalgtraveltime.png}
	\caption{Comparison of travel time of the three algorithms}
	\label{toyalgotraveltime}
\end{figure} 
\vspace{0.3cm}
\section{Marine weather forecasting}
Marine meteorology is a subfield of meteorology that deals with the weather and climate and the associated oceanographic conditions in the marine, island, and coastal environment.
Marine meteorologists' job is to support marine and coastal activities, including but not limited to shipping, fishing, tourism, offshore oil drilling and mining operations, offshore wind and tidal energy harvesting, search and rescue at sea, and naval operations \cite{XIE2015287}. \\
Marine weather programs were born towards the end of the 19\textsuperscript{th} century to determine requirements for safer ocean voyages and to avoid maritime disasters.
\newpage
\subsection{Weather prediction system and USV safety}
A weather prediction system collects quantitative data about the current atmospheric and oceanic states, and predicts how they will evolve based on a scientific understanding of the atmospheric, oceanic, and air-sea interaction processes.\\
Data acquisition is made through \textit{in situ} and ground-based observations, like weather ships and weather buoys, and with remote sensing from satellites of both the atmosphere and the sea surface.

A marine weather forecast system typically consists of a suite of weather, ocean circulation, and forecast models of sea surface waves. The sea surface wave model, which will be mentioned again later, is driven by surface winds from an atmospheric model and provides forecasts for sea states, including significant wave heights and mean wave periods \cite{XIE2015287}. For local weather phenomena, it is better to rely on regional models: besides operating at finer resolutions, they provide the most benefits in areas where factors influence local weather, such as islands, coastlines, topography, and land coverage. The perfect example is the Mediterranean weather.

%Regional model are better, especially in the places where usvs operate.In places where there are local weather phenomena due to mountains, islands, coastlines, topography and land coverage it's better to use regional models. Mediterranean weather is a true example for which it is recommended to consult regional models. regional models run over limited regions and manage to operate at finer resolutions down to 1 km.
% \subsection{Forecasts for USV safety}
While sailing in open water, an autonomous surface vessel, as well as crewed boats, especially of small size, faces many risks due to harsh weather conditions, which can affect the safety of the vehicle, travel time, and energy consumption \cite{vagale2021path}.
Before planning a survey mission, scenarios such as bad weather situations or adverse ocean currents in the region of interest need to be considered.  

The USV should be envisioned as an agent making strategic exploration decisions: it can take advantage of weather forecasts to plan missions, select targets, optimize routes by avoiding opposing currents and riding helpful ones, and plan energy efficient collision-free paths \cite{krell2020autonomous}. Therefore, we need to provide precise forecasts of the expected environmental disturbances (wind, waves, and current) to enrich the planning information system. 

The thesis idea to face this problem is to use \textbf{NetCDF weather forecast data}, which can be easily handled in Python and integrated into whichever algorithm. In this study, as wave forecasts are driven by marine wind fields \cite{NICLASEN20101, XIE2015287}, and by assuming negligible surface currents for the boat speed, we will explicitly focus on waves.    
\newpage
%\begin{figure}[H]
%	\centering
%	\includegraphics[width=80mm]{Figures/sailingcoper.jpg}
%	\caption{Example of small vessel}
%	\label{sailing}
%\end{figure}
\subsection{NetCDF format}
NetCDF, the acronym that stands for \textit{Network Common Data Form} is a set of software libraries, machine-independent data formats, and a community standard for sharing scientific data. It is commonly used in climatology, meteorology, oceanography, and GIS applications. The data in a NetCDF file is stored in the form of multidimensional arrays in georeferenced coordinates \cite{unicar} (Figure \ref{netcdf}). \\
We used the Python package \texttt{xarray} that is particularly tailored to working with this type of data.

\begin{figure}[h]
	\centering 
	\subfloat[Three-dimensional data example]{\includegraphics[width=0.6\textwidth]{Figures/netcdf3.png}\label{3ddata}}
	\hspace{0.5cm}
	\subfloat[Four-dimensional data example]{\includegraphics[width=0.6\textwidth]{Figures/netcdf4.png}\label{4ddata}}
	\caption{Storage of NetCDF data} 
	\label{netcdf}
\end{figure}
Copernicus Marine Service (CMS) provides free and open marine data and services to enable marine policy implementation and support Blue growth and scientific innovation. These data include hindcasts, nowcasts, and forecasts, delivered in NetCDF format (.nc).
%Since my research project is focused on path planning and decision-making of autonomous surface vessels depending on marine weather, I decided to start from these data for modeling obstacles in the configuration space.
%In this research project, I mainly made use of 3 Copernicus dataset:
%\begin{itemize}[itemsep=0pt]
%	\item Mediterranean Sea Physics Analysis And Forecast
%	\item Mediterranean Sea Waves Analysis And Forecast
%	\item Mediterranean Sea Biogeochemistry Analysis And Forecast
%\end{itemize}
%Focusing on Mediterranean Sea we find that
%I considered as regional domain the Mediterranean Sea for obvious reasons, and also because it is very well monitored and many data are provided, like physics, waves, biochemistry analysis and forecasts, composed by hourly parameters updated daily.
The dataset, \textit{Mediterranean Sea Waves Analysis And Forecast}, contains information ranging from 2019-05-04 to the present, with a spatial resolution of $0.042$°$\times0.042$°, an hourly instantaneous temporal resolution, and is updated two times a day.
%(older information are in addition to the Reanalysis datasets, to deliver a complete and consistent picture of the past weather).
  %Information in wave dataset refers to surface only.
Wave data are produced by the Mediterranean Forecasting System (MFS), a numerical ocean prediction service that does analyses, reanalyses, and short-term forecasts for the entire Mediterranean Sea, based on the upgraded spectral wave model WAM Cycle 4.6.2. \\
The wave system includes 2 forecast cycles providing a Mediterranean wave analysis twice per day and 10 days of wave forecasts \cite{copernicus}.
\section{Significant wave height}
%(https://marine.copernicus.eu/services/use-cases/sailgrib-ship-routing-application)
Ocean data include a wide variety of parameters that cannot be considered all at once. Therefore, we took into account only the height of the waves, whose most important specification is the significant wave height, available with the acronym \texttt{VHM0}. This sea state product is critical for all maritime safety and rescue operations \cite{esa}.

%Sea dynamics can be summarized in 3 main factors that need to be forecast for safety of vessel: sea wind, waves and current. Each of them is composed of many parameters to be accounted \cite{niclasen2010wave}.
%Wind load is generally ignored in path planning since USVs have a high draft compared to an air projection area and operations are generally restricted in an environment with wind speed less than 10 m/s \cite{lee2015energy}.
%Wave forecasting is conducted by the third-generation wave models, like WAM ("WAve Model"), WW3 ("WaveWatch III") and SWAN ("Simulating WAves Nearshore") after given them wind forecasts.
Sea Surface Wave Significant Height (SWH, or $H_s$) is defined as \textit{a statistic computed from wave measurements and corresponds to the average height of the highest one third of the waves, where the height is defined as the vertical distance from a wave trough to the following wave crest} (definition by CF Standard Names \cite{cmip6}, expressed in Equation \ref{eq1}).
\begin{equation}\label{eq1}
	H_{1/3}=\frac{1}{N/3}\sum_{i=1}^{N/3}H_i
\end{equation}
$N$ is the number of individual wave heights, and $H_i$ is a series of wave heights ranked from highest to lowest. Looking at Figure \ref{wave}, which shows the statistical distribution of ocean wave height\footnote{\href{https://media.bom.gov.au/social/blog/870/ruling-the-waves-how-a-simple-wave-height-concept-can-help-you-judge-the-size-of-the-sea/}{Bureau of Meteorology}}, we can state that the most common waves are lower than $H_s$, but it is possible to encounter a wave that is much higher than the significant wave height.
Nowadays, the more modern definition considers the frequency-domain analysis and is defined based on the zero moment, $m_0$, which is the area under the energy density spectrum curve \cite{BAI201673}.
\begin{equation}\label{eq2}
	H_{m0}=4\sqrt{m_0}
\end{equation}
%based on frequency-domain analysis it is usually defined as four times the standard deviation of the surface elevation – or equivalently as four times the square root of the zeroth-order moment (area) of the wave spectrum. 
%In general, it includes both the waves formed by local wind and swell waves. 
%It can be calculated with the following equation:
%\begin{equation}
%	H_{m0}=4\sqrt{m_0}=\sqrt{H_{wind}^2+H_{swell}^2}
%\end{equation}
%https://marine.copernicus.eu/news/new-satellite-wave-product-released)
%https://sentinels.copernicus.eu/web/sentinel/user-guides/sentinel-3-altimetry/overview/geophysical-measurements/significant-wave-height
%\begin{itemize}
%	\item https://www.stormgeo.com/products/s-suite/s-routing/articles/why-douglas-sea-state-3-should-be-eliminated-from-good-weather-clauses/
%	\item https://it.wikipedia.org/wiki/Scala\_Douglas
%	\item https://www.google.com/search?channel=nrow5\&client=firefox-b-d\&q=significant+wave+height+vessels
%	\item https://www.meteorologiaenred.com/en/douglas-scale.html
%\end{itemize}
 
As we will discuss in \autoref{approach}, we considered the Tyrrhenian Sea as the navigation area. By manipulating the NetCDF files, we analyzed historical wave data for 2021 to see how the sea changes throughout the seasons and understand which is the best period to carry out missions at sea.\\
We started by plotting the annual mean significant wave height, displayed in Figure \ref{annual}. It can be seen that the coast of Corsica shelters the region of interest, and higher values are concentrated in the upper and lower part of the black perimeter. %through a data resampling. 
\begin{figure}[H]
	\centering
	\includegraphics[width=70mm]{Figures/wavespectrum.jpg}
	\caption{Wave statistical distribution}
	\label{wave}
\end{figure}
\begin{figure}[H]
	\centering
	\includegraphics[width=100mm]{Figures/annualmeanregofint2.png}
	\caption{Annual mean significant wave height in 2021}
	\label{annual}
\end{figure}
By adding the mean wave direction (\texttt{VMDR}) of 2021, expressed in degrees from due north, we examined more in detail significant spots for routing, marked with black dots in the previous figure, through the wave rose. Meteorologists use this type of diagram to view how wave height and direction are distributed at a particular location over a specific period. The length of each spoke shows how often waves come \textit{from} that direction, expressed in percentage of occurrence, while the different colors provide information about the height.\\
From Figure \ref{waverose}, we can see that most of the waves come from the South in all four cases, meaning that the vessel should pay particular attention to that exposed side of the boat. Higher waves occur mainly at Montecristo and Elba Island, the two points farther out to sea.
%Most of the waves are generated from the South  The length of each color spoke stands for the percentage of the waves arriving from that particular direction. From the plots the majority of waves come from south east Gives information which are the wave height, the direction of the waves and the percentage of occurrence
\begin{figure}[h]
	\centering 
	\captionsetup[subfloat]{oneside,margin={0.2cm,2cm}}
	\subfloat[Harbour of Porto Vecchio]{\includegraphics[width=0.45\textwidth]{Figures/waverosestart.png}\label{waverose1}}
	\hspace{0.3cm}
	\captionsetup[subfloat]{oneside,margin={0.4cm,2cm}}
	\subfloat[Montecristo Island]{\includegraphics[width=0.45\textwidth]{Figures/waverosemezzo.png}\label{waverose2}}
	\hspace{0.3cm}
	\captionsetup[subfloat]{oneside,margin={0.4cm,2cm}}
	\subfloat[Argentario]{\includegraphics[width=0.45\textwidth]{Figures/waveroseend1.png}\label{waverose3}}
	\hspace{0.3cm}
	\captionsetup[subfloat]{oneside,margin={0.5cm,2cm}}
	\subfloat[Elba Island]{\includegraphics[width=0.45\textwidth]{Figures/waveroseend2.png}\label{waverose4}}
	\caption{Wave rose of significant spots} 
	\label{waverose}
\end{figure}

After that, we computed a monthly and daily average of the bounded region, obtaining the charts in Figure \ref{mean}. Instead, the histogram in Figure \ref{histogram} shows the distribution of $H_{m0}$ and its kernel density estimation, which estimates the probability density function of the sea parameter. The number of the waves has been normalized, showing the percentage.

Taking as reference Douglas scale, the so-called ``international sea and swell scale'', whose purpose is to estimate the roughness of the sea for navigation, the sea in this region can be categorized as \textit{slight} (3\textsuperscript{rd} degree of Douglas scale), as the annual mean of $0.84$ $m$ lies between $0.5$ and $1.25$ $m$. 
\newpage
\begin{figure}[H]
	\centering 
	\subfloat[Monthly mean significant wave height ]{\includegraphics[width=0.8\textwidth]{Figures/monthlymean.png}\label{monthly}}
	\hspace{0.5cm}
	\subfloat[Daily mean significant wave height ]{\includegraphics[width=0.8\textwidth]{Figures/dailymean.png}\label{daily}}
	\caption{Mean significant wave height in 2021} 
	\label{mean}
\end{figure}
%\begin{figure}[H]
%	\centering 
%	\subfloat[Number of observation for wave height]{\includegraphics[width=0.45\textwidth]{Figures/histogramhourlymean.png}\label{histdaily}}
%	\hspace{0.5cm}
%	\subfloat[Number of observation for wave height]{\includegraphics[width=0.45\textwidth]{Figures/histogramdailymean.png}\label{histhourly}}
%	\caption{Mean significant wave height in 2021} 
%	\label{hist}
%\end{figure}
\begin{figure}[H]
	\centering
	\includegraphics[width=110mm]{Figures/histogramhourlymean.png}
	\caption{Significant wave height distribution}
	\label{histogram}
\end{figure} 
\noindent
The best conditions are present during summer, where the mean value reaches a minimum of $0.5$ $m$ while the roughest sea is found in winter, particularly in January, where the significant wave height reaches a monthly peak of $1.3$ $m$, most probably due to seasonal strong winds and storms.
Looking at Figure \ref{daily}, we can assume that other factors influence the sea conditions besides the time of year. 
% I assumed to 0.8 m the threshold value of SWH, belonging to the 3 degree of the scale, slight. 
%We can state that the season represents an important factor on sea waves leading to a calm sea
%We can observe that higher values occur in colder months, while they are smaller in summer. The reason is most probably due to the atmospheric pressure, which produces respectively rain, wind and storms on the one hand, and calm and no precipitations on the other. \\
%From , we can see the changes throughout the months. $H_{m0}$ can vary by over a meter in a month, leading to the conclusion that other factors come into play during this process. An idea is the influence that Lunar phases have on tidal waves. 
%By noting that in a month the height can also change of 1.5 m (like in January) it's seems to be other factors besides the season. Most probably this is due to the tidal waves, bigger in the Lunar Phase of New Moon and Full Moon.
%Finally in the last figure the distribution of wave values with their frequency. There are many wave between 0.4 and 0.8 m. In fact the annual mean is $H_{m0}=0.84m$.
\section{Summary}
This chapter describes the path planning problem and the three algorithms RRT*, A*, and APF. Tests were conducted in three different scenarios to evaluate their characteristics and have shown a better behavior of the first two algorithms than the potential field method to create a global path.

When performing autonomous navigation at sea, the time-varying marine environment represents a crucial aspect to consider.
Therefore, we introduce the field of marine meteorology and understand the importance of receiving good marine forecasts during navigation.
An analysis of the significant wave height, the parameter on which this study will base, and the sea conditions in the Tyrrhenian Sea is reported.
% June is not significant as the mean wave height is always smaller than 0.8m.\\


\vspace{1cm}
\begin{comment}
	Copernicus data provide many other useful products about Mediterranean Sea. Besides the dataset on waves, sea physics is kept under observation and biogeochemistry too. For example, in the figure \ref{chloro} and \ref{zoo} the mass concentrations of Chlorofill and Phytoplankton Biomass in sea water are displayed.
	\begin{figure}[H]
		\centering
		\includegraphics[width=90mm]{Figures/Chlorofill.jpg}
		\caption{Chlorofill}
		\label{chloro}
	\end{figure} 
	\begin{figure}[H]
		\centering
		\includegraphics[width=90mm]{Figures/Zooplankton.jpg}
		\caption{Zooplankton}
		\label{zoo}
	\end{figure} 
\end{comment}





\chapter{Spatio-temporal map generation}
\label{approach}
%\vspace{5cm}

This chapter describes the approach used to deal with weather forecasts and, in general, moving obstacle prediction. The configuration space and the obstacles of a real environment are defined, and the spatio-temporal approaches are presented.\\
The entire methodology is based on the NetCDF files described in the previous chapter. 
%Before moving on to the resuThen RRT*, A* and the potential field method are presented and compared in 3 basic scenarios to evaluate their performance. Then, time windows approaches are presented and tested in non-real scenarios. Finally, the method is described in the various steps to build a real environment scenarios and to perform path planning simulations.
\section{Description of the environmental modeling}
The map generation needed to perform real path planning simulations can be summarized in 3 steps:
\begin{enumerate}[itemsep=0pt]
	\item Definition of the configuration space $\mathcal{C}$;
	\item Embedding of environmental data;
	\item Development of time window approaches for path planning.
\end{enumerate}
\subsection{Configuration space}
Witted has the vision of monitoring the biodiversity of the Italian coasts with autonomous drones, with a strong interest in the protection of \textit{Posidonia oceanica}, aquatic plants protected internationally. Hence, we decided to focus on the Mediterranean sea.

For the choice of the simulation area, different factors have been considered. Starting from \textit{Posidonia oceanica}, we could examine a georeferenced dataset in QGIS about the presence of this plant along the Mediterranean coast. The other environmental factor was the requirement of clean and shallow water to facilitate underwater surveys, which was verified through an online search and analysis of sea bathymetry.
Regarding the dimension of the sea region, we considered the possibility to navigate it in a day and, above all, the necessity to take a large sea area to see changes in atmospheric factors. Finally, the presence of islands and coasts was considered for testing the algorithms in a more complex scenario.     

The choice fell on the area of the Tyrrhenian Sea, which includes the east coast of Corsica, the Argentario archipelago, the island of Elba, and the Tuscan coast. Figure \ref{mapspace} displays an image of the sea area on Google Maps, while Figure \ref{posidonia} the presence of \textit{Posidonia oceanica}, marked with the red color.

\begin{figure}[h]
	\centering 
	\subfloat[Simulation area from Google Maps]{\includegraphics[height=0.37\textwidth]{Figures/confspacemaps.png}\label{mapspace}}
	\hspace{0.3cm}
	\subfloat[Presence of \textit{Posidonia oceanica}]{\includegraphics[height=0.37\textwidth]{Figures/posidonia.png}\label{posidonia}}
	%\captionsetup[subfloat]{font=scriptsize,labelfont=scriptsize}
	\caption{Configuration space}
	\label{Configuration spac}
\end{figure} 

Using the default grid resolution of Copernicus NetCDF files, the area is represented by a grid of $62\times47$ points ($longitude\times latitude$), which is equal to just over $200\times200$ $km^2$.
% Sea factors are considered as  spatio-temporal obstacles that changes position every hour made by dangerous wave heights from NetCDF file.
% 41°35'41''N, 8°95'83''E and 43°27'08''N 11°50'00''E
%As described in detail in the previous chapter, NetCDF data are set of multidimensional array, and so points. Consequently, for semplicity  I took two point coordinates belonging to that set as extremes of the space:  (a matrix of $62\times47$ point coordinates). 
Figure \ref{startend} shows the configuration space in Python; maps are generated by the \textit{matplotlib basemap toolkit}, a library for plotting 2D data on maps in Python. As can be seen, three points are highlighted:
\begin{itemize}[itemsep=0pt]
	\item 1 starting point: the harbor of Porto Vecchio in the east Corsica
	\item 2 ending points: WWF Oasis of Orbetello in the Argentario and the island of Elba with the National Parc of the Tuscan Archipelago, chosen as survey areas for their naturalistic importance
\end{itemize}
%5 days were considered: 18-07-21, 23-08-21, 24-08-21, 10-09-21, 18-09-21 \textcolor{red}{search more days}
%Depending on the sea conditions, the vessel can decide which survey destination to chose at the beginning, or to decide to stay moored.
\begin{figure}[H]
	\centering
	\includegraphics[width=100mm]{Figures/conf_space2223.png}
	\caption{Configuration space with start and end points}
	\label{startend}
\end{figure} 
%Porto vecchio is a port in the east Corsica and has coordinates: 41°60'07''N 9°29'49''E. By analyzing wave height at this coordinate, the values are always smaller than the threshold.
\subsection{Obstacle definition}
In path planning, an obstacle is defined first of all by its position in space.\\
In this study, we have to distinguish between static and dynamic obstacles: the former is represented by islands and coasts, while the others by wave heights. Copernicus data are available in georeferenced coordinates, so their position will be specified in longitude and latitude.

\texttt{MED-MFC\_006\_017\_mask\_bathy.nc} and \texttt{med-hcmr-wav-an-fc-h.nc} are the two files used for this purpose. The first provides binary data of land and sea, useful to distinguish between navigable points from those not, but especially for reproducing the profile of the coasts. The second one includes wave forecast data and is exploited for representing dangerous waves.

The method used to define obstacles is the same for both cases and makes use of the function \texttt{binary\_erosion} of the module \texttt{morphology} of scikit-image, an image processing toolbox for SciPy.
%If the image is concave, the final profile will be the same.
%Ocean data are in georeferenced coordinates. Therefore, also the path planning will use this configuration. The discrete map, necessary for the simulations, is gridded with the default spatial resolution, $0.042$°$\times0.042$°, which corresponds approximately to $5$ $km$. 

%And so my resolution will be the default one given by the Copernicus dataset 0.042°x0.042° that  of movement that we can consider acceptable for my purpose. ($2\pi R_e \frac{\Delta lat}{360°}$). My configuration space becomes a gridded matrix of points. The grid size is approximated to 0.04166°, and so I need to assume robot radius at least equal to the resolution, in order to avoid that the boat can go between 2 points.
\subsubsection{Static obstacle}
From the mask of the landmass, we extracted the coordinates of the points of its profile and defined them as static obstacles. The only drawback of this method is that it neglects the islands with dimensions below the data resolution. 
%By knowing which points are sea and which land, I could extract the profile of the coastline, and model it as a static obstacle. 
\subsubsection{Dynamic obstacle}
Wave height is not always constant. Indeed the positions of the obstacles will vary according to the temporal resolution of NetCDF data, which is 1 hour.

By considering a vessel drone of small dimensions (max 3 m of length), and the categories of sea roughness seen in \autoref{second}, a wave height of $0.8$ $m$ has been assumed as the threshold above which the vessel can no longer handle the navigation. We obtained a binary multi-dimensional array $=1$ where the waves are high and $=0$ where they are lower than $0.8$. At that point, the function executes the erosion obtaining the contour of the unsafe zones. \\
The contour extraction was made to define obstacles and lower the computing time of the algorithm, which has to compute the distances from each obstacle. Analyzing an internal point of the unsafe area would be unnecessary since the vessel must not overcome its contour in any case. Figure \ref{staticdynamicobs} shows the obstacles obtained through this mask processing.
\begin{figure}[h]
	\centering 
	\subfloat[Italy mask]{\includegraphics[height=0.35\textwidth]{Figures/maskera.png}\label{mask}}
	\hspace{0.5cm}
	\subfloat[Land contour]{\includegraphics[height=0.35\textwidth]{Figures/staticobs.png}\label{staticcont}}
	\hspace{0.5cm}
	\subfloat[Unsafe area at hour $t_i$]{\includegraphics[height=0.35\textwidth]{Figures/dynobscost.png}\label{3ob}}
	\hspace{0.5cm}
	\subfloat[Unsafe area contour]{\includegraphics[height=0.35\textwidth]{Figures/dynobs.png}\label{4ob}}
	\vspace{0.0cm}
	\caption{Static and dynamic obstacle definition} 
	\label{staticdynamicobs}
\end{figure}
%Figure \ref{dynamicobs} displays the configuration space with obstacle points at four time windows. 
\subsection{Vessel model approximation}
In this study, the effect of vehicle dynamics will be neglected. For the scale of the path planning problem, the vessel is regarded as a point with a radius equal to the grid size  \cite{garau2009path, SONG2017301, xiong2020rapidly, zhou2020review}.
As a first approximation, the assumption is valid since the long travel distance reduces the necessity to study how the vehicle will follow the path. Besides, the dimension of the boat is much smaller than the dimension of the sea basin, and then the focus remains on path planning. 

Even if the dynamic model is not integrated, its kinematics is needed to deal with dynamic obstacles \cite{SONG2017301}. We assumed a constant velocity $v=5$ $m/s$ for simplicity, which corresponds to the average speed of a fishing boat\footnote{\href{https://shipfever.com/how-fast-can-a-boat-go/}{`How fast can a boat go?'}}. Even if autonomous surface vessels can navigate at smaller speeds, we took it as a reference value to facilitate the processing of the time windows, as will be evident in the following section. Once the method is validated, any speed can be set.

%At the moment, it does not need to know full details about how the boat will reach the destination. We should regard the drone as a particle and ignore other factors as its specific shape and equations of motion. So we take the assumption of route planning, but applied to a smaller scale. . One of the most widely used ways of environment modeling in Route Planning is Grid (cit The Review Unmanned Surface Vehicle Path Planning: Based on Multi-modality Constrain)

\section{Spatio-temporal approaches}
After defining the configuration space and implementing wave data, we may ask how to perform path planning in this environment and with this type of obstacles. The solution consists in adding time to the configuration space $\mathcal{C}_{space}$, which raises its dimensions from two to three and makes the problem more complex.

The boat can navigate the sea in which $H_{m0}<0.8$ $m$, the so-called $\mathcal{C}_{free}$. Unsafe areas $\mathcal{C}_{obs}$, i.e., where $H_{m0}>0.8$ $m$, are bordered by a contour of point obstacles that do not allow the vehicle to pass through. According to the significant wave height value, these dangerous zones hourly change position, but especially shape. So, their shift cannot be predicted with classical methods, and generating the shortest possible path can be more complicated than at first glance.

We thought about predicting their movements by building a set of temporal maps of obstacles representing the environment at a precise hour of the day, as illustrated in Figure \ref{temporalslice}. Depending on how these maps are used and merged, the path generated by the path planning algorithm will be different, as well as the probability of failing the planning or creating a path where the safety of the USV is not guaranteed.\\
%\begin{enumerate}
%	\item the following path planning will generate a different path
%	\item it has to assure that the boat will never finish in dangerous areas
%\end{enumerate}
\begin{figure}[H]
	\centering
	\includegraphics[width=55mm]{Figures/resultsprova.png}
	\caption{Wave forecast at subsequent hour instants}
	\label{temporalslice}
\end{figure} 
\noindent
Three spatio-temporal approaches have been designed: \textbf{\textit{Sum}}, \textbf{\textit{Global}}, and \textbf{\textit{Local}}. 
\begin{comment}
	\subsection{Algorithm comparison}
	The three algorithms examined are:
	\begin{itemize}[itemsep=0pt]
		\item Potential field method
		\item a searching-based method: A*
		\item a sampling-based method: RRT (and RRT*)
	\end{itemize}
	I compared the 3 path planning and collision avoidance algorithms using 3 different approaches in order to evaluate their performance on the same level.\\
\end{comment}
\subsection{Sum}
This approach merges hourly unsafe areas in a single forecast map, which means that all predictions are considered together during path planning. Since this method is highly conservative, the resulting path will not be the shortest but assures the vehicle's safety. The step procedure is described in the pseudo-code of Algorithm \ref{summing}.
\RestyleAlgo{ruled}
\SetKwComment{Comment}{/* }{ */}

\begin{algorithm}[h]
	\caption{\textbf{\textit{Sum}} approach}\label{summing}
	\KwData{\\
		$H_{m0}\gets array(time, latitude, longitude)$\;
		$N \gets$ \textit{number of time windows}\;
		$cost \gets$ \textit{boolean} $array(time, latitude, longitude)$;
	}
	\KwResult{$[(o_x,o_y)] \gets$ \textit{list of obstacle points}}
	\For{$i$ in range($N$))}{
		$cost$\texttt{[i]} $\gets H_{m0}$\texttt{[i]} $> 0.8$ $m$\;}
	\textit{cost sum} $\gets \sum_{i=0}^{N-1} cost$\texttt{[i]}\;
	\textit{wave contour} $=$ \texttt{np.logical\_xor}(\textit{\small cost sum,} \texttt{binary\_erosion}(\textit{\small cost sum}))\;
	\textit{land contour} $=$ \texttt{np.logical\_xor}(\textit{mask,} \texttt{binary\_erosion}(\textit{mask}))\;
	%\{\textit{extract contour of wave region from $sum\_cost$}\}\;
	%\{\textit{extract land contours from the mask}\}\;
	$[(o_x,o_y)]\gets \text{list of x and y coordinates of the obstacle contour}$
\end{algorithm}

\subsection{Global}
This method takes its name from \textit{global planning} and consists in generating a path step by step using only the information of the current time window in which the boat is. It can be seen as a global path planning limited to an hour's drive, so the environment is unknown outside that time. We can summarize it in 3 steps, which are repeated till reaching the destination:
\begin{enumerate}[itemsep=0pt]
	\item Generate the complete path by considering the time window of the current boat position;
	\item Shift the starting point to the path point reached in 1 hour of travel;
	\item Change the time window to that of the next hour.
\end{enumerate}
It simulates the real case where the boat plans a route without considering how weather changes, with the purpose not to meet any danger in the following hour. We described the entire process in Algorithm \ref{global}.
%As you can imagine, this method can be useful in some cases when the path is short or static. However not considering weather prediction leads to the possibility of being in "non-safe" zones at the edges of the time windows.
There are 2 cases in which the algorithm cannot find a path:
\begin{enumerate}[itemsep=0pt]
	\item  No feasible paths to reach the goal are available at a specific hour due to obstacle regions;
	\item During navigation, the boat ends up within an unsafe zone when the temporal map updates.
\end{enumerate}
In the algorithm, the first issue is worked out by moving the vessel for 1 hour along the previous path, generated with the time window $t_{i-1}$, till the new update. The second one cannot be controlled, implying that re-planning at hourly intervals without considering future obstacle predictions does not guarantee the method's feasibility, regardless of the results in terms of travel time.\\
Lastly, we remind that this planning, as for the other methods, is performed in advance, i.e., before the survey mission starts. As a result, possible collisions or sinkings prevent the vehicle from leaving, determining only an efficiency reduction and not the approach's safety.
\newpage
% in reality, as soon as the weather map changes, the USV could suddenly be in a dangerous sea area. \textcolor{red}{spiegare/risolvere}
\subsection{Local} 
The last developed approach is \textbf{\textit{Local}}, which gets its name from \textit{local planning} since it well approximates the behavior of a local planner to avoid dynamic obstacles. Moving obstacle prediction is addressed using a unique forecast map divided into circular time bands generated with the center at the starting point. Each annulus has a width that corresponds to 1 hour of ideal travel, $18$ $km$, and will include the corresponding unsafe contributions of that specific time window. The generation of the obstacle map is described in Algorithm \ref{local}, while Figure \ref{localobs} illustrates the concept.

The current method is based on the strong assumption that the USV will cross each temporal band in precisely 1 hour. However, this condition cannot be met because the vessel should always move in a straight line. Therefore, the further away the vehicle gets, the higher will be the safety risk since the information used does not correspond to the actual forecast conditions.
\begin{algorithm}[!htbp]
	\SetKwBlock{Begin}{Begin}{}
	\Begin{
		\caption{\textbf{\textit{Global}} approach}\label{global}
		\KwData{\\
			$H_{m0}\gets array(time, latitude, longitude)$\;
			$N \gets$ \textit{number of time windows}\;
			$cost \gets$ \textit{boolean} $array(time, latitude, longitude)$\;
			$(s_x, s_y)\gets start$ , $(g_x, g_y)\gets goal$\;
			$res\gets$ \textit{grid size};}
		\KwResult{Path points and list of re-planning way-points}
		\For{$i$ in range($N$))}{
			$cost$\texttt{[i]} $\gets H_{m0}$\texttt{[i]} $> 0.8$ $m$\;
			{\small\textit{ wave contour} $=$ \texttt{np.logical\_xor}(\textit{\small cost}\texttt{[i]}, \texttt{binary\_erosion}(\textit{\small cost}\texttt{[i]}))}\;
		}
	\For{$i$ in range($N$)}{
		\For{$k$ in range(len(latitude))}{
			\For{$j$ in range(len(longitude))}{
				\If{cost\texttt{[i][k][j]}}{\textit{add x and y coordinates to the map of obstacles}}
				\If{wave contour\texttt{[i][k][j]}}{\textit{add x and y coordinates to the map of obstacle contour}}
			}
		}
	}
	}
\end{algorithm}
\begin{algorithm}[!htbp]	
	\SetKwBlock{Begin}{}{end}
	\Begin{
		$k\gets0$\;
		$dist \gets \sqrt{(s_x-g_x)^2+(s_y+g_y)^2}$\;
		\textit{path cost} $\gets0$\;
		\Comment{\textcolor{blue}{// as long as it does not reach destination}}
		\While{$dist\leq res$ \hspace{1cm} \Comment{\textcolor{blue}{// start planning the route}}}{
			$[(r_x,$ $r_y)]\gets$ \textit{\small list of resulting coordinates from path planner}\;
			$q \gets 0$ \;
			\While{\textit{path cost} $\leq 18(k+1)$}{
				{\small \textit{path cost} $\mathrel{+}=$ \textit{distance between subsequent path points in km}}\;
				\If{\textit{path cost} $\geq 18(k+1)$}{\textit{come back to the previous path point}\; \texttt{break}\;}
				$q \mathrel{+}= 1$	
			}
			${s_x}_{new} = r_x$\texttt{[q]}; \Comment{\textcolor{blue}{// re-planning way-point}}
			${s_y}_{new} = r_y$\texttt{[q]}\;
			\textit{final path}$_x$.\texttt{extend}$(r_x$\texttt{[:q]})\;
			\textit{final path}$_y$.\texttt{extend}$(r_y$\texttt{[:q]})\;
			
			\If{path is not found }{
				\Comment{\textcolor{red}{// $1^{st}$ problem}}
				\textit{move of 1 hour's drive along the path generated with the previous time window}
			}
			$k \mathrel{+}= 1$\;	
			\If{${s_x}_{new}$, ${s_y}_{new}$ are within cost\texttt{[k]}}{
				\Comment{\textcolor{red}{// $2^{st}$ problem}}
				\textit{the USV is lost}\;
				\texttt{break}}
		}
	}
\end{algorithm}

%Another method to merge the time windows and predict obstacle movement 
\begin{algorithm}[h]
	\caption{\textbf{\textit{Local}} approach}\label{local}
	\KwData{\\
		$H_{m0}\gets array(time, latitude, longitude)$\;
		$N \gets$ \textit{number of time windows}\;
		$cost \gets$ \textit{boolean} $array(time, latitude, longitude)$;}
	\KwResult{$[(o_x,o_y)] \gets$ \textit{list of obstacle points}}
	\For{$i$ in range($N$))}{
		$cost$\texttt{[i]} $\gets H_{m0}$\texttt{[i]} $> 0.8$ $m$\;}
	$image \gets $ \texttt{np.zeros}((len(latitude), len(longitude)))\;
	\For{$i$ in range($N$)}{
		\For{k in range(len(latitude))}{
			\For{j in range(len(longitude))}{{\small\textit{ compute geodetic distance between point}\texttt{[k][j]} \textit{and the harbor}}\;
				\If{$18i<$ \textit{geo distance} $<18(i+1)$ and cost\texttt{[i][k][j]}}{\textit{add x and y coordinates to the map of obstacles}\;
					image\texttt{[k][j]} = 1}
			}
		}
	}
	\textit{wave contour} $=$ \texttt{np.logical\_xor}(\textit{image,} \texttt{binary\_erosion}(\textit{image}))\;
	$[(o_x,o_y)]\gets \text{list of x and y coordinates of obstacle contour}$
	%\{\textit{extract contour of the map of obstacles}\}\;
	%\{\textit{extract land contours from the mask}\}\;
\end{algorithm}


\begin{figure}[H]
	\centering 
	\subfloat[]{\includegraphics[width=0.29\textwidth]{Figures/localexample2.png}\label{t4}}
	\hspace{9cm}
	\subfloat[]{\includegraphics[width=0.29\textwidth]{Figures/localexample3.png}\label{t4}}
	\hspace{1cm}
	\subfloat[]{\includegraphics[width=0.29\textwidth]{Figures/localexample4.png}\label{t5}}
	\vspace{0.0cm}
	\captionsetup{font=footnotesize,labelfont=footnotesize}
	\caption{(a) Unsafe marine areas at two consecutive hours. (b) Point obstacles considered in the fourth band.  (c) Point obstacles considered in the fifth band.} 
	\label{localobs}
\end{figure}

\subsection{Uncertainty of the forecasts}
Motion planning in the real world is generally subjected to uncertainty in the environment, especially when the adverse weather represents the obstacles. Due to initial condition uncertainties and model errors, forecasts are never accurate \cite{slingo2011uncertainty}, and the error becomes more prominent with the horizon length.\\
Wave forecasts of Copernicus Marine Service are updated two times daily, at 06:00 UTC and 20:00 UTC. So the USV should manage prediction uncertainty for a temporal horizon of 14 h.
%Considering a temporal horizon of 14 h, the amount of prediction uncertainty will increase with the advancing time.

In order to improve the safety, a potential solution would be to expand the unsafe regions, represented by areas of the configuration space in which $H_{m0} > 0.8$ $m$, by appropriate bounded margins, depending on the degree of uncertainty. Figure \ref{safmargin} displays possible safety margins at three times of the day from the forecast update.
It should be noted that an overestimation of the uncertainty could lead to excessive coverage of $\mathcal{C}_{free}$, affecting the optimality of the trajectory and even possibly preventing from finding a solution \cite{mothes2019trajectory}, as the conservative forecast map obtained by the \textbf{\textit{Sum}} approach.\\
%This uncertainty could be taken into account explicitly by bounded margins, i.e.,bounded uncertainty.
In this thesis work, this aspect will not be examined. Nevertheless, it was appropriate to mention it for performing more realistic simulations.
\vspace{0.5cm}
\begin{figure}[h]
	\centering 
	\subfloat[06:00 UTC]{\includegraphics[width=0.32\textwidth]{Figures/uncertainty0.png}\label{0margin}}
	\hspace{0.1cm}
	\subfloat[13:00 UTC]{\includegraphics[width=0.32\textwidth]{Figures/uncertainty1.png}\label{1margin}}
	\hspace{0.1cm}
	\subfloat[18:00 UTC]{\includegraphics[width=0.32\textwidth]{Figures/uncertainty2.png}\label{2margin}}
	\vspace{0.0cm}
	\caption{Safety margins at three different hours of the forecast} 
	\label{safmargin}
\end{figure}
%In this work, the position of obstacles, represented by wave forecasts, is considered deterministic, despite of the stocasticity of the data.Data are stochastic because they come from numerical weather models.
%Uncertainties associated with forecast data naturally exists, and become more prominent with the horizon length of the forecast. The hazardous regions can be modeled as probabilistic closed sets unsafe to fly through  (planning random obstacles)
%The immanent uncertainty in the environmental prediction is taken into account explicitly by bounded margins
\newpage
\section{Summary}
In this chapter, we approached the problem of path planning with moving obstacle prediction.\\
We chose the region of the Tyrrhenian Sea that washes the coasts of Tuscany as configuration space due to its naturalistic spots and dimensions suitable for the simulations. Areas with significant wave height over 0.8 m represent the obstacles to avoid during navigation. \\
The spatio-temporal map generation is performed through three different methods, \textbf{\textit{Sum}}, \textbf{\textit{Global}}, and \textbf{\textit{Local}}, which will be evaluated in the next chapter.

%\subsection{Implementation of sea currents}
%\begin{figure}[H]
%	\centering
%	\includegraphics[width=70mm]{Figures/current.png}
%	\caption{Sea current planning}
%	\label{current}
%\end{figure}

%\textcolor{red}{describe better how I implemented the planning part} 

%\subsection{Uncertainty of the forecasts}
%We know that predictions suffer of a grade of uncertainty: they are not exact, especially if they provide information at very small resolution, as in this case. Although they are short-range forecasts (1-2 days), it's better to account for the variability.


\begin{comment}
	\begin{enumerate}
		\item \textbf{Improved sum}: basic sum method is too much conservative. Summing the temporal maps all together leads to consider $t_5$ when the boat starts. And, same thing, to consider $t_1$ very far from the starting point. The idea would be to remove first map after 1h of theoretical travel. In a way, it could be considered as an union between basic sum and time-band (local) method 
		\item \textbf{Improved Global}: this method is based on a careful study of wavefronts movements. The classic algorithm tries to return the optimal path by remaining near the obstacle. A tolerance around dangerous waves could indicate a safety risk and reduce the possibility that the problem at the edge of time windows occurred. 
		\item \textbf{Improved Local}: instead of disjoint time window bands, consider overlapped windows to take approximate better the behavior in real situation: the boat will never travel each band in the theoretical time
	\end{enumerate}

\end{comment} 


\chapter{Weather avoidance simulations}
\label{results}
This chapter shows the results of the developed spatio-temporal approaches for global path planning with large moving obstacles. After a brief test in artificial scenarios, we will see the paths taken in a real environment and a detailed comparison between algorithms and approaches. To follow, a discussion summaries the results obtained.
\section{Artificial disturbance scenarios}
Before performing real simulations, \textbf{\textit{Sum}}, \textbf{\textit{Global}}, and \textbf{\textit{Local}} are firstly tested in simple scenarios using the three path planning algorithms discussed in \autoref{second}.
%In all of them, we use the three path planning algorithms.
\begin{itemize}[itemsep=0pt]
	\item \textbf{Scenario 1} represents a storm that moves in the vertical direction without changing shape;
	\item \textbf{Scenario 2} simulates a cell of convective weather expanding in the middle of the navigation; % spreading
	\item \textbf{Scenario 3} is a perturbation that moves and changes size in time.
\end{itemize}
We chose such scenarios to see the behavior of the algorithms in the presence of probable marine weather events.
An area of $100\times80$ km is necessary to consider the time evolution of the fake atmospheric factors, which keep a temporal resolution of 1 hour. Results are illustrated in Figure \ref{scenario123}, where, for each scenario, the three approaches are displayed side by side.

Each spatio-temporal approach sees the weather differently due to the generated map. This results in distinct navigational decision-making that can lead the vessel to travel a longer path (as with \textbf{\textit{Sum}}) or refrain from sailing due to the failure of the path planning with \textbf{\textit{Global}}.
The same consideration can be done regarding the three algorithms: RRT* and A* search for the optimal solution by choosing similar path directions that leads to comparable path lengths. Instead, APF is the least efficient at generating a global route. Despite this, it is the only one that finds a complete path with \textbf{\textit{Global}} in scenario 1 (Figure \ref{toyglo1}), allowing the vehicle to start the mission.
\begin{figure}[h]
	\centering 
	\subfloat[Scenario 1 - \textbf{\textit{Sum}}]{\includegraphics[width=0.32\textwidth]{Figures/toysum1.png}\label{toysum1}}
	\hspace{0.1cm}
	\subfloat[Scenario 1 - \textbf{\textit{Global}}]{\includegraphics[width=0.32\textwidth]{Figures/toyglo1.png}\label{toyglo1}}
	\hspace{0.1cm}
	\subfloat[Scenario 1 - \textbf{\textit{Local}}]{\includegraphics[width=0.32\textwidth]{Figures/toyloc1.png}\label{toyloc1}}
	\hspace{0.1cm}
	\subfloat[Scenario 2 - \textbf{\textit{Sum}}]{\includegraphics[width=0.32\textwidth]{Figures/toysum2.png}\label{toysum2}}
	\hspace{0.1cm}
	\subfloat[Scenario 2 - \textbf{\textit{Global}}]{\includegraphics[width=0.32\textwidth]{Figures/toyglo2.png}\label{toyglo2}}
	\hspace{0.1cm}
	\subfloat[Scenario 2 - \textbf{\textit{Local}}]{\includegraphics[width=0.32\textwidth]{Figures/toyloc2.png}\label{toyloc2}}
	\hspace{0.1cm}
	\subfloat[Scenario 3 - \textbf{\textit{Sum}}]{\includegraphics[width=0.32\textwidth]{Figures/toysum3.png}\label{toysum3}}
	\hspace{0.1cm}
	\subfloat[Scenario 3 - \textbf{\textit{Global}}]{\includegraphics[width=0.32\textwidth]{Figures/toyglo3.png}\label{toyglo3}}
	\hspace{0.1cm}
	\subfloat[Scenario 3 - \textbf{\textit{Local}}]{\includegraphics[width=0.32\textwidth]{Figures/toyloc3.png}\label{toyloc3}}
	\vspace{0.0cm}
	%\captionsetup{skip=-1mm}
	\caption{Artificial disturbance scenarios} 
	\label{scenario123}
\end{figure}

% because it tends to move following the repulsive swirling potential of the obstacles. .
%\vspace{7.5cm}
\section{Real environment simulations}
The following figures show the paths generated in the first five days of navigation, each with its specific marine conditions. In the \hyperref[appendix]{Appendix}, instead, we displayed a series of additional days.\\ 
Due to a lack of data and the reliance on the sea state for navigation, the current study does not represent a robust statistical analysis, for whom thousands of routes would be necessary. However, several simulations have been performed to strengthen the obtained results: we analyzed 12 days of 2021 and took two survey points, evaluating the reverse paths; the four nautical routes are listed below.
Given a distance as the crow flies of about 8-9 hours, we considered a variable number between 12 and 15 forecast time windows. 
\vspace{0.1cm}
\begin{center}
	\begin{tabular}{@{} r @{${}\longrightarrow{}$} l @{}}
		Porto Vecchio harbor & Argentario\\
		Porto Vecchio harbor & Elba Island\\
		Argentario & Porto Vecchio harbor\\
		Elba Island & Porto Vecchio harbor
	\end{tabular}
\end{center}
\vspace{0.1cm}
%These are run along 12 days of 2021 and two ending point. We chose days and time windows that could influence the navigation of the unmanned vessel. \\
%As \textit{Global}} uses every time window for planning, in . Therefore, depending on the case, another specific map will be displayed. 
%The symbol "$\times$" present in global paths corresponds to the way-point from which the vessel re-plan the route. There should be one point every $\Delta t=1$ hour of travel approximately.
%Before performing an accurate analysis, let's see the results of the real simulations. 
%The limited number of scenarios is due to the fact that, given a small configuration space, and constrained to choose start and goal with a certain meaning, I ended up to take two. Also, the choice of reasonable days of wave forecast (having to see how it moves and how it can affect path planning) takes time, and I chose five in the end. Repeating the simulations without changes leads to same results (except for RRT*), and so it was unnecessary. \\

Along with the map images, Table \ref{tab:addlabel} and \ref{tabellaelba} indicate the travel time $T_{path}$ and computing time $t_{exec}$ of each run. In particular, the word "\textcolor{red}{None}" reports when the vehicle decides not to sail due to excessive coverage of the free space or due to a possible predicted sinking in case the USV crosses that path.\\
Figure \ref{legendglobal} provides a legend with color bars to better understand the movements of the waves during \textbf{\textit{Global}}'s navigation.
\begin{figure}[H]
	\centering
	\includegraphics[width=100mm]{Figures/legendglobal.png}
	\caption{Legend for the time windows in \textbf{\textit{Global}}}
	\label{legendglobal}
\end{figure} 
Still on \textbf{\textit{Global}}'s charts, the symbols '$\times$' in the paths represent the points where the vessel re-plans the route and are approximately at a distance of 1 hour's drive away from each other. Lastly, the red crosses signal where a possible sinking may occur.

Concerning the three path planning algorithms, we can make the following statements by observing the figures:
\begin{itemize}[itemsep=0pt]
	\item \textbf{A*} path smoothness is not high due to redundant nodes;
	\item \textbf{APF} shows the same behavior as in the artificial scenarios: it focuses on getting closer to its destination because of the attractive potential to the goal, without considering the path's cost so far. Despite its lower performance, the current results do not highlight significant increases in the travel time;
	\item \textbf{RRT*} is the preferable algorithm even considering the high computing time to optimize the tree: thanks to the fact that it is not grid-based, its generated paths are shorter and smoother.
\end{itemize}  
%, while the symbol ``\textcolor{red}{$\times$}'' when it is not able to manage the navigation anymore.
\vspace{1cm}
\begin{table}[h]
	\centering
	\resizebox{\textwidth}{!}{%
		\begin{tabular}{|c|c|c|c|c|c|c|c|c|c|c|c|}
			\hline
			\multicolumn{2}{|c|}{} & \multicolumn{2}{c|}{\textbf{Day1}} & \multicolumn{2}{c|}{\textbf{Day2}} & \multicolumn{2}{c|}{\textbf{Day3}} & \multicolumn{2}{c|}{\textbf{Day4}} & \multicolumn{2}{c|}{\textbf{Day5}} \bigstrut\\
			\hline
			Algorithm & Planning & \multicolumn{1}{p{3em}|}{$T_{path}$($h$)} & \multicolumn{1}{p{3em}|}{$t_{exec}$($s$)} & \multicolumn{1}{p{3em}|}{$T_{path}$($h$)} & \multicolumn{1}{p{3em}|}{$t_{exec}$($s$)} & \multicolumn{1}{p{3em}|}{$T_{path}$($h$)} & \multicolumn{1}{p{3em}|}{$t_{exec}$($s$)} & \multicolumn{1}{p{3em}|}{$T_{path}$($h$)} & \multicolumn{1}{p{3em}|}{$t_{exec}$($s$)} & \multicolumn{1}{p{3em}|}{$T_{path}$($h$)} & \multicolumn{1}{p{3em}|}{$t_{exec}$($s$)} \bigstrut\\
			\hline
			\multirow{3}[6]{*}{\textbf{RRT*}} & \textit{Global}  & 9.93  & 35.361 & 9.13  & 60.924 & 10.10 & 51.197 &  \textcolor{red}{None}     &  \textcolor{red}{None}     & 10.61  & 59.267 \bigstrut\\
			\cline{2-12}          & \textit{Local} & 9.49  & 11.053 & 9.07  & 12.595 & 10.08 & 6.240 & 10.25 & 8.346 & 9.99  & 12.013 \bigstrut\\
			\cline{2-12}          & \textit{Sum} & \textcolor{red}{None}  & \textcolor{red}{None}  & 10.95 & 5.125 & 11.01 & 7.037 & 10.76 & 9.617 & \textcolor{red}{None}  & \textcolor{red}{None} \bigstrut\\
			\hline
			\multirow{3}[6]{*}{\textbf{A*}} & \textit{Global}  & 10.10 & 35.100 & 9.98  & 41.236 & 10.99 & 39.342 &  \textcolor{red}{None}     &  \textcolor{red}{None}     & \textcolor{red}{None}     &  \textcolor{red}{None} \bigstrut\\
			\cline{2-12}          & \textit{Local} & 10.10 & 3.361 & 9.98  & 4.157 & 10.99 & 3.421 & 10.47 & 4.028 & 10.35 & 7.469 \bigstrut\\
			\cline{2-12}          & \textit{Sum} & \textcolor{red}{None}  & \textcolor{red}{None}  & 11.75 & 3.597 & 12.27 & 3.690 & 11.24 & 3.305 & \textcolor{red}{None}  & \textcolor{red}{None} \bigstrut\\
			\hline
			\multirow{3}[6]{*}{\textbf{APF}} & \textit{Global}  & 9.70  & 4.262 & 8.80  & 11.195 & 10.23 & 4.029 &  \textcolor{red}{None}     &  \textcolor{red}{None}     &  \textcolor{red}{None}     &  \textcolor{red}{None} \bigstrut\\
			\cline{2-12}          & \textit{Local} & 10.15 & 0.722 & 9.00  & 0.767 & 10.10 & 0.661 & 10.62 & 1.259 & 10.61 & 1.381 \bigstrut\\
			\cline{2-12}          & \textit{Sum} & \textcolor{red}{None}  & \textcolor{red}{None}  & 11.63 & 1.016 & 11.78 & 1.024 & 11.13 & 1.140 & \textcolor{red}{None}  & \textcolor{red}{None} \bigstrut\\
			\hline
	\end{tabular}}%
	\caption{Argentario survey - Travel and computing time}
	\label{tab:addlabel}%
\end{table}%
\begin{figure}[H]
	\centering 
	\subfloat[Day 1 - \textbf{\textit{Sum}}]{\includegraphics[width=0.32\textwidth]{Figures/day1s.png}\label{day1s}}
	\hspace{0.1cm}
	\subfloat[Day 1 - \textbf{\textit{Global}}]{\includegraphics[width=0.32\textwidth]{Figures/day1g2.png}\label{day1g}}
	\hspace{0.1cm}
	\subfloat[Day 1 -  \textbf{\textit{Local}}]{\includegraphics[width=0.32\textwidth]{Figures/day1l.png}\label{day1l}}
	\vspace{0.0cm}
	\caption{Argentario survey - Day 1} 
	\label{Survey1}
\end{figure}
\newpage
\begin{figure}[H]
	\centering
	\subfloat[Day 2 -  \textbf{\textit{Sum}}]{\includegraphics[width=0.32\textwidth]{Figures/day2s.png}\label{day2s}}
	\hspace{0.1cm}
	\subfloat[Day 2 -  \textbf{\textit{Global}}]{\includegraphics[width=0.32\textwidth]{Figures/day2g2.png}\label{day2g}}
	\hspace{0.1cm}
	\subfloat[Day 2 -  \textbf{\textit{Local}}]{\includegraphics[width=0.32\textwidth]{Figures/day2l.png}\label{day2l}}
	\hspace{0.1cm}
	\subfloat[Day 3 -  \textbf{\textit{Sum}}]{\includegraphics[width=0.32\textwidth]{Figures/day3s.png}\label{day3s}}
	\hspace{0.1cm}
	\subfloat[Day 3 -  \textbf{\textit{Global}}]{\includegraphics[width=0.32\textwidth]{Figures/day3g2.png}\label{day3g}}
	\hspace{0.1cm}
	\subfloat[Day 3 -  \textbf{\textit{Local}}]{\includegraphics[width=0.32\textwidth]{Figures/day3l.png}\label{day3l}}
	\hspace{0.1cm}
	\subfloat[Day 4 -  \textbf{\textit{Sum}}]{\includegraphics[width=0.32\textwidth]{Figures/day4s.png}\label{day4s}}
	\hspace{0.1cm}
	\subfloat[Day 4 -  \textbf{\textit{Global}}]{\includegraphics[width=0.32\textwidth]{Figures/day4g2.png}\label{day4g}}
	\hspace{0.1cm}
	\subfloat[Day 4 -  \textbf{\textit{Local}}]{\includegraphics[width=0.32\textwidth]{Figures/day4l.png}\label{day4l}}
	\hspace{0.1cm}
	\subfloat[Day 5 -  \textbf{\textit{Sum}}]{\includegraphics[width=0.32\textwidth]{Figures/day5s.png}\label{day5s}}
	\hspace{0.1cm}
	\subfloat[Day 5 -  \textbf{\textit{Global}}]{\includegraphics[width=0.32\textwidth]{Figures/day5gnew.png}\label{day5g}}
	\hspace{0.1cm}
	\subfloat[Day 5 - \textbf{\textit{Local}}]{\includegraphics[width=0.32\textwidth]{Figures/day5l.png}\label{day5l}}
	\vspace{0.0cm}
	\caption{Argentario survey - Day 2 to 5} 
	\label{Survey11}
\end{figure}
% Table generated by Excel2LaTeX from sheet 'Foglio1'
\newpage
\begin{table}[h]
	\centering
	\resizebox{\textwidth}{!}{%
		\begin{tabular}{|c|c|c|c|c|c|c|c|c|c|c|c|}
			\hline
			\multicolumn{2}{|c|}{} & \multicolumn{2}{c|}{\textbf{Day 1}} & \multicolumn{2}{c|}{\textbf{Day 2}} & \multicolumn{2}{c|}{\textbf{Day 3}} & \multicolumn{2}{c|}{\textbf{Day 4}} & \multicolumn{2}{c|}{\textbf{Day 5}} \bigstrut\\
			\hline
			Algorithm & Planning & \multicolumn{1}{p{3em}|}{$T_{path}$($h$)} & \multicolumn{1}{p{3em}|}{$t_{exec}$($s$)} & \multicolumn{1}{p{3em}|}{$T_{path}$($h$)} & \multicolumn{1}{p{3em}|}{$t_{exec}$($s$)} & \multicolumn{1}{p{3em}|}{$T_{path}$($h$)} & \multicolumn{1}{p{3em}|}{$t_{exec}$($s$)} & \multicolumn{1}{p{3em}|}{$T_{path}$($h$)} & \multicolumn{1}{p{3em}|}{$t_{exec}$($s$)} & \multicolumn{1}{p{3em}|}{$T_{path}$($h$)} & \multicolumn{1}{p{3em}|}{$t_{exec}$($s$)} \bigstrut\\
			\hline
			\multirow{3}[6]{*}{\textbf{RRT*}} & \textit{Global}  & 8.76  & 28.301 & 8.59  & 40.481 & 8.78  & 31.269 & \textcolor{red}{None}     & \textcolor{red}{None}     & \textcolor{red}{None}     & \textcolor{red}{None} \bigstrut\\
			\cline{2-12}          & \textit{Local} & 8.51  & 10.719 & 8.56  & 9.615 & 8.94  & 4.550 & 8.61  & 6.085 & 8.47  & 8.093 \bigstrut\\
			\cline{2-12}          & \textit{Sum} & \textcolor{red}{None}  & \textcolor{red}{None}  & 9.01  & 3.984 & 8.98  & 3.922 & 8.83  & 2.807 & \textcolor{red}{None}  & \textcolor{red}{None} \bigstrut\\
			\hline
			\multirow{3}[6]{*}{ \textbf{A*}} & \textit{Global}  & 8.65  & 31.229 & 8.52  & 36.585 & 9.53  & 36.032 & \textcolor{red}{None}     & \textcolor{red}{None}     & \textcolor{red}{None}  & \textcolor{red}{None} \bigstrut\\
			\cline{2-12}          & \textit{Local} & 8.65  & 3.198 & 8.52  & 4.360 & 9.53  & 3.647 & 9.03  & 4.333 & 8.65  & 6.111 \bigstrut\\
			\cline{2-12}          & \textit{Sum} & \textcolor{red}{None}  & \textcolor{red}{None}  & 9.78  & 3.610 & 9.79  & 3.951 & 9.28  & 3.620 & \textcolor{red}{None}  & \textcolor{red}{None} \bigstrut\\
			\hline
			\multirow{3}[6]{*}{\textbf{APF}} & \textit{Global}  & 8.39  & 3.912 & 8.11  & 5.103 & 8.74  & 4.611 & \textcolor{red}{None}  & \textcolor{red}{None} & \textcolor{red}{None}     & \textcolor{red}{None} \bigstrut\\
			\cline{2-12}          & \textit{Local} & 8.30  & 0.951 & 8.11  & 0.690 & 9.45  & 0.668 & 8.70  & 0.827 & 8.30  & 1.237 \bigstrut\\
			\cline{2-12}          & \textit{Sum} & \textcolor{red}{None}  & \textcolor{red}{None}  & 9.56  & 0.756 & 9.35  & 0.638 & 9.36  & 0.953 & \textcolor{red}{None}  & \textcolor{red}{None} \bigstrut\\
			\hline
	\end{tabular}}%
	\caption{Elba Island survey - Travel and computing time}
	\label{tabellaelba}
\end{table}%
\vspace{1cm}
\begin{figure}[h]
	\centering 
	\subfloat[Day 1 -  \textbf{\textit{Sum}}]{\includegraphics[width=0.32\textwidth]{Figures/day1_2s.png}\label{day12s}}
	\hspace{0.1cm}
	\subfloat[Day 1 -  \textbf{\textit{Global}}]{\includegraphics[width=0.32\textwidth]{Figures/day1_2g2.png}\label{day12g}}
	\hspace{0.1cm}
	\subfloat[Day 1 -  \textbf{\textit{Local}}]{\includegraphics[width=0.32\textwidth]{Figures/day1_2l.png}\label{day12l}}
	\hspace{0.1cm}
	\subfloat[Day 2 -  \textbf{\textit{Sum}}]{\includegraphics[width=0.32\textwidth]{Figures/day2_2s.png}\label{day22s}}
	\hspace{0.1cm}
	\subfloat[Day 2 -  \textbf{\textit{Global}}]{\includegraphics[width=0.32\textwidth]{Figures/day2_2g2.png}\label{day22g}}
	\hspace{0.1cm}
	\subfloat[Day 2 -  \textbf{\textit{Local}}]{\includegraphics[width=0.32\textwidth]{Figures/day2_2l.png}\label{day22l}}
	\vspace{0.0cm}
	\caption{Elba Island survey - Day 1 to 2} 
	\label{Survey2}
\end{figure}
\newpage
\begin{figure}[H]
	\centering
	\subfloat[Day 3 -  \textbf{\textit{Sum}}]{\includegraphics[width=0.32\textwidth]{Figures/day3_2s.png}\label{day32s}}
	\hspace{0.1cm}
	\subfloat[Day 3 -  \textbf{\textit{Global}}]{\includegraphics[width=0.32\textwidth]{Figures/day3_2g2.png}\label{day32g}}
	\hspace{0.1cm}
	\subfloat[Day 3 -  \textbf{\textit{Local}}]{\includegraphics[width=0.32\textwidth]{Figures/day3_2l.png}\label{day32l}}
	\hspace{0.1cm}
	\subfloat[Day 4 -  \textbf{\textit{Sum}}]{\includegraphics[width=0.32\textwidth]{Figures/day4_2s.png}\label{day42s}}
	\hspace{0.1cm}
	\subfloat[Day 4 -  \textbf{\textit{Global}}]{\includegraphics[width=0.32\textwidth]{Figures/day4_2gnew.png}\label{day42g}}
	\hspace{0.1cm}
	\subfloat[Day 4 -  \textbf{\textit{Local}}]{\includegraphics[width=0.32\textwidth]{Figures/day4_2l.png}\label{day42l}}
	\hspace{0.1cm}
	\subfloat[Day 5 -  \textbf{\textit{Sum}}]{\includegraphics[width=0.32\textwidth]{Figures/day5_2s.png}\label{day52s}}
	\hspace{0.1cm}
	\subfloat[Day 5 -  \textbf{\textit{Global}}]{\includegraphics[width=0.32\textwidth]{Figures/day5_2g2.png}\label{day52g}}
	\hspace{0.1cm}
	\subfloat[Day 5 - \textbf{\textit{Local}}]{\includegraphics[width=0.32\textwidth]{Figures/day5_2l.png}\label{day52l}}
	\vspace{0.0cm}
	\caption{Elba Island survey - Day 3 to 5} 
	\label{Survey22}
\end{figure}
% Table generated by Excel2LaTeX from sheet 'Foglio1'

%As can be seen there are 30 plots: every day of simulation has 3 figures side by side that represent each planning approach by comparison. 
%There should be theoretically one point every $\Delta t=1$ hour of travel, but in practice, especially for RRT* (since it does not use geographical coordinates), they are not completely accurate.\\ 
%From the figures, as a first glance, we can state that the three algorithms produce similar paths. 
%During the plannings of second survey area, time difference is less evident, but the problem related to path generation still remains.\\ 
The comparison between the spatio-temporal approaches points out that the paths of \textbf{\textit{Sum}} are clearly longer, and the time difference can be even 2 hours (see Day 2 in Table \ref{tab:addlabel}). Furthermore, this approach is heavily influenced by marine forecasts. For example, on Day 1 and 5 (Figure \ref{day1s} and \ref{day5s}), the presence of unsafe regions is so high that it cannot generate any path. In this case, the assurance of safety heavily affects the approach's efficiency.

Regarding \textbf{\textit{Global}} and \textbf{\textit{Local}}, the results are quite compatible limited to the completed paths. In fact, the former usually fails the planning (see Figure \ref{day4g}, \ref{day5g}, \ref{day42g}, \ref{day52g}): it does not realize how waves change until it enters into the unsafe area.
As can be seen, it can happen that in a single day, not all of the algorithms manage to reach the destination with \textbf{\textit{Global}}. The reason is that the maps generated by this approach are directly related to the resulting paths. Therefore different outcomes derive from the progress of the nautical routes.
% which means that the final path will never coincide between each other and so their performance will not be equal.
\subsection{Performance evaluation}
As mentioned previously, the three key metrics are \textbf{travel time}, \textbf{efficiency}, and \textbf{safety}. The former determines the effectiveness of the path in terms of spent time and energy consumption. It is computed by summing the geodetic distances between every subsequent way-points of the path. \\
The other two indexes are used to evaluate the spatio-temporal approaches for the navigation of the USV. The first reports the percentage of finding a viable path, while the other is the percentage of the navigation safety to minimize potential risks. The ratio between the navigable days and the entire set of days determines the efficiency of the approach, which will change depending on the accuracy in estimating obstacles' positions. At the same time, the safety index is intrinsic in the development of the approach, except for \textbf{\textit{Local}}, as explained in the previous chapter. 
% , which depend on the forecast map considered,
Figure \ref{compalg} and \ref{compapproach} show in detail the statistical comparison of the travel time among the various approaches and algorithms, with a prominence to the percentage change to the fastest approach\slash algorithm. To follow, two barplots and a 2D chart depict the results of the efficiency and safety. 

Results of the travel time show that \textbf{\textit{Global}} and \textbf{\textit{Local}} generate paths of similar lengths, proved by a maximum percentage variation of 2.3\% with respect to the fastest route with RRT* (see Table \ref{tablecompalg}). Although the difference is small, it is clear how relying on the forecast leads to more efficient and shorter paths.\\
In the algorithm comparison, RRT* achieves best results both with \textbf{\textit{Sum}} and \textbf{\textit{Local}}, and differs from APF by less than 3\% with \textbf{\textit{Global}}. The reason is related to the better re-planning performance of the potential field method than the other algorithms.  

  
%However the \textbf{\textit{Sum}} and \textbf{\textit{Global}}'s outcome are biased by fewer data used due to the lack of complete paths. Normally, avoiding considering obstacle prediction would lengthen the route. In this case, it is better to rely on the efficiency and safety index before reading these results.
%The two histograms below compare data from the two tables: each bar includes travel time and computing time of the algorithms during five days for two destinations. The difference between local and global approach can be approximately considered negligible. But in practice the latter can be unfeasible. Instead the difference in computing time is considerable, because of how this algorithm is structured: it does repeated plannings until arriving at the goal, while the others do only one.
\newpage
\begin{figure}[H]
	\centering 
	\subfloat{\includegraphics[width=0.30\textwidth]{Figures/barrrtnew3.png}\label{rrtcomp}}
	\hspace{0.2cm}
	\subfloat{\includegraphics[width=0.30\textwidth]{Figures/barastarnew3.png}\label{astarcomp}}
	\hspace{0.2cm}
	\subfloat{\includegraphics[width=0.30\textwidth]{Figures/barapfnew3.png}\label{apfcomp}}
	\captionsetup{font=footnotesize,labelfont=footnotesize}
	\caption{Comparison of travel time among the various approaches divided by algorithms} 
	\label{compalg}
\end{figure}

\begin{table}[htbp]
	\small
	\centering
	
	\begin{tabular}{|c|c|c|c|c|c|c|}
		\hline
		\multirow{3}{*}{}  & \multicolumn{2}{c|}{\textbf{RRT*}} & \multicolumn{2}{c|}{\textbf{A*}} & \multicolumn{2}{c|}{\textbf{APF}} \bigstrut\\
		\cline{2-7}
		& \multirow{2}{*}{\parbox{3em}{\centering \small Mean $T_{path}$($h$)}} & \multirow{2}{*}{\parbox{4em}{\centering \scriptsize Variation (\%) w.r.t. $min(T_{path})$}} & \multirow{2}{*}{\parbox{3em}{\centering \small Mean $T_{path}$($h$)}} & \multirow{2}{*}{\parbox{4em}{\centering \scriptsize Variation (\%) w.r.t. $min(T_{path})$}} & \multirow{2}{*}{\parbox{3em}{\centering \small Mean $T_{path}$($h$)}} & \multirow{2}{*}{\parbox{4em}{\centering \scriptsize Variation (\%) w.r.t. $min(T_{path})$}} \bigstrut[t]\\
		&   &   &   &   & & \bigstrut[b]\\
		\hline
		\small \textit{Sum} & 9.6  & 7.3   & 10.4   & 8.2  & 10.5   & 13.5 \bigstrut\\
		\hline
		\small \textit{Global} & 9.2  & 2.3   & 9.7   & 0.4   & 9.3   & 0.4 \bigstrut\\
		\hline
		\small \textit{Local} & \textbf{9.0}  & \textbf{0.0}   & \textbf{9.62}   & \textbf{0.0}   & \textbf{9.23}   & \textbf{0.0} \bigstrut\\
		\hline
	\end{tabular}%
	\captionsetup{font=footnotesize,labelfont=footnotesize}
	\caption{Data comparison of travel time among the various approaches divided by algorithms}
	\label{tablecompalg}%
\end{table}%

\begin{figure}[h]
	\centering
	\subfloat{\includegraphics[width=0.30\textwidth]{Figures/barsumnew3.png}\label{sumcomp}}
	\hspace{0.2cm}
	\subfloat{\includegraphics[width=0.30\textwidth]{Figures/barglonew3.png}\label{glocomp}}
	\hspace{0.2cm}
	\subfloat{\includegraphics[width=0.30\textwidth]{Figures/barlocnew2.png}\label{loccomp}}
	\captionsetup{font=footnotesize,labelfont=footnotesize}
	\caption{Comparison of travel time among the various algorithms divided by approaches} 
	\label{compapproach}
\end{figure}

\begin{table}[htbp]
	\small
	\centering
	
	\begin{tabular}{|c|c|c|c|c|c|c|}
		\hline
		\multirow{3}{*}{}  & \multicolumn{2}{c|}{\textit{Sum}} & \multicolumn{2}{c|}{\textit{Global}} & \multicolumn{2}{c|}{\textit{Local}} \bigstrut\\
		\cline{2-7}
		& \multirow{2}{*}{\parbox{3em}{\centering \small Mean $T_{path}$($h$)}} & \multirow{2}{*}{\parbox{4em}{\centering \scriptsize Variation (\%) w.r.t. $min(T_{path})$}} & \multirow{2}{*}{\parbox{3em}{\centering \small Mean $T_{path}$($h$)}} & \multirow{2}{*}{\parbox{4em}{\centering \scriptsize Variation (\%) w.r.t. $min(T_{path})$}} & \multirow{2}{*}{\parbox{3em}{\centering \small Mean $T_{path}$($h$)}} & \multirow{2}{*}{\parbox{4em}{\centering \scriptsize Variation (\%) w.r.t. $min(T_{path})$}} \bigstrut[t]\\
		&   &   &   &   & & \bigstrut[b]\\
		\hline
		\small \textbf{RRT*}   & \textbf{10.1}  & \textbf{0.0}   & 9.3   & 2.9   & \textbf{9.3}   & \textbf{0.0} \bigstrut\\
		\hline
		\small \textbf{A*}    & 10.7  & 5.9   & 9.5   & 4.9   & 9.8   & 4.9 \bigstrut\\
		\hline
		\small \textbf{APF}   & 10.5  & 4.7   & \textbf{9.1}   & \textbf{0.0}   & 9.5   & 2.3 \bigstrut\\
		\hline
	\end{tabular}%
	\captionsetup{font=footnotesize,labelfont=footnotesize}
	\caption{Data comparison of travel time among the various algorithms divided by approaches}
	\label{tablecompapproach}%
\end{table}%
\newpage
The values illustrated in Figure \ref{compeffsaf1} and \ref{measureeffsaf} show that the efficiency of \textbf{\textit{Sum}} and \textbf{\textit{Global}} is respectively of 56.9 and 52.8\%, while 87.5\% for \textbf{\textit{Local}}. Taking as reference the latter, which represents the ideal approach that always finds a feasible path thanks to its strong assumption on safety, the first two approaches work approximately 35 \% and 40\% worse. 
\begin{figure}[H]
	\centering 
	\subfloat[Efficiency index]{\includegraphics[width=0.45\textwidth]{Figures/efficiencybar1.png}\label{eff}}
	\hspace{0.3cm}
	\subfloat[Safety index]{\includegraphics[width=0.45\textwidth]{Figures/safetybar1.png}\label{saf}}
	\vspace{0.0cm}
	\captionsetup{font=footnotesize,labelfont=footnotesize}
	\caption{Comparison of performance metrics between the 3 spatio-temporal approaches} 
	\label{compeffsaf1}
\end{figure}
\begin{figure}[H]
	\centering
	\includegraphics[width=85mm]{Figures/safety_eff.png}
	\caption{Safety-efficiency chart}
	\label{measureeffsaf}
\end{figure} 
% \caption{Measure of safety and efficiency of the algorithms} 
%\begin{figure}[h]
%	\centering
%	\subfloat[Travel planning success ratio]{\includegraphics[width=0.46\textwidth]{Figures/planning_new.png}\label{planningratio}}
%	\hspace{0.1cm}
%	\subfloat[Safety-efficiency chart]{\includegraphics[width=0.46\textwidth]{Figures/safety_eff.png}\label{safetyeff}}
%	\caption{Measure of safety and efficiency of the algorithms} 
%	\label{measureeffsaf}
%\end{figure}

As said before, the safety index is related to the type of approach: 
\begin{itemize}[itemsep=0pt]
	\item \textbf{\textit{Sum}} has a 100\% safety because it overestimates the risk of the waves by considering all predictions at any moment of the navigation;
	\item \textbf{\textit{Global}} has a 100\% safety too since the algorithm prevents sailing in case of possible sinkings;
	\item In Figure \ref{compeffsaf1}, \textbf{\textit{Local}}'s bar indicates safety of around 70\%, as confirmed in the chart of Figure \ref{measureeffsaf}. This value was computed by detecting potential collisions of the generated paths with the hourly forecasts. 
\end{itemize}
% and \textbf{\textit{Global}} have a value of 100\%, while \textbf{\textit{Local}}, due to its strong assumption regarding its operation,  a focus entirely on the safety at the expense of the efficiency.
%This value, besides the possibility of navigation during the day, depends on the probability of finding the goal in safe zone when the vessel will arrive there. 
 
In conclusion, considering the results of the algorithm comparison in \autoref{second} and of the actual real simulations, RRT* turns out to be the preferable algorithm for this study. %Instead, we should exclude the potential field method, which is not good in this application, despite of its ability to turning around circle obstacles and its low computing time.
For the spatio-temporal approaches we can conclude that:
\setstretch{1.4}
\begin{itemize}[itemsep=0pt]
	% [\color{green}{\cmark}]
	\item [\textbf{\textit{Sum}}]
	\item[\color{green}{\cmark}] Resulting paths are always in safe zone;
	\item[\color{red}{\xmark}] It is too conservative since efficiency and travel time performance are low;
	\item[\color{red}{\xmark}] As long as the goal is in high waves in one time window, the vessel does not sail.
	\item [\textbf{\textit{Global}}]
	\item[\color{green}{\cmark}] No approximation in the map generation since all time windows are considered independently;
	\item[\color{green}{\cmark}] High travel time performance in case it reaches destination;
	\item[\color{red}{\xmark}] Forecasts are not considered: low efficiency due to potential sinkings that could be avoided. 
	\item[\textbf{\textit{Local}}]
	\item[\color{green}{\cmark}] High efficiency and travel time performance;
	\item[\color{red}{\xmark}] Too strong assumption, not sufficient to always ensure USV safety, because $f_1 \geq t_1$ (view in Figure \ref{localassumption}).
\end{itemize}
\begin{figure}[h]
\centering
\includegraphics[width=60mm]{Figures/localband.png}
\caption{\textbf{\textit{Local}}'s assumption}
\label{localassumption}
\end{figure}
\setstretch{1.5}
\section{Summary}
These spatio-temporal approaches, designed to predict changing shape obstacles, have highlighted issues that prevent their use in real applications.

Considering all marine forecasts \textit{a priori} cannot be feasible because the overestimation strongly reduces performance. As well as ignoring future wave predictions, given that relying on reactive planning when marine conditions change can cause delays or possible failures (\textbf{\textit{Sum}} and \textbf{\textit{Global}} have an efficiency below 60\%). This latter could be applicable if the temporal resolution of NetCDF files were smaller, resulting in a higher re-planning frequency.\\ \textbf{\textit{Local}} represents the approach that maximizes the efficiency of the travel at the cost of safety. Indeed, it is based on an assumption that cannot be applied in reality. The results show the need to develop an advanced method that firstly ensures USV safety and then converges to the \textbf{\textit{Local}}'s value of efficiency. In the next chapter, we will discuss it.

%From the results, global planning fails sometimes to find a feasible path since it does not consider the forecasts, and so it can be in a place which is not safe when map changes.
%To conclude, global planning in this study is not useful, not considering forecast causes to find not feasible path. Sum planning, that means a planning where I consider all forecast in a single map finds a feasible path when the goal is in a safe location all the time and there is always a route to reach it. Moreover it's not optimal at all. 
%\textbf{\textit{Sum}}, \textbf{\textit{Global}} , and \textbf{\textit{Local}} , are three methods designed to deal with the prediction of changing shape obstacles. As seen, they are not optimal. Furthermore, they do not always guarantee the USV safety. 



\chapter{Improved spatio-temporal map}
\label{improvedmethod}
Results of \autoref{results} have shown the need to develop an approach that generates a spatio-temporal map that accounts for wave forecasts to constrain the navigation in safe zones.
%with the certainty of finding in safe zone during the navigation.
This chapter covers its development, introducing the basic idea and the creation steps. Then, the improved method is tested with simulations performed by the path planning algorithm RRT*.

\section{Review of basic spatio-temporal approaches}
\textbf{\textit{Sum}}, \textbf{\textit{Global}}, and \textbf{\textit{Local}}'s features represent the basis for creating a more accurate map. For example, the idea of summing time windows is necessary to consider wave conditions in the nearest future. The \textbf{\textit{Local}}'s concept of dividing the configuration space into temporal bands is the key principle of the new method. Finally, the possibility of re-planning every amount of time, conducted by the \textbf{\textit{Global}} approach, could be implemented in the future if an update of forecasts is provided.
%each of them have some aspects that could be exploited for the development of the improved method. 
% conducted
Regarding the \textbf{\textit{Sum}} approach, adding the forecasts of every hour throughout the region of interest is ineffective. By dividing the configuration space into temporal bands, we can see that wave conditions at $t_{i-1}$ are negligible in the temporal band $i$. The reason is that the vessel will take at least one hour to cross each band, and so the added time window would represent information about the past. Figure \ref{improved1} displays this statement.
\begin{figure}[H]
	\centering
	\includegraphics[width=65mm]{Figures/improved1.png}
	\caption{Improvement of \textbf{\textit{Sum}} approach}
	\label{improved1}
\end{figure}

On the other hand, the map generated by \textbf{\textit{Local}} corresponds to the ideal case in which the boat will reach its destination by crossing each strip in precisely 1 hour. Therefore, the aim is to generate a global weather map using the least number of time windows to guarantee that the vessel will always remain in the navigation area. This approach solves the \textbf{\textit{Sum}} and \textbf{\textit{Global}}'s problem of low efficiency and the strong assumption on which \textbf{\textit{Local}} is designed that penalizes the safety.
% that is a middle ground between the two approaches,
\section{Optimal map generation}
The improved approach represents a middle ground between \textbf{\textit{Local}} and \textbf{\textit{Sum}}, and it consists in starting from the ideal path generated by \textbf{\textit{Local}} - since it is always feasible - updating the map, and recalculating the entire path. In order to facilitate this operation, we created a multidimensional masked array that includes the presence of wave conditions at $t_i$, where $i=0,1,\dots,N-1$, in every temporal band $b_j$, where $j=0,1,\dots,N_b-1$ (see Figure \ref{bandmap567}). 
\newpage
\begin{figure}[H]
	\centering 
	\subfloat[\texttt{band\_map[5][5]}]{\includegraphics[width=0.32\textwidth]{Figures/bandmap55.png}\label{bandmap55}}
	\hspace{0.1cm}
	\subfloat[\texttt{band\_map[5][6]}]{\includegraphics[width=0.32\textwidth]{Figures/bandmap56.png}\label{bandmap56}}
	\hspace{0.1cm}
	\subfloat[\texttt{band\_map[5][7]}]{\includegraphics[width=0.32\textwidth]{Figures/bandmap57.png}\label{bandmap57}}
	\vspace{0.0cm}
	%\captionsetup[subfigure]{font=tiny,labelfont=tiny}
	%\captionsetup{font=normalsize,labelfont=footnotesize}
	\caption{$H_{m0}>0.8$ of time window $t_5, t_6, t_7$ in band $b_5$ for an arbitrary day} 
	\label{bandmap567}
\end{figure}

The algorithm can be schematized in the following steps:
\begin{enumerate}[itemsep=0pt]
	\item Perform path planning using \textbf{\textit{Local}} approach;
	\item Find the intersections of the path with the circular bands: \\ 
	$P_j$, $\forall j \in \{0,1,\ldots,N_b-1\}$ where $N_b$ is the number of intersections until the survey point;
	\item Calculate the travel time $f_j$ between consecutive intersections $P_j,P_{j+1}$ and the corresponding time from the starting point ${f_{tot}}_j = \sum_{k=0}^{j}f_k$ ;
	\item Based on ${f_{tot}}_j$, adjust the number of time windows to the minimum next higher integer and regenerate the map;
	\item Repeat the path planning with the new map of obstacles;
	\item Start the next iteration of the loop from step 2 and verify that ${f_{tot}}_j< {T_{max}}_j$, which is the maximum time to cross band $b_j$ while preserving USV safety;
	\item If condition 6 is met, return the path. Otherwise, continue from step 4.
\end{enumerate}
%A map generated in this way provides a better path than the one obtained with \textbf{\textit{Sum}} because fewer obstacles will be present in the configuration space $\mathcal{C}_{space}$.
%This modification represents just a slight improvement. That is why another approach has been conceived.
%\textcolor{red}{The optimal solution would a tradeoff between all the methods. Sum is the suboptimal but it assures not to go in unsafe zone. Global is a possibility, and we could account for a grade of prediction on how the wavefront moves based on velocity and direction of the waves and/or currents. Local method is the best but it still an approximation. An improved version could be overlapped bands, with length which depends on the time distance between the theoretical end point of the band, and the one actually reached. Indeed the drawback of this method is that the boat will never reach the end of each band in one hour}.
The pseudo-code of the improved method is illustrated in Algorithm \ref{improvedlocal}. 

\vspace{0.8cm}
Figure \ref{compsumlocimpr} compares the maps generated by three spatio-temporal approaches for the same day. Depending on how the time windows are summed, $\mathcal{C}_{obs}$ will change, resulting in different deliberative paths. \textbf{\textit{Sum}} and \textbf{\textit{Local}} approaches are completely opposite in the map development. Conversely, the third map is the product of the presented algorithm: it is made based on the minimum travel time necessary to reach the destination, preserving the condition of safety during navigation.
\newpage
\begin{algorithm}[h]
	\small
	\caption{\textbf{\textit{Improved method}} approach}\label{improvedlocal}
	\KwData{\\
		%$H_{m0}\gets array(time, latitude, longitude)$\;
		%$cost \gets$ \textit{boolean} $array(time, latitude, longitude)$\;
		$N \gets$ \textit{number of available time windows}\;
	    $N_b \gets$ \textit{number of temporal bands to reach destination}\;
       	$[(o_x,o_y)]\gets$ \textit{list of x and y coordinates of obstacles of \textbf{Local} approach}\;
        $image \gets $ \texttt{np.zeros}((len(latitude), len(longitude)));}

	\KwResult{Points $(x,y)$ of the final path}
	\Comment{\textcolor{blue}{// Creation of the multidimensional masked array}}
	\For{$k$ in range(len(latitude))}{
		\For{$j$ in range(len(longitude))}{
			$geodist \gets$ \textit{geodetic distance between point}\texttt{[k][j]} \textit{and the harbor}\;
			\textit{depending on $geodist$ select the temporal band $b_q$ and generate the}\\ \textit{masked array} \texttt{band\_map[q][s][k][j]}, \textit{where} \\ $q=0,\ldots,N_b-1$, \textit{and} $s=0,\dots,N-1$;}}
			%\For{q in range($N_b$)}{
				%\If{$18(q)<$ \textit{geo distance} $<18(q+1)$}{
					%\textit{generate the mask image band\_map\texttt{[q][s][k][j]}};
					%\For{s in range($N$)}{
					%	\If{cost\texttt{[s][k][j]}}{
					%		band\_map\texttt{[q][s][k][j]}=1;
					%	}
					%}
				%}
			%}
	\textit{Run the planner using} $[(o_x,o_y)]$ \textit{as obstacles}; \Comment{\footnotesize \textcolor{blue}{// Planning with \textbf{\textit{Local}}}}\
	$T_{max}=[1,2,\ldots,N_b+1]$\;
	\While{condition 6 is not \texttt{True}}{
		\textit{oversample the resulting path and the edges of the bands}\;
		\textit{find the intersections} $P_i$ using \texttt{np.allclose}\;
		\For{$i$ in range($N_b$)}{
		\textit{compute} $f_i$ and ${f_{tot}}_i$\;
		\textit{verify condition} 6: ${f_{tot}}_i < {T_{max}}_i$;}
		\If{condition 6 $=$ \texttt{True}}{
			\textit{return} $(x,y)$ \textit{points of the safest path};}
		\Else{
			\textit{image $\gets$ weather map update by adding contributions of} \texttt{band\_map}\;
			\textit{recalculate} ${T_{max}}_i$\;
			\textit{wave contour} $=$ \texttt{np.logical\_xor}(\textit{image,} \texttt{binary\_erosion}(\textit{image}))\;
			\textit{repeat the path planning with RRT*};}
		}
		%path\_h = [\texttt{round}(path\_dist[j]/18,2) for $j$ in range(len(path\_dist))]\;
		%path\_until\_map = [\texttt{round}(\texttt{np.sum}(path\_h\texttt{[:j]}),2) for $j$ in range(1,len(path\_dist)$+1$)]
\end{algorithm}
\begin{figure}[H]
	\centering 
	\subfloat[]{\includegraphics[width=0.32\textwidth]{Figures/costsumtry.png}\label{sumcost}}
	\hspace{0.1cm}
	\subfloat[]{\includegraphics[width=0.32\textwidth]{Figures/costlocaltry.png}\label{localcost}}
	\hspace{0.1cm}
	\subfloat[]{\includegraphics[width=0.32\textwidth]{Figures/tryband.png}\label{imprcost}}
	\vspace{0.0cm}
	\caption{Difference between the maps made by three approaches} 
	\label{compsumlocimpr}
\end{figure}
\newpage
Figure \ref{localproblem} depicts two examples that confirms the problem of \textbf{\textit{Local}}'s initial hypothesis. In the first one, we see the route that avoids the obstacles viewed by \textbf{\textit{Local}}. From the travel time marked at the end of each band, we can note that the boat will take more than 1 hour to cross the temporal bands, especially in the second part of the navigation. This action must be addressed by considering additional time windows. For example, since the first band is traveled in 1.06 h $> 1$ h, we will consider $t_0+t_1$ for $b_0$; for the second band $1.06+0.96>2$ h, meaning $t_1+t_2$ and so on. The re-planned path is represented in the right chart: in this case, the condition of ${f_{tot}}_j< {T_{max}}_j$ is met, and this is verified by the fact that the unsafe region for a possible next iteration overlaps with the current map. \\ 
%This example proves that \textbf{\textit{Local}} does not always guarantee safe navigation, unlike the new approach.
%The other figures reiterate this statement: 
In the second example, using \textbf{\textit{Local}} may cause sinking, while with the \textbf{\textit{Improved method}}, the boat does not sail because wave conditions prevent navigation.  

\begin{figure}[H]
	\centering 
	\captionsetup{font=footnotesize,labelfont=footnotesize}
	\subfloat[]{\includegraphics[width=0.40\textwidth]{Figures/loc+1impr.png}\label{localiter}}
	\hspace{0.5cm}
	\subfloat[]{\includegraphics[width=0.40\textwidth]{Figures/loc+3impr.png}\label{impriter}}
	\hspace{0.5cm}
	\subfloat[]{\includegraphics[width=0.40\textwidth]{Figures/locsiimpno.png}\label{locsi}}
	\hspace{0.5cm}
	\subfloat[]{\includegraphics[width=0.40\textwidth]{Figures/locsiimpno2.png}\label{imprno}}
	\vspace{0.0cm}
	%\captionsetup[subfigure]{font=tiny,labelfont=tiny}
	\caption{(a),(c) Safety risk and potential sinking with \textbf{\textit{Local}}. (b),(d) Safety guaranteed and navigation not allowed with \textbf{\textit{Improved method}}.}
	\label{localproblem}
\end{figure}
% Map correction from \textbf{\textit{Local}} to the \textbf{\textit{Improved method}} approach
\section{Simulations}
The new approach is tested in 12 simulation days, with two survey points and considering the reverse routes (some of these latter are displayed in the \hyperref[appendix]{Appendix}). Only RRT* has been exploited for path planning, having obtained better results than A* and APF in \autoref{results}. Below here are the two detailed tables with travel time $T_{path}$ and computing time $t_{exec}$ of the nautical routes from Porto Vecchio. 
\begin{figure}[H]
	\centering
	\subfloat[Day 1]{\includegraphics[width=0.32\textwidth]{Figures/day1impr1.png}\label{day1impr1}}
	\hspace{0.1cm}
	\subfloat[Day 2]{\includegraphics[width=0.32\textwidth]{Figures/day2impr1.png}\label{day2impr1}}
	\hspace{0.1cm}
	\subfloat[Day 3]{\includegraphics[width=0.32\textwidth]{Figures/day3impr1.png}\label{day3impr1}}
	\hspace{0.1cm}
	\subfloat[Day 4]{\includegraphics[width=0.32\textwidth]{Figures/day4impr1.png}\label{day4impr1}}
	\hspace{0.1cm}
	\subfloat[Day 5]{\includegraphics[width=0.32\textwidth]{Figures/day5impr1.png}\label{day5impr1}}
	\hspace{0.1cm}
	\subfloat[Day 6]{\includegraphics[width=0.32\textwidth]{Figures/day6impr1.png}\label{day6impr1}}
	\hspace{0.1cm}
	\subfloat[Day 7]{\includegraphics[width=0.32\textwidth]{Figures/day7impr1.png}\label{day7impr1}}
	\hspace{0.1cm}
	\subfloat[Day 8]{\includegraphics[width=0.32\textwidth]{Figures/day8impr1.png}\label{day8impr1}}
	\hspace{0.1cm}
	\subfloat[Day 9]{\includegraphics[width=0.32\textwidth]{Figures/day9impr1.png}\label{day9impr1}}
	\vspace{0.0cm}
	\caption{Argentario survey Day 1 to 9 - \textbf{\textit{Improved method}}} 
	\label{Survey1impr}
\end{figure}
\newpage
\begin{figure}[H]
	\centering 
	\subfloat[Day 10]{\includegraphics[width=0.32\textwidth]{Figures/day10impr1.png}\label{day10impr1}}
	\hspace{0.1cm}
	\subfloat[Day 11]{\includegraphics[width=0.32\textwidth]{Figures/day11impr1.png}\label{day11impr1}}
	\hspace{0.1cm}
	\subfloat[Day 12]{\includegraphics[width=0.32\textwidth]{Figures/day12impr1.png}\label{day12impr1}}
	\vspace{0.0cm}
	%\captionsetup{font=normalsize,labelfont=footnotesize}
	\caption{Argentario survey Day 10 to 12 - \textbf{\textit{Improved method}}} 
	\label{Survey1impr1}
\end{figure}
\begin{table}[!htbp]
	\centering
	\resizebox{\textwidth}{!}{%
		\renewcommand{\arraystretch}{1.5}
		\begin{tabular}{|c|c|c|c|c|c|c|c|c|c|c|c|c|c|}
			\hline
			\multicolumn{2}{|c|}{} & \textbf{Day 1} & \textbf{Day 2} & \textbf{Day 3} & \textbf{Day 4} & \textbf{Day 5} & \textbf{Day 6} & \textbf{Day 7} & \textbf{Day 8} &\textbf{Day 9} & \textbf{Day 10} & \textbf{Day 11} & \textbf{Day 12} \bigstrut\\
			\hline
			\multicolumn{1}{|c|}{\multirow{2}[4]{*}{\parbox{5em}{\centering \textbf{RRT*} \textit{Improved method}}}} & $T_{path}$($h$) & 9.88  & 9.12  & 10.20  & 10.42  & 10.20  & 10.15 & 11.00  & 9.78  & 9.51  & 9.51  & 9.88  & 9.68 \bigstrut\\
			\cline{2-14}          & $t_{exec}$($s$) & 94.917 & 73.043 & 96.444 & 117.157 & 107.686 & 106.792 & 118.273 & 74.561 & 114.337 & 92.934 & 83.794 & 82.465 \bigstrut\\
			\hline
	\end{tabular}}%
	\caption{Argentario survey - Travel and computing time}
	\label{tab:addlabel3}%
\end{table}%
\begin{figure}[H]
	\centering 
	\subfloat[Day 1]{\includegraphics[width=0.32\textwidth]{Figures/day1impr2.png}\label{day1impr2}}
	\hspace{0.1cm}
	\subfloat[Day 2]{\includegraphics[width=0.32\textwidth]{Figures/day2impr2.png}\label{day2impr2}}
	\hspace{0.1cm}
	\subfloat[Day 3]{\includegraphics[width=0.32\textwidth]{Figures/day3impr2.png}\label{day3impr2}}
	\hspace{0.1cm}
	\subfloat[Day 4]{\includegraphics[width=0.32\textwidth]{Figures/day4impr2.png}\label{day4impr2}}
	\hspace{0.1cm}
	\subfloat[Day 5]{\includegraphics[width=0.32\textwidth]{Figures/day5impr2.png}\label{day5impr2}}
	\hspace{0.1cm}
	\subfloat[Day 6]{\includegraphics[width=0.32\textwidth]{Figures/day6impr2.png}\label{day6impr2}}
	\hspace{0.1cm}
	\vspace{0.0cm}
	\caption{Elba Island survey Day 1 to 6 - \textbf{\textit{Improved method}}} 
	\label{Survey2impr1}
\end{figure}
\newpage
\begin{figure}[H]
	\centering
	\subfloat[Day 7]{\includegraphics[width=0.32\textwidth]{Figures/day7impr2.png}\label{day7impr2}}
	\hspace{0.1cm}
	\subfloat[Day 8]{\includegraphics[width=0.32\textwidth]{Figures/day8impr2.png}\label{day8impr2}}
	\hspace{0.1cm}
	\subfloat[Day 9]{\includegraphics[width=0.32\textwidth]{Figures/day9impr2upd.png}\label{day9impr2}}
	\hspace{0.1cm}
	\subfloat[Day 10]{\includegraphics[width=0.32\textwidth]{Figures/day10impr2.png}\label{day10impr2}}
	\hspace{0.1cm}
	\subfloat[Day 11]{\includegraphics[width=0.32\textwidth]{Figures/day11impr2.png}\label{day11impr2}}
	\hspace{0.1cm}
	\subfloat[Day 12]{\includegraphics[width=0.32\textwidth]{Figures/day12impr2.png}\label{day12impr2}}
	\vspace{0.0cm}
	\caption{Elba Island survey Day 7 to 12 - \textbf{\textit{Improved method}}} 
	\label{Survey2impr}
\end{figure}
% Table generated by Excel2LaTeX from sheet 'Foglio1'


\begin{table}[!htbp]
	\centering	
	\resizebox{\textwidth}{!}{%
		\renewcommand{\arraystretch}{1.5}
		\begin{tabular}{|c|c|c|c|c|c|c|c|c|c|c|c|c|c|}
			\hline
			\multicolumn{2}{|c|}{} & \textbf{Day 1} & \textbf{Day 2} & \textbf{Day 3} & \textbf{Day 4} & \textbf{Day 5} & \textbf{Day 6} & \textbf{Day 7} & \textbf{Day 8} &\textbf{Day 9} & \textbf{Day 10} & \textbf{Day 11} & \textbf{Day 12} \bigstrut\\
			\hline
			\multicolumn{1}{|c|}{\multirow{2}[4]{*}{\parbox{5em}{\centering \textbf{RRT*} \textit{Improved method}}}} & $T_{path}$($h$) & 8.56  & 8.47  & 8.83  & 8.72  & 8.58  & 8.66  & 9.04  & 8.76  & 9.83  & 8.79  & 8.79  & 8.49 \bigstrut\\
			\cline{2-14} & $t_{exec}$($s$) & 84.639 & 74.349 & 87.418 & 87.165 & 81.896 & 79.588 & 142.700 & 79.866 & 75.743 & 79.065 & 81.206 & 114.583 \bigstrut\\
			\hline
	\end{tabular}}%
	\caption{Elba Island survey - Travel and computing time}
	\label{tab:addlabel5}%
\end{table}%

By carefully comparing the charts just plotted with the ones of \textbf{\textit{Local}} in \autoref{results}, there are no substantial differences. However, those small changes in the maps could mean the success or failure of the whole mission, as seen in Figure \ref{localproblem}. The most evident changes in $\mathcal{C}_{obs}$ occur on Day 4, 5, 6, 9, 10, and 11.\\
%One last note regarding the high computing time: 
From Table \ref{tab:addlabel3} and \ref{tab:addlabel5}, the values of $t_{exec}$ catch the eye at first glance. This aspect is due to the expensive functions of the recursive algorithm, which repeats its operations until it finds a viable path. Nevertheless, we remember that time optimization is unnecessary for this application, and computing time of a couple of minutes is acceptable.
\newpage
\section{Results}
The performance of the \textbf{\textit{Improved method}} is evaluated through a comparison of the path lengths and a measure of the efficiency and safety index.

Figure \ref{traveltimeimpr} compares the four approaches in the simulations performed by RRT*. 
%Excluding \textbf{\textit{Sum}}, whose possible paths are $7.1\%$ longer, and \textbf{\textit{Global}} the other two have provided very close results. 
\textbf{\textit{Local}} represents the fastest approach, followed closely by the \textbf{\textit{Improved method}}, and then \textbf{\textit{Global}} and \textbf{\textit{Sum}}, whose possible paths are respectively $2.3\%$ and $7.3\%$ longer. A $0.1\%$ variation of the new approach denotes a remarkable similarity with \textbf{\textit{Local}}, even though it considers more time windows.
%to compare the two metrics: in the basic approaches, it is difficult to achieve high values in both cases. With the new approach both metrics are maximized.
\vspace{1cm}
\begin{figure}[h]
	\centering
	\includegraphics[width=75mm]{Figures/barimprovednew2.png}
	\caption{Comparison of travel time using RRT* divided by approaches}
	\label{traveltimeimpr}
\end{figure} 
\vspace{0.5cm}
\begin{table}[htbp]
	\centering
	\renewcommand{\arraystretch}{2}
	\begin{tabular}{|c|c|c|c|c|c|}
		\hline
		\multicolumn{2}{|c|}{} & \textit{Sum}  & \textit{Global} & \textit{Local} & \parbox{4em}{\centering \small \textit{Improved method}} \bigstrut[b]\\
		\hline
		\multicolumn{1}{|c|}{\multirow{2}[4]{*}{\textbf{RRT*}}} & \parbox{4em}{\centering \small Mean $T_{path}$($h$)} & 9.6  & 9.19  & 8.99  & 9.0 \bigstrut[b]\\
		\cline{2-6}          & \parbox{4em}{\centering \scriptsize Variation (\%) w.r.t. $min(T_{path})$} & 7.3   & 2.3   & 0.0   & 0.1 \bigstrut[b]\\
		\hline
	\end{tabular}%
	\caption{Data comparison of travel time using RRT* divided by approaches}
	\label{tabl improved}%
\end{table}%
\newpage
Table \ref{tabl improved} exposes the quantitative results depicted in the previous figure. These outcomes are related to the respective degree of safety, illustrated in Figure \ref{compeffsaf} and \ref{safeeffimpr} together with the efficiency index. Safety determines the amount of obstacles on the map, so higher values push the boat to be more careful and find longer paths.

From the barplots, we can see how the \textbf{\textit{Improved method}} achieves similar values to \textbf{\textit{Local}}, reaching an efficiency of 85.4\%, still preserving the 100\% safety assurance, as \textbf{\textit{Sum}} and \textbf{\textit{Global}}. The same results are summarized in the 2D plot of Figure \ref{safeeffimpr}.
\begin{figure}[H]
	\centering 
	\subfloat[Efficiency index]{\includegraphics[width=0.47\textwidth]{Figures/efficiencyimpr.png}\label{effimpr}}
	\hspace{0.3cm}
	\subfloat[Safety index]{\includegraphics[width=0.47\textwidth]{Figures/safetyimpr.png}\label{safimpr}}
	\vspace{0.0cm}
	\captionsetup{font=footnotesize,labelfont=footnotesize}
	\caption{Comparison of performance metrics between the 4 spatio-temporal approaches} 
	\label{compeffsaf}
\end{figure}
\begin{figure}[h]
	\centering
	\includegraphics[width=85mm]{Figures/safety_effimpr.png}
	\caption{Efficiency-safety chart}
	\label{safeeffimpr}
\end{figure} 
\section{Summary}
In this chapter, we have developed an innovative approach to consider large hourly time-varying areas of obstacles \textit{a priori}, represented by risky wave conditions.\\
The \textbf{\textit{Improved method}} has achieved excellent results at the cost of computing time of a few minutes. It has solved the two issues encountered with the basic approaches regarding efficiency and safety during navigation.
%\section{Uncertainty of the forecasts} 
%We know that predictions suffer of a grade of uncertainty: they are not exact, especially if they provide information at very small resolution, as in this case. Although they are short-range forecasts (1-2 days), it’s better to account for the variability.




\chapter{Conclusion}
%\addcontentsline{toc}{chapter}{Conclusion}
\label{conclusions}
%atmospheric disturbances at sea with the aim of improving the navigation of autonomous surface vessels, and even manned boats.
% that avoid the predicted obstacle movement
%This thesis has focused on developing a path planning approach that considers wave forecasts during path computation. Given that weather frequently changes at sea and can significantly affect vehicles' motion, autonomous boats need to consider this potential disturbance before actually sailing. 
This thesis has proposed a path planning approach that considers wave forecasts during path computation of autonomous surface vessels. Given that weather frequently changes at sea, the boat has to make appropriate decisions considering potential disturbances based on the forecast of the sea state.

The implementation of meteorological data in the path planning problem has been performed using NetCDF files, which store sea forecasts in a multidimensional array form with hourly temporal resolution. Moreover, they have been employed to examine time averages, wave distribution, and direction of origin in order to ensure safer navigation. This format has achieved good results in terms of performance, simplicity, and usability in simulations and data processing.

The environmental map modeling is the operating step that mostly affects path planning results. Therefore we focused on generating an accurate map that included the time-varying unsafe marine areas expected for the following hours. 
Three methods have been conceived based on how hourly time windows are considered: \textbf{\textit{Sum}}, \textbf{\textit{Global}}, and \textbf{\textit{Local}}. They have been tested over 12 simulation days in the region of the Tyrrhenian sea, considering two nautical routes in both directions, and using three path planners, i.e., the improved version of Rapidly-Exploring Random Tree (RRT*), A*, and the Artificial Potential Field (APF).
The first results have shown a low performance of the three methods in terms of efficiency or navigation safety, highlighting the need to improve the model of the map. So we designed the \textbf{\textit{Improved method}}. \\
By integrating the division in temporal bands and the sum of temporal maps in a recursive algorithm, it creates a forecast map that prioritizes both the USV safety and the planning efficiency. The map is then used by the asymptotically optimal algorithm, RRT*, to build the shortest global path.
Its performance of 85.4\% efficiency and 100\% safety shown in \autoref{improvedmethod} proves that the model of an accurate environmental map designed by combining time windows of marine forecasts can be employed to generate deliberative and safe paths predicting the shifting of dynamic obstacles \textit{a priori}. 
%The last proposed approach is called \textbf{\textit{Improved method}}. It takes its name from the fact that it represents a better version than the originally conceived three spatio-temporal approaches. By integrating the division in temporal bands and the sum of temporal maps in a recursive algorithm, it creates a forecast map that includes the time-varying unsafe marine areas expected for the following hours. The map is then used by the asymptotically optimal algorithm, RRT*, to build the shortest global path. This functionality permits prioritizing both the USV safety and the optimality of the route.\\
%The methodology based on NetCDF files derives from how to consider marine weather factors in a path planning problem. This format has achieved good results in terms of performance, simplicity, and usability in the simulations and for data processing. Moreover, it allowed carrying out an accurate analysis of the sea state necessary to ensure safe navigation.\\
%Thus, the aim is to enhance the USV performance for environmental monitoring applications.
\vspace{0.3cm}\\
In this thesis, we have used sea waves as obstacles, particularly the significant wave height $H_{m0}$, based on the assumption that they represent the most influential marine factor for the safety of a small vessel. Additional environmental data could be embedded to make the simulations more realistic and prepare the vehicle for a real-world application. Examples include the velocity of sea currents, wave direction, and wind forces, many of which can be directly downloaded in NetCDF format. In this regard, Figure \ref{currentspeed} displays the current speed and direction of 2021 in the concerned Mediterranean region, along with wave direction.
%Looking more broadly, precipitation and solar irradiance may considerably affect the USV mission. 

\begin{figure}[h]
	\centering
	\includegraphics[width=90mm]{Figures/wavedircurr2021.png}
	\caption{Annual mean current speed in 2021}
	\label{currentspeed}
\end{figure} 
Future work should focus on two main objectives: improving the global path planning solution and developing the control and decision system while sailing. The first can be achieved by operating on the following aspects:
\begin{itemize}[itemsep=0pt]
	\item Choice of the path planning algorithm to reduce computing time and improve path smoothness;
	%\item Reduction of the computing time;
	%\item Route optimization: improving path smoothness;
	\item Consider forecast uncertainty, as mentioned at the end of \autoref{approach};
	\item Increase the number of simulations to validate the method proposed.
\end{itemize}
The other objective should examine the boat's behavior while effectively in motion, which means analyzing the local planning of the vehicle. These include two major topics in which time and effort could be invested:
\begin{itemize}[itemsep=0pt]
	\item Model the dynamics of the vessel and design a control system for unpredictable sea disturbances;
	\item Add visual sensors with artificial intelligence to augment USV real-time localization and adaptability. The increased perception may alert the boat to sudden weather changes and make appropriate decisions accordingly \cite{zhang2021unmanned}.
\end{itemize}

In conclusion, the insights gained with this work could be applied in the future, paving the way for a stronger connection between satellite data, \textit{in situ} data, and autonomous navigation. This thesis opens up the possibility of embedding weather in the decision-making process of surface vessels with potential implications for shipping and monitoring applications of marine robots.
% , given that marine vessels are more affected by the environment respect to classic robots.
%this work builds a stronger connection between autonomous navigation and the environmental field, allowing considering sea parameters without using additional sensors. 
\vspace{0.5cm}
\begin{figure}[h]
	\centering
	\includegraphics[width=80mm]{Figures/wittedusv1.JPG}
	\caption{Argo USV at the Argentario}
	\label{wittedargo}
\end{figure} 
%data measurements of the boat increasing navigation safety, optimize missions and the autonomy during environmental monitoring.
%The spatio-temporal path planning approach opens up 
%The current study represents the starting point of a more comprehensive project that concerns the development of a weather avoidance decision-making system that generates human-like driving strategies, including target selection and route optimization, based on the understanding of the environment. 
%The path planning results represent a fundamental aspect that the decision-making process will take into account to decide the daily mission. 
%Working on these methods, we understood that there was a flaw in each of them that prevent their use in real applications. Therefore we designed a new method that could improve the others' performance and guarantee USV safety for an eventual future implementation.
%The meteorological data used are stored in multidimensional array form as NetCDF files with hourly temporal resolution. They have been used to see the daily and monthly mean of the waves but also examine their distribution, and the direction of origin with the wave rose plots. They allow, above all, to model static obstacles, i.e., the islands, and the dynamic obstacles, i.e., the waves.
%They allow to analyze deeper the sea state, like the period where it is preferable to navigate, the time mean, the distribution of the waves and the direction of origin  


%The first one sums all maps, the second considers one map at a time, and the third divides the space in annulus which behaves as a time window and include only the corresponding obstacles of that hour.

%Global is the perfect method for safety. Moreover sometimes it can assure better results than Local and Improved because it does  not suffer of the time resolution mapping.
%Often we get confused between the actual athmospheric conditions, which could be measured in real-time by sensors and weather predictions, which instead cannot be measured or embedded autonomously by the vessel in order to perform a reaction. This thesis proposes such an approach. Moreover, people can think that the appropriate vehicles could withstand athmosferic factors at sea. The answer is that "if it was possible to avoid these weather disturbances and perform a safe navigation, why don't do that?"
%A weather avoidance system could significantly improve the autonomous mission capability leading to faster surveillances, and coastal monitoring.

\backmatter
\cleardoublepage
\addcontentsline{toc}{chapter}{\bibname}
\setstretch{1}
\bibliographystyle{plainnat}
\bibliography{Bibliography}

%\pagenumbering{roman}
\appendix
\setstretch{1.5}
\chapter{Appendix}
\fancyhead{}
\fancyhead[RO]{\slshape\nouppercase{Appendix}}
\fancyhead[LE]{\slshape\nouppercase{Appendix}}
\label{appendix}
This appendix includes a series of additional simulation days and the related tables with the travel time and computing time of each run. The chosen days indicate the different mapping between the three approaches, but above all, between \textbf{\textit{Local}} and \textbf{\textit{Improved method}}. 

From Table \ref{tabell4} and \ref{tabella5}, we can note that the boat cannot sail on Day 7, 8, and 9. The reason is that the destination spot will be in the unsafe region once reached.
A straightforward solution could be changing the departure time or focusing on other target areas.
\begin{table}[htbp]
	\centering
	\resizebox{\textwidth}{!}{%
		\begin{tabular}{|c|c|c|c|c|c|c|c|c|c|c|c|c|c|}
			\hline
			\multicolumn{2}{|c|}{\multirow{2}[4]{*}{}} & \multicolumn{6}{c|}{\textbf{Argentario $\longrightarrow$ Porto Vecchio harbor}} & \multicolumn{6}{c|}{\textbf{Elba Island $\longrightarrow$ Porto Vecchio harbor}} \bigstrut\\
			\cline{3-14}    \multicolumn{2}{|c|}{} & \multicolumn{2}{c|}{\textbf{Day 3}} & \multicolumn{2}{c|}{\textbf{Day 10}} & \multicolumn{2}{c|}{\textbf{Day 11}} & \multicolumn{2}{c|}{\textbf{Day 4}} & \multicolumn{2}{c|}{\textbf{Day 6}} & \multicolumn{2}{c|}{\textbf{Day 9}} \bigstrut\\
			\hline
			Algorithm & Planning & \multicolumn{1}{p{3em}|}{$T_{path}$($h$)} & \multicolumn{1}{p{3em}|}{$t_{exec}$($s$)} & \multicolumn{1}{p{3em}|}{$T_{path}$($h$)} & \multicolumn{1}{p{3em}|}{$t_{exec}$($s$)} & \multicolumn{1}{p{3em}|}{$T_{path}$($h$)} & \multicolumn{1}{p{3em}|}{$t_{exec}$($s$)} & \multicolumn{1}{p{3em}|}{$T_{path}$($h$)} & \multicolumn{1}{p{3em}|}{$t_{exec}$($s$)} & \multicolumn{1}{p{3em}|}{$T_{path}$($h$)} & \multicolumn{1}{p{3em}|}{$t_{exec}$($s$)} & \multicolumn{1}{p{3em}|}{$T_{path}$($h$)} & \multicolumn{1}{p{3em}|}{$t_{exec}$($s$)} \bigstrut\\
			\hline
			\multirow{3}[6]{*}{\textbf{RRT*}} & \textit{Global}  & 11.11 & 47.106 & \textcolor{red}{None}  & \textcolor{red}{None}  & 9.55  & 41.975 & \textcolor{red}{None}  & \textcolor{red}{None}  & \multicolumn{1}{c|}{8.59} & 24.101 & \textcolor{red}{None}  & \textcolor{red}{None} \bigstrut\\
			\cline{2-14}          & \textit{Local} & 10.08 & 4.623 & 11.28 & 6.896 & 9.00  & 2.742 & 8.63  & 11.968 & 8.58  & 5.483 & \textcolor{red}{None}  & \textcolor{red}{None} \bigstrut\\
			\cline{2-14}          & \textit{Sum} & 11.01 & 7,037 & 12.19 & 8.564 & 10.71 & 2.579 & 8.83  & 2.807 & 9.22  & 8.530 & \textcolor{red}{None}  & \textcolor{red}{None} \bigstrut\\
			\hline
			\multirow{3}[6]{*}{\textbf{A*}} & \textit{Global}  & 10.75 & 26.689 & 11.62 & 21.960 & 9.86  & 28.043 & 9.03  & 27.553 & \textcolor{red}{None}  & \textcolor{red}{None}  & \textcolor{red}{None}  & \textcolor{red}{None} \bigstrut\\
			\cline{2-14}          & \textit{Local} & 10.74 & 2.810 & 11.62 & 2.300 & 9.72  & 2.122 & 9.03  & 3.384 & 8.77  & 1.288 & \textcolor{red}{None}  & \textcolor{red}{None} \bigstrut\\
			\cline{2-14}          & \textit{Sum} & 12.27 & 3.690 & 12.65 & 2.716 & 11.11 & 1.192 & 9.28  & 3.620 & 9.91  & 1.295 & \textcolor{red}{None}  & \textcolor{red}{None} \bigstrut\\
			\hline
			\multirow{3}[6]{*}{\textbf{APF}} & \textit{Global}  & 10.42 & 0.879 & 11.87 & 0.672 & 9.14  & 0.595 & \textcolor{red}{None}  & \textcolor{red}{None}  & 8,89  & 0,528 & \textcolor{red}{None}  & \textcolor{red}{None} \bigstrut\\
			\cline{2-14}          & \textit{Local} & 10.71 & 0.044 & 12.43 & 0.086 & 8.74  & 0.025 & 8.33  & 0,056 & 8.35  & 0.049 & \textcolor{red}{None}  & \textcolor{red}{None} \bigstrut\\
			\cline{2-14}          & \textit{Sum} & 11.78 & 1.024 & 12.95 & 0.650 & 10.68 & 0.236 & 9.3567 & 0.953 & 9.41  & 0.236 & \textcolor{red}{None}  & \textcolor{red}{None} \bigstrut\\
			\hline
	\end{tabular}}%
	\caption{Additional days - Travel and computing time}
	\label{tabella3}%
\end{table}%
\begin{table}[htbp]
	\centering
	
	\resizebox{\textwidth}{!}{%
		\renewcommand{\arraystretch}{1.5}
		\begin{tabular}{|c|c|c|c|c|c|c|c|c|c|c|c|c|c|}
			\hline
			\multicolumn{2}{|c|}{} & \textbf{Day 1} & \textbf{Day 2} & \textbf{Day 3} & \textbf{Day 4} & \textbf{Day 5} & \textbf{Day 6} & \textbf{Day 7} & \textbf{Day 8} &\textbf{Day 9} & \textbf{Day 10} & \textbf{Day 11} & \textbf{Day 12} \bigstrut\\
			\hline
			\multicolumn{1}{|c|}{\multirow{2}[4]{*}{\parbox{5em}{\centering \textbf{RRT*} \textit{Improved method}}}} & $T_{path}$($h$) & 10.44  & 9.31  & 10.09  & 10.72  & \textcolor{red}{None}  & 11.19  & \textcolor{red}{None}  & \textcolor{red}{None}  & \textcolor{red}{None}  & 10.77  & 8.91  & 9.41 \bigstrut\\
			\cline{2-14} & $t_{exec}$($s$) & 155.573 & 190.166 & 87.209 & 166.97 & \textcolor{red}{None} & 151.489 & \textcolor{red}{None} & \textcolor{red}{None} & \textcolor{red}{None} & 188.916 & 62.912 & 75.372 \bigstrut\\
			\hline
	\end{tabular}}%
	\caption{Argentario $\longrightarrow$ Porto Vecchio harbor - \textbf{\textit{Improved method}}}
	\label{tabell4}%
\end{table}%

\begin{table}[htbp]
	\centering
	
	\resizebox{\textwidth}{!}{%
		\renewcommand{\arraystretch}{1.5}
		\begin{tabular}{|c|c|c|c|c|c|c|c|c|c|c|c|c|c|}
			\hline
			\multicolumn{2}{|c|}{} & \textbf{Day 1} & \textbf{Day 2} & \textbf{Day 3} & \textbf{Day 4} & \textbf{Day 5} & \textbf{Day 6} & \textbf{Day 7} & \textbf{Day 8} &\textbf{Day 9} & \textbf{Day 10} & \textbf{Day 11} & \textbf{Day 12} \bigstrut\\
			\hline
			\multicolumn{1}{|c|}{\multirow{2}[4]{*}{\parbox{5em}{\centering \textbf{RRT*} \textit{Improved method}}}} & $T_{path}$($h$) & 9.88  & 8.49  & 8.45  & 8.53  & 8.48  & 8.78  & \textcolor{red}{None}  & \textcolor{red}{None}  & \textcolor{red}{None}  & 8.56  & 8.33  & 8.46 \bigstrut\\
			\cline{2-14} & $t_{exec}$($s$) & 95.642 & 193.995 & 182.016 & 89.894 & 72.751 & 193.621 & \textcolor{red}{None} & \textcolor{red}{None} & \textcolor{red}{None} & 214.517 & 57.084 & 52.286 \bigstrut\\
			\hline
	\end{tabular}}%
	\caption{Elba Island $\longrightarrow$ Porto Vecchio harbor - \textbf{\textit{Improved method}}}
	\label{tabella5}%
\end{table}%
\newpage
\begin{figure}[H]
	\centering
	\subfloat[Day 3 - \textbf{\textit{Sum}}]{\includegraphics[width=0.32\textwidth]{Figures/day3revs.png}\label{day3rev1s}}
	\hspace{0.1cm}
	\subfloat[Day 3 - \textbf{\textit{Global}}]{\includegraphics[width=0.32\textwidth]{Figures/day3revg.png}\label{day3rev1g}}
	\hspace{0.1cm}
	\subfloat[Day 3 - \textbf{\textit{Local}}]{\includegraphics[width=0.32\textwidth]{Figures/day3revl.png}\label{day3rev1l}}
	\hspace{0.1cm}
	\subfloat[Day 10 - \textbf{\textit{Sum}}]{\includegraphics[width=0.32\textwidth]{Figures/day10revs.png}\label{day10rev1s}}
	\hspace{0.1cm}
	\subfloat[Day 10 - \textbf{\textit{Global}}]{\includegraphics[width=0.32\textwidth]{Figures/day10revg.png}\label{day10rev1g}}
	\hspace{0.1cm}
	\subfloat[Day 10 - \textbf{\textit{Local}}]{\includegraphics[width=0.32\textwidth]{Figures/day10revl.png}\label{day10rev1l}}
	\hspace{0.1cm}
	\subfloat[Day 11 - \textbf{\textit{Sum}}]{\includegraphics[width=0.32\textwidth]{Figures/day11revs.png}\label{day11rev1s}}
	\hspace{0.1cm}
	\subfloat[Day 11 - \textbf{\textit{Global}}]{\includegraphics[width=0.32\textwidth]{Figures/day11revg.png}\label{day11rev1g}}
	\hspace{0.1cm}
	\subfloat[Day 11 - \textbf{\textit{Local}}]{\includegraphics[width=0.32\textwidth]{Figures/day11revl.png}\label{day11rev1l}}
	\vspace{0.0cm}
	\caption{Argentario $\longrightarrow$ Porto Vecchio harbor - Day 3, 10, 11} 
	\label{addarg}
\end{figure}

\begin{figure}[H]
	\centering
	\subfloat[Day 4 - \textbf{\textit{Sum}}]{\includegraphics[width=0.32\textwidth]{Figures/day4revs2.png}\label{day4rev1s}}
	\hspace{0.1cm}
	\subfloat[Day 4 - \textbf{\textit{Global}}]{\includegraphics[width=0.32\textwidth]{Figures/day4revg2.png}\label{day4rev1g}}
	\hspace{0.1cm}
	\subfloat[Day 4 - \textbf{\textit{Local}}]{\includegraphics[width=0.32\textwidth]{Figures/day4revl2.png}\label{day4rev1l}}
	\hspace{0.1cm}
	\subfloat[Day 6 - \textbf{\textit{Sum}}]{\includegraphics[width=0.32\textwidth]{Figures/day6revs2.png}\label{day6rev1s}}
	\hspace{0.1cm}
	\subfloat[Day 6 - \textbf{\textit{Global}}]{\includegraphics[width=0.32\textwidth]{Figures/day6revg2.png}\label{day6rev1g}}
	\hspace{0.1cm}
	\subfloat[Day 6 - \textbf{\textit{Local}}]{\includegraphics[width=0.32\textwidth]{Figures/day6revl2.png}\label{day6rev1l}}
	\hspace{0.1cm}
	\subfloat[Day 9 - \textbf{\textit{Sum}}]{\includegraphics[width=0.32\textwidth]{Figures/day9revs2.png}\label{day9rev1s}}
	\hspace{0.1cm}
	\subfloat[Day 9 - \textbf{\textit{Global}}]{\includegraphics[width=0.32\textwidth]{Figures/day9revg2.png}\label{day9rev1g}}
	\hspace{0.1cm}
	\subfloat[Day 9 - \textbf{\textit{Local}}]{\includegraphics[width=0.32\textwidth]{Figures/day9revl2.png}\label{day9rev1l}}
	\vspace{0.0cm}
	\caption{Elba Island $\longrightarrow$ Porto Vecchio harbor - Day 4, 6, 9} 
	\label{addelb}
\end{figure}

\begin{figure}[H]
	\centering
	\subfloat[Day 1]{\includegraphics[width=0.32\textwidth]{Figures/day1impr39.png}\label{day1impr50}}
	\hspace{0.1cm}
	\subfloat[Day 4]{\includegraphics[width=0.32\textwidth]{Figures/day4impr50.png}\label{day4impr50}}
	\hspace{0.1cm}
	\subfloat[Day 4]{\includegraphics[width=0.32\textwidth]{Figures/day4impr39.png}\label{day4impr39}}
	\hspace{0.1cm}
	\subfloat[Day 5]{\includegraphics[width=0.32\textwidth]{Figures/day5impr50.png}\label{day5impr50}}
	\hspace{0.1cm}
	\subfloat[Day 6]{\includegraphics[width=0.32\textwidth]{Figures/day6impr50.png}\label{day6impr50}}
	\hspace{0.1cm}
	\subfloat[Day 6]{\includegraphics[width=0.32\textwidth]{Figures/day6impr39.png}\label{day6impr39}}
	\hspace{0.1cm}
	\subfloat[Day 10]{\includegraphics[width=0.32\textwidth]{Figures/day10impr50.png}\label{day10impr50}}
	\hspace{0.1cm}
	\subfloat[Day 10]{\includegraphics[width=0.32\textwidth]{Figures/day10impr39.png}\label{day10impr39}}
	\hspace{0.1cm}
	\subfloat[Day 12]{\includegraphics[width=0.32\textwidth]{Figures/day12impr50.png}\label{day12impr50}}
	\vspace{0.0cm}
	\captionsetup{font=footnotesize, labelfont=footnotesize}
	\caption{(b), (d), (e), (g), (i) Argentario $\longrightarrow$ Porto Vecchio harbor.
		     (a), (c), (f), (h) Elba Island $\longrightarrow$ Porto Vecchio harbor - \textbf{\textit{Improved method}}} 
	\label{addimpr}
\end{figure}

\chapter*{Ringraziamenti}
Arrivato a questo punto, mentre scrivo queste righe, mi rendo conto di aver ormai terminato il mio percorso universitario. Perciò non mi resta che ringraziare chi mi ha permesso di arrivare fin qui.\\
Ringrazio in primis Emanuele e Andrea di Witted per avermi dato l'opportunità di svolgere il progetto di tesi nella loro tech start up e il professor Fontanelli che mi ha fatto da relatore. Nadir, il primo ad interessarsi al mio lavoro e con cui ho avuto la fortuna di confrontarmi durante i primi mesi in azienda. Simone a cui, nonostante fosse preso dai suoi compiti, potevo sempre chiedere un aiuto. E infine Sergio, che mi ha supportato enormemente negli ultimi mesi di tesi.\\
Passando alla mia famiglia, ringrazio mia madre che ha avuto la pazienza di sopportare le mie frustrazioni incoraggiandomi sempre; mio padre, che seppur col suo fare un po' da so-tutto-io, non ha smesso di consigliare un testardo come me; e i miei fratelli su cui, nonostante non comunichiamo spesso, posso comunque contare.\\
Rimanendo all'interno dell'università vorrei ringraziare Riccardo, e ancora più Antonello, due menti superiori con cui ho condiviso lezioni infinite e periodi d'esame infernali e che mi hanno aiutato a stare al loro passo. Senza dimenticare Domenico, sempre disponibile per consigli, e Andrea Boiardi, correttore personale nel periodo di tesi.
Ringrazio poi Federico con cui ho potuto continuare gli allenamenti di karate, momento di sfogo e di svago in giornate sempre intense, ma anche per tutti i passaggi che mi ha offerto durante il periodo a Rovereto.\\
E infine i miei amici di Reggio su cui posso sempre contare nonostante i miei sporadici ritorni, e tutti quelli venuti oggi a condividere questo mio risultato.\\
Bisogna proprio dirlo: ``\'E FINITAAAAAAAA''

\end{document}
